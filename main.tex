% !TeX spellcheck = de_DE
\documentclass[a4paper,12pt]{book}

\usepackage{kpfonts}
\usepackage{fancyhdr}
\pagestyle{fancy}
\fancyhead{}
\fancyhead[RE]{\textsc{\nouppercase{\leftmark}}}
\fancyhead[LO]{\textsc{\nouppercase{\rightmark}}}

\usepackage[explicit]{titlesec}


% SECTIONS STYLE
\titleformat{\section}
{\normalfont\LARGE\bfseries}{\thesection.}{1em}{\textsc{#1}}
\titleformat{\subsection}
{\normalfont\Large\bfseries}{\thesubsection.}{1em}{\textsc{#1}}
\titleformat{\subsubsection}
{\normalfont\large\bfseries}{\thesubsubsection.}{1em}{\textsc{#1}}

% CHAP STYLE
\def\hrulefillthick{\leavevmode\leaders\hrule height3pt\hfill\kern0pt}
\titleformat{\chapter}[display]
{\normalfont\normalsize\scshape}
{\centering\hrulefillthick\hspace*{.5cm}\chaptertitlename\  \thechapter\hspace*{.5cm}\hrulefillthick}
{5pt}
{\titlerule\centering\huge#1}
[\titlerule]

\titleformat{name=\chapter,numberless}[display]
{\normalfont\normalsize\scshape}
{\centering\hrulefillthick\hspace*{0cm}}
{5pt}
{\titlerule\centering\huge#1}
[\titlerule]




\usepackage[utf8]{inputenc}
% no quoting para usar flechas en TIKZ!!!
\usepackage[spanish,es-nodecimaldot,,es-noquoting]{babel}


\usepackage{hyperref}
\hypersetup{
	colorlinks=true,
	menucolor=black,
	linkcolor=black,
	citecolor=black,
	urlcolor=blue}

\usepackage{apacite}
\bibliographystyle{apacite}

\usepackage{graphicx}
\usepackage{tabularx}
\usepackage{listings}
\usepackage{textcomp}
\usepackage{color}
\usepackage{tikz}
\usepackage{pgfplots}
\usepackage{qtree}
\usepackage{url} 
\usepackage{amsmath}
\usepackage{pdfpages}
\usepackage[font=small,labelfont=bf]{caption}

\usepackage{wrapfig}
\setlength{\columnsep}{20pt}

% No indent
\usepackage[parfill]{parskip}

%qtree captions
\usepackage{newfloat}
\DeclareFloatingEnvironment[fileext=lod]{diagram}

%exercises
\usepackage[answerdelayed,lastexercise]{exercise}

% code
\definecolor{codegreen}{rgb}{0,0.6,0}
\definecolor{codered}{rgb}{0.6,0,0}
\definecolor{codegray}{rgb}{0.5,0.5,0.5}
\definecolor{codepurple}{rgb}{0.58,0,0.82}
\definecolor{backcolour}{rgb}{1,1,1} 
\lstdefinestyle{mystyle}{
    backgroundcolor=\color{backcolour},   
    commentstyle=\color{codegreen},
    keywordstyle=\color{blue},
    numberstyle=\tiny\color{codegray},
    stringstyle=\color{codered},
    basicstyle=\footnotesize,
    breakatwhitespace=false,         
    breaklines=true,                 
    captionpos=b,                    
    keepspaces=true,                 
    numbers=left,                    
    numbersep=5pt,                  
    showspaces=false,                
    showstringspaces=false,
    showtabs=false,                  
    tabsize=2,
    basicstyle=\small\ttfamily,
    frame=tlbr,framesep=4pt,framerule=0.1pt
}
\lstset{style=mystyle}
\lstset{upquote=true}

% Graphics
\usetikzlibrary{arrows,positioning,shapes} 
\tikzset{
    %Define standard arrow tip
    >=stealth',
    %Define style for boxes
    box/.style={
           rectangle,
           rounded corners,
           draw=black, very thick,
           text width=6.5em,
           minimum height=2em,
           text centered},
    % Define arrow style
    arrow/.style={
           ->,
           thick,
           shorten <=2pt,
           shorten >=2pt,},
    rect/.style={
               rectangle,
               draw=black, very thick,
               text width=6.5em,
               minimum height=2em,
               text centered}
}
\usetikzlibrary{arrows,positioning,shapes} 
\tikzset{
    %Define standard arrow tip
    >=stealth',
    %Define style for boxes
    punkt/.style={
           rectangle,
           rounded corners,
           draw=black, very thick,
           text width=6.5em,
           minimum height=2em,
           text centered},
    % Define arrow style
    pil/.style={
           ->,
           thick,
           shorten <=2pt,
           shorten >=2pt,}
}


%\setcounter{secnumdepth}{3}
%\setcounter{tocdepth}{3}

\renewcommand{\lstlistingname}{Código} 


%TITLE
\newcommand*{\plogo}{\fbox{$\mathcal{PL}$}} % Generic publisher logo
%----------------------------------------------------------------------------------------
%	TITLE PAGE
%----------------------------------------------------------------------------------------

\newcommand*{\titleGP}{\begingroup % Create the command for including the title page in the document
	\centering % Center all text
	\vspace*{\baselineskip} % White space at the top of the page
	
	\rule{\textwidth}{1.6pt}\vspace*{-\baselineskip}\vspace*{2pt} % Thick horizontal line
	\rule{\textwidth}{0.4pt}\\[\baselineskip] % Thin horizontal line
	
	{\LARGE INTRODUCCIÓN PRÁCTICA\\ A \\[0.3\baselineskip] LAS CIENCIAS DE LA COMPUTACIÓN }\\[0.2\baselineskip] % Title
	
	\rule{\textwidth}{0.4pt}\vspace*{-\baselineskip}\vspace{3.2pt} % Thin horizontal line
	\rule{\textwidth}{1.6pt}\\[\baselineskip] % Thick horizontal line
	
	\scshape % Small caps
	Un viaje por el mundo de la computación \\ % Tagline(s) or further description
	acompañado de exploraciones apasionantes \\[\baselineskip] % Tagline(s) or further description
	Trabajo inicial 2015--2017. \\ Retomado en 2020.\par % Location and year
	
	\vspace*{2\baselineskip} % Whitespace between location/year and editors
	
	Escrito y editado por \\[\baselineskip]
	{\Large DANIEL CAÑIZARES CORRALES \\ \par} % Editor list
	{\itshape Cofundador de The Science of Code \\ Programador\par} % Editor affiliation
	
	\vfill % Whitespace between editor names and publisher logo
	
	
	{\scshape 2020} \\[0.3\baselineskip] % Year published
	{\large The Science of Code}\par % Publisher
	
	\endgroup}
%ENDTITLE



\begin{document}

\includepdf{./Images/cover.png}
\newpage
\thispagestyle{empty}

\frontmatter
\thispagestyle{empty}
\titleGP

%\thispagestyle{empty}
    \null\vspace{\stretch {1}}
        \begin{flushright}
                This is the Dedication.
        \end{flushright}
\vspace{\stretch{2}}\null
%\newenvironment{abstract}%
    {\cleardoublepage\thispagestyle{empty}\null\vfill\begin{center}%
    \bfseries\abstractname\end{center}}%
    {\vfill\null}
        \begin{abstract}
        This is the abstract.
        \end{abstract}
\thispagestyle{empty}

\chapter{Licencia}

Esta obra está bajo una licencia Creative Commons de Reconocimiento, No Comercial, Compartir Igual 4.0 Internacional. \\

 Usted puede compartir, copiar y distribuir el material en cualquier medio o formato, pero no explotar comercialmente esta obra. \\

Para más información consulte: \\ \url{http://creativecommons.org/licenses/by-nc-sa/4.0/}

\includegraphics{./Images/ccbyncsa.png}
\thispagestyle{empty}

\chapter{Prefacio}
    La mayoría de libros sobre programación disponibles en nuestro idioma, omiten muchos conceptos matemáticos y computacionales que son importantes para quien desee realizar una carrera como programador. Esta publicación pretende ser una alternativa rigurosa, pero al mismo tiempo práctica y apasionante, que llevará al lector en un viaje por el mundo de las \emph{Ciencias de la Computación} (CS, por sus siglas en inglés).    
    
    Esta metodología no pretende ser la mejor, ni la única, es solamente una alternativa que he desarrollado y mejorado durante algún tiempo. Puede interesarle a muchos estudiantes que no han tenido la oportunidad de acercarse a la programación (entendida más allá del simple desarrollo de software). Igualmente, este libro puede resultar interesante para profesores que quieran experimentar otras formas de enseñar; no puedo garantizarle que todos sus estudiantes aprenderán a programar, pero sí estoy seguro de que esté método es mucho más efectivo que el tradicional, sobre todo en la transmisión de buenas prácticas de programación y en el desarrollo de pensamiento computacional.
    
    Este trabajo fue inspirado por clases como \emph{CS-101 de Udacity} y el libro del profesor David Evans \emph{Introduction to Computing}. 
    
    La forma de trabajo que propongo confía en que usted se ponga manos a la obra ya que en vez de mostrar mil formas de escribir una instrucción, considero que es más apasionante sentarse a programar cosas de verdad.
    
  Finalmente, quiero invitarlo a que visite \url{http://thescienceofcode.com}, una comunidad abierta y dedicada a la enseñanza de las ciencias de la computación como arte liberal.

    
    \newpage
    \thispagestyle{empty}
        \textbf{Nota para profesores}: Empezar directamente con este libro puede ser difícil para un estudiante que no ha tenido ningún tipo de exposición previa a las Ciencias de la Computación, por eso recomiendo a los profesores que expongan a sus alumnos a programas como \emph{Code.org}, que explican de forma muy didáctica la mayoría de conceptos básicos de CS.
      
    
  
    
 
\tableofcontents

\newpage
\thispagestyle{empty}

\mainmatter

\chapter{Introducción}

\begin{quote}
\emph{Ya sea que quieras desvelar los secretos del universo, o quieras hacer una carrera [en ciencias de la computación], la programación básica de computadores es una habilidad esencial por aprender en el siglo XXI.} \\
Stephen Hawking.
\end{quote}

Los esfuerzos humanos por crear máquinas para realizar cómputos de manera automática datan de muchos siglos atrás, pero no fue sino hasta la \emph{Segunda Guerra Mundial} que se contó con el dinero y los componentes necesarios para construir máquinas programables \cite[p.~108]{evansIntro}. En un comienzo su uso fue meramente bélico, la tareas más comunes estaban relacionadas con cálculos de trayectorias para cohetes balísticos. Sin embargo, estás máquinas marcaron un punto de inflexión en la historia de la humanidad: por primera vez un mismo aparato servía para casi cualquier propósito (siempre y cuándo éste fuese \emph{computable}, discusión que abordaremos en el último capítulo de este libro).

Parece increíble, pero los computadores que tenemos en la actualidad están basados en el mismo modelo que planteó Alan Turing en la década de 1930. A grandes rasgos, solamente tenemos equipos más rápidos y con más memoria. Esa mayor capacidad de cómputo nos permite contar con programas cada vez más sofisticados, pero también con herramientas que hacen más fácil nuestra tarea como programadores.
 \newpage

\begin{wrapfigure}{r}{0.5\textwidth}
	\begin{center}
	\includegraphics[width=0.5\textwidth]{./Images/eniac.png}
	\end{center}	
	\caption{El ENIAC, uno de los primeros computadores de la historia. Foto bajo  Dominio Público.}
\end{wrapfigure}

En este sentido, la programación de los primeros computadores era bastante difícil y se realizaba básicamente conectando y desconectando cables. Conforme avanzó el tiempo y aumentó la capacidad de cómputo, aparecieron los \textbf{compiladores}. Básicamente lo que hacen es tomar un programa de computador escrito en un lenguaje que es más fácil de entender para los humanos (\emph{lenguaje de alto nivel}), lo traducen a un lenguaje que sea más cercano al que entiende la máquina (\emph{lenguaje de bajo nivel}), produciendo lo que conocemos como un \emph{ejecutable}. Al final del día, un programa termina siendo una secuencia de bits (unos y ceros) que representa una instrucción que puede ejecutar la máquina. Los primeros compiladores fueron desarrollados por la Almirante Grace Hopper en la década de 1950.

Existe otro tipo de programas que realizan una tarea similar, pero a diferencia de los compiladores que toman el código, lo traducen completamente, produciendo un programa de bajo nivel que puede ejecutar la máquina, los \textbf{intérpretes} realizan lo que podríamos llamar una \emph{traducción en simultáneo}. 

Cada enfoque tiene sus ventajas y desventajas: el código compilado es más rápido (gracias a la traducción previa que se realiza) pero en el código interpretado no es necesario que ejecutar una herramienta que compile el código con anterioridad de forma que es más fácil realizar cambios y ver el programa en ejecución.

El mecanismo utilizado para traducir los códigos a instrucciones que entiende la máquina es una forma en la cuál podemos clasificar los lenguajes de programación, sin embargo no es la única: la forma en que resolvemos problemas es una clasificación recurrente con la que nos encontraremos. A saber, existen dos grandes corrientes: la \textbf{programación imperativa} y la \textbf{programación funcional}. El enfoque de la primera es escribir instrucciones para modificar el estado de la memoria del computador, hasta eventualmente llegar a un estado que contenga la solución del problema. Por su parte, la programación funcional está relacionada con la aplicación de funciones que no manejan estado y que pueden unirse para resolver el problema. 

En la realidad nos encontraremos con programas que mezclan lenguajes compilados e interpretados. Por lo general, las partes más críticas de los sistemas son escritas en lenguajes compilados como \emph{C++} y son extendidas por lenguajes interpretados como \emph{Python} o \emph{Lua}. En cuanto a la programación imperativa, es la gran ganadora, la mayoría de sistemas son mayoritariamente imperativos, con pequeños trozos de programación funcional.

En este libro usaremos \emph{Python}, un lenguaje simple pero poderoso: interpretado, imperativo y con ciertos toques de programación funcional. Sin embargo, cabe anotar, que la mayoría de conceptos aquí expuestos pueden utilizarse en otros lenguajes, y que lo más importante no es la tecnología aplicada en un programa específico. La forma de aproximarse y construir soluciones computacionales, es lo que realmente debería importarle al lector.



\chapter{Nociones básicas}

En este capítulo se exponen la mayoría de conceptos involucrados en la programación de ordenadores a nivel básico. No se pretende que el lector domine todos los conceptos con sólo ver este capítulo, al contrario, la idea es dar un vistazo general sobre lo que utilizaremos y profundizaremos a lo largo del libro, de forma que, teniendo una imagen más completa de la programación los temas a tratar no parezcan obstáculos sin sentido, que sólo tienden a ralentizar el proceso de aprendizaje. 

Lo primero será comprender cómo las instrucciones de un programa operan directamente en la máquina, para esto introduciremos un modelo abstracto que representará nuestro computador: la Máquina de Acceso Aleatorio (RAM, por sus siglas en inglés).

\section{Máquina RAM}

La idea es mantener el modelo lo más simple posible, razón por la cuál sólo consideraremos los siguientes componentes:

\begin{itemize}
\item La memoria: representa el estado de la máquina en un momento del tiempo. Está dividida en múltiples secciones que llamamos posiciones de memoria, cada posición tiene asociados un número (generalmente un hexadecimal) que permite distinguir las posiciones entre sí y un valor, que representa el dato almacenado.

\item La entrada: son los datos sobre los cuáles se va a operar. Puede ser una pulsación del teclado, un clic, un dato que viaja por Internet, la información de un sensor, etc.
 
\item El programa: es una secuencia de instrucciones que le dice a la máquina cómo modificar su memoria de acuerdo a la entrada que recibe y al estado mismo de la memoria. 

\item La salida: es el resultado que se obtiene después de ejecutar total o parcialmente el programa.
\end{itemize}


\begin{figure}[h!]
\centering
  \begin{tikzpicture}
   \coordinate (programa) at (2,1);
   \coordinate (pa) at (0,2);   
   \coordinate (pb) at (2,2);
   \coordinate (pc) at (2,0);
   \coordinate (pd) at (0,0);   
  
   \coordinate (memoria) at (8,1);
   \coordinate (ma) at (8,2);   
   \coordinate (mb) at (10,2);
   \coordinate (mc) at (10,0);
   \coordinate (md) at (8,0);  
   \coordinate (mml) at (8,1.5); 
   \coordinate (mmr) at (10,1.5); 
     
  
  % programa
  \draw (pa) -- (pb) node[above, midway] {Programa};
  \draw (pb) -- (pc);
  \draw (pc) -- (pd);
  \draw (pd) -- (pa);   
  % lineas codigo
  \foreach \y in {0.2, 0.4,...,1.8}
  	\draw (0.2, \y) -- (1.8, \y);  
  
  %memoria
  \draw (ma) -- (mb) node[above, midway] {Memoria};
  \draw (mb) -- (mc);
  \draw (mc) -- (md);
  \draw (md) -- (ma); 
  %memset
  \draw (mml) -- (mmr); 
  % lineas mem
  \foreach \x in {8, 8.4,...,10}   	
	  \draw (\x,2) -- (\x,1.5);  
  
  	%ram
    \node [right=1cm of programa] [box] (ram) {RAM};  
  	\node [above=1cm of ram] (salida) {Salida};
  	\node [below=1cm of ram] (entrada) {Entrada}; 
  	
  	% arrows  	
  	\draw [arrow] (ram.north) -- node[] {} (salida.south); 	\draw [arrow] (entrada.north) -- node[] {} (ram.south);
 	\draw [arrow] (programa.east) -- node[] {} (ram.west);
 	\draw [arrow] (memoria.west) -- node[] {} (ram.east); 	
 	\draw [arrow] (ram.east) -- node[] {} (memoria.west);  
  \end{tikzpicture}

\caption{Máquina RAM}
\label{figram}
\end{figure}

De una u otra forma, el programa tendrá que interactuar con todos los componentes de la máquina, luego, los lenguajes de programación deben proveer mecanismos para manipular y representar la memoria, la entrada y la salida de datos. Cabe anotar que, como el programa depende del estado actual de la máquina, también se debería contar con estructuras que permitan repetir o ejecutar ciertas instrucciones de acuerdo a los datos almacenados en memoria. Por ejemplo, se deben ejecutar algunas instrucciones si la máquina entra o en un estado de error o se deben repetir algunas instrucciones para realizar una multiplicación (similar a como lo hacemos en papel).

Ahora, la pregunta es: ¿Cómo hace todo esto Python en un computador real? Después de muchos intentos con distintas explicaciones, creo que la mejor forma de responder esta pregunta es mediante ejemplos reales en el lenguaje de programación. Se recomienda encarecidamente al lector, que pruebe todos los programas en su máquina, para que los vea en funcionamiento. De igual forma, nada de esto tendrá sentido si el lector pasa rápidamente sobre este libro y no se toma el tiempo para entender qué hace cada instrucción.

\section{Programación en Python}

Aunque en un computador todo termina siendo unos y ceros, Python tiene algunas abstracciones básicas (otras un poco más avanzadas, que veremos en el capítulos posteriores) que harán todo más fácil. Una abstracción permite manipular la máquina sin necesidad de conocer a fondo su funcionamiento. Por ejemplo, el pedal de un carro permite que usted acelere sin necesidad de conocer el funcionamiento del motor. Si hablamos de la memoria, en Python contamos con diferentes \textbf{tipos de datos} que determinan las acciones podemos realizar con los datos almacenados en memoria, sin necesidad de conocer a profundidad qué hace el computador con ellos. A saber:

\begin{itemize}
\item Números: con los números podemos efectuar operaciones aritméticas como sumas (+), restas (-), multiplicaciones (*), divisiones(/), residuos(\%) y potencias (**). En Python, los enteros se llaman \textbf{int} (de \emph{integer}, en inglés) y los que tienen decimales se conocen como \textbf{float}. Por ejemplo: 0, -1, 1, 40, 52, son enteros y 0.1, 0.7, -1.8 son decimales.

\item Valores lógicos: en Python se conocen como \textbf{bool} (booleanos) y sólo tienen dos valores posibles: verdadero (\textbf{True}) o falso (\textbf{False}). Podemos aplicar los operadores Y (\textbf{and}), O(\textbf{or}), Negación (\textbf{not}), entre otros, para componer proposiciones más complejas.  Ver código \ref{cod-asignacion}.

\item Texto: se conocen como \textbf{strings} o secuencias de caracteres. Empiezan por una comilla doble o sencilla y terminan por el mismo símbolo, esto, con el fin de distinguir los textos arbitrarios definidos por el programador de las instrucciones propias del programa. Ver código \ref{cod-asignacion}.
\end{itemize}

Python no sólo facilita la representación de los datos en memoria, sino también su almacenamiento y manipulación. Para esto, existe el concepto de \textbf{variables}, que permiten dar un nombre arbitrario \footnote{Sólo hay tres restricciones: usar únicamente letras, números o guiones bajos, no empezar por un número y no dejar espacios en blanco} y un valor a partes de la memoria solamente utilizando el operador de igualdad (=), como se ve en el código \ref{cod-asignacion}. Python detecta automáticamente el tipo dato que se le asigna a una variable. \\

\newpage

\lstinputlisting[language=Python,caption=Asignaciones en Python,label=cod-asignacion]{./py/nociones/asignacion.py}

Si ejecutamos ese código, el computador no mostrará nada, eso es porque aún no hemos definido instrucciones para la salida de datos. La forma básica de mostrar información en la pantalla es usando la instrucción \emph{print}. Basta con escribirla y seguido de un espacio colocar el valor, la operación o la variable que se desea mostrar. En Python podemos comentar el código, añadiendo el caracter \#, de forma que podamos escribir aclaraciones del programa, sin afectar su comportamiento. Todo lo anterior se ve en el código \ref{cod-salida}. \\


\lstinputlisting[language=Python,caption=Salida de datos en Python,label=cod-salida]{./py/nociones/salida.py}

Por otro, la entrada de datos básica se hace con la instrucción \emph{input()}; entre los paréntesis se puede colocar un mensaje para aparezca a la persona que ejecuta el programa. Luego de ingresar los caracteres por teclado y presionar la tecla \emph{enter}, los datos quedan almacenados en la variable a la cuál se haya asignado la instrucción. Por ejemplo, el código \ref{cod-suma} muestra como solicitar dos números por teclado para luego imprimir el valor que resulta al sumarlos \footnote{Como lo importante es aprender, por ahora confiaremos en que el usuario no cometerá errores al ingresar los datos. He visto códigos que complican al estudiante con mil y una validaciones, bajo la excusa de que eso no se puede omitir en un programa comercial. Eso no tiene sentido, al fin y al cabo, cuando se realizan programas grandes, se usan otras herramientas que ayudan a prevenir estos errores.}. \\

\newpage

\lstinputlisting[language=Python,caption=Lectura y suma de dos números en Python,label=cod-suma]{./py/nociones/suma.py}

En Python también es posible condicionar la ejecución de ciertas instrucciones utilizando la expresión \emph{if}, que traduce \emph{si} en español. Dicha expresión, debe ir acompañada de una \emph{condición}, que no es más que la evaluación, \emph{verdadera} o \emph{falsa}, de una proposición utilizando los operadores lógicos: es igual (==) \footnote{Nótese que se usa doble igual para distinguir la comparación de la asignación.}, mayor ($>$), menor($<$), mayor o igual ($>=$), menor o igual ($<=$) o diferente (!=). Por ejemplo, el código \ref{cod-if} sólo ejecuta la instrucción que imprime \emph{Es cero}, si  el valor ingresado es igual a 0. Las instrucciones que están condicionadas \textbf{deben ir tabuladas}, como se observa en el código de ejemplo, y pueden ser una o más instrucciones de cualquier tipo. \\

\lstinputlisting[language=Python,caption=Un \emph{if} en Python,label=cod-if]{./py/nociones/if.py}


La instrucción \emph{if} puede ir acompañada de un \emph{else}, que traduce \emph{si no} en español, y que se utiliza para indicar qué instrucciones ejecutar cuando la condición no se cumple. Esta situación se puede observar en el código \ref{cod-ifelse}, en el cuál se reciben dos números y se imprime cuál de ellos es el mayor (en caso de ser iguales, no importará cuál mostremos). Nótese que el texto entre comillas se muestra tal cuál en patalla, mientras que el valor después de la coma se reemplaza por el valor que tenga la variable en el momento de la ejecución. \\

\lstinputlisting[language=Python,caption=Un \emph{if-else} en Python,label=cod-ifelse]{./py/nociones/ifelse.py}

Si se necesitan condicionar más caminos, se puede utilizar la expresión \emph{elif}, que es una abreviación de \emph{else if}. Por ejemplo, si quisiéramos saber cuál es el mayor de tres números (llámense $a$, $b$ y $c$), habría tres opciones: 
\begin{itemize}
\item Que $a$ sea mayor que $b$ y que $c$, en tal caso, $a$ sería el mayor.

\item Si no, se debe descartar entre $b$ y $c$ cuál es mayor, luego, si $b$ es mayor que $c$, $b$ es el mayor.

\item Si ninguna de las anteriores condiciones se cumple, es porque el mayor era $c$. 
\end{itemize}


Lo expresado anteriormente se observa en el código \ref{cod-ifelifelse}.  \\

\lstinputlisting[language=Python,caption=El mayor de tres números,label=cod-ifelifelse]{./py/nociones/ifelifelse.py}

En cuanto a la repetición de instrucciones, se utiliza la instrucción \emph{for}, que traduce \emph{para} en español. Veamos como funciona: \\

\lstinputlisting[language=Python,caption=Repetición de instrucciones en Python,label=cod-for]{./py/nociones/for.py}

El código anterior imprime los números desde el cero hasta el nueve (como puede ver, el límite superior no se incluye). Su lectura en español nos ayudará a entenderlo: \emph{para i en el rango de 0 a 10, imprima i}. La $i$ es una variable cualquiera, por lo cuál podemos utilizar otro nombre; es común usar $i$ por su relación con los índices y subíndices en matemáticas. Los dos números dentro del paréntesis de \emph{range} indican desde que valor inicia la repetición y en cuál termina. Al igual que con la estructura \emph{if}, las instrucciones que se repiten deben ir tabuladas.

Es posible mezclar todas las instrucciones que hemos visto (y las que veremos), por ejemplo, tener programas como el del código \ref{cod-forif} que recibe dos números, imprime todos los enteros que hay entre ellos, indicando si son negativos o positivos.


\lstinputlisting[language=Python,caption=Repeticiones y decisiones en Python,label=cod-forif]{./py/nociones/forif.py}

Hasta aquí hemos visto las instrucciones básicas usaremos para controlar todos los componentes de nuestra máquina RAM. Sin embargo en los ejemplos propuesto, no se modificó demasiado la memoria del equipo. A continuación hablaremos sobre algunos mecanismos propios de la programación imperativa que permiten resolver problemas mucho más complejos.



\section{Pensamiento imperativo}

Los algoritmos, al contrario de lo que parece, no son complicados para la mayoría de personas. Por ejemplo las multiplicaciones, divisiones, sumas y restas que hacemos en el papel son algoritmos que aprendimos por allá en la escuela primaria. Estamos en capacidad de enseñar a otras personas las instrucciones necesarias para realizar dichas operaciones; sin embargo, cuando se trata de explicarle a una máquina, las restricciones propias de memoria y procesamiento, hacen que el proceso sea un poco menos sencillo.

Muchas personas, sin saberlo, realizan pequeños programas de computador usando hojas de cálculo, determinando qué campos operar y dónde almacenar los resultados de dichas operaciones. La idea de los algoritmos que vamos a escribir es similar, la diferencia es que \emph{los datos no son estáticos y una posición de memoria cambiará muchas veces durante la ejecución del programa}, por lo cuál debemos abstraernos un poco más, de forma que podamos \emph{comprender el significado de nuestras instrucciones en el tiempo}.

Mediante unos problemas, exploraremos las soluciones típicas que se pueden dar a los problemas usando \emph{programación imperativa}. Como podrá ver el lector, en ocasiones es posible encontrar repuestas matemáticas más cortas y rápidas, pero en algunos casos no es posible este tipo de aproximaciones. Dicho esto, ¡Es hora de programar!

\subsection{Las sumas de Gauss}

\begin{minipage}{.7\textwidth} 
Cuando Gauss estaba en la escuela una profesora quería mantener ocupados a sus estudiantes, como no preparó clase, les puso la aburrida tarea de sumar los números enteros entre 1 y 100. La sorpresa de la maestra fue mayúscula cuando Gauss exclamó, después de muy poco tiempo: ¡5050! Efectivamente era la respuesta correcta. La pregunta es, ¿cómo lo hizo Gauss? A continuación construiremos la respuesta.
\end{minipage}
\begin{minipage}{.30\textwidth}
  \centering
  \includegraphics[height=4cm]{./Images/gauss.jpg}
\end{minipage}

\subsubsection{Solución imperativa}
Si Gauss hubiese tenido un computador a la mano, seguro habría escrito un programa en Python, cómo vimos, es posible repetir instrucciones de forma muy simple, utilizando la instrucción \emph{for}. Ahora, la pregunta es ¿cómo repetimos las instrucciones, de manera que al finalizar, el estado de la máquina tenga la respuesta a este problema?

La respuesta es corta, pero no muy intuitiva, en especial si esta es la primera vez que usted deja el mundo del papel y lápiz, por el de los bits. Así que, antes de continuar, es recomendable que se tome un par de minutos y escriba un programa de computador, con las instrucciones que usted cree, darán la respuesta. No importa si al final el código tiene errores, poner a trabajar las neuronas es más importante que llegar a la solución al primer intento. Una vez tenga su respuesta, continúe leyendo.
\\

Ya que no tenemos una hoja de papel para hacer un montón de operaciones, sólo posiciones de memoria que guardan bits (representadas en Python por el concepto de variables), necesitamos pensar muy bien como expresar las instrucciones. Hagámonos la siguiente pregunta, ¿será necesario almacenar en una variable diferente cada operación realizada? La respuesta es no. Imagine que usted tiene solo un pequeño trozo de papel, ¿cómo lo utilizaría para sumar los números del 1 al 100? Usted podría realizar operaciones y a medida que avanza ir borrando lo que ya no necesita guardarlas todas. Por ejemplo podría:

\begin{itemize}
\item Sumar 1 + 2 = 3. Como ya sabe la respuesta, puede borrar la operación y sólo conservar en su mente el número 3.
\item Luego, hay que sumar 3 + 3 = 6. De igual forma, sólo es necesario conservar el 6 presente, el resto se puede borrar.
\item La siguiente suma en el trozo de papel sería 6 + 4 = 10. Siguiendo el mismo órden de ideas, pasamos a la siguiente suma.
\item 10 + 5 = 15
\item 15 + 6 = 21
\item (...)
\item Y repetir, en el mismo pedazo de papel, hasta sumar los 100 números. 
\end{itemize}

En el computador la idea es la misma. No necesitamos 100 variables, usaremos una sola variable que iremos cambiando a medida que avance el tiempo, de forma que se acumule la suma. Ahora, si observamos cuidadosamente el proceso anterior, podemos encontrar que estamos sumando el valor que vamos acumulando, con cada uno de los 100 números. Luego, si damos el nombre de $suma$ a la variable en la que acumulamos el resultado, y le damos el nombre de $i$ a cada uno de los 100 número, la operación en el computador sería: 

\begin{equation}
suma = suma + i
\end{equation}

Puede revisar el código \ref{cod-for} para mayor claridad sobre el tratamiento que tendremos con la variable $i$.

Si metemos la expresión dentro de un \emph{for}, ¿funcionará? de hecho no. Notemos que el computador acumula haciendo $suma$ igual a lo que tenía más $i$. Y entonces, ¿qué tiene $suma$ la primera vez? Nada, pero hay que decírselo al computador iniciándola en cero. Al final, nuestro programa será como el que se observa en el código \ref{cod-gauss}. \\

\lstinputlisting[language=Python,caption=La suma de Gauss en Python,label=cod-gauss]{./py/nociones/gauss.py}

\subsubsection{Ejecución en la máquina RAM}

Vamos a ver cómo se modifica la memoria de la máquina hasta dar con la respuesta. La principal razón para no repetir este tipo de cosas es que \textbf{el programador debe aprender a leer el código por lo que significa y no por la forma en que la máquina lo ejecuta}; de otra manera nada ganaríamos con tener procesadores cada vez más rápidos, si a la hora de programar tenemos que repasar como ejecuta el computador las miles de iteraciones de nuestros programas. Por otro lado, el lector debe tener completamente claro lo que acabamos de programar, para que el comportamiento del resto de programas que veremos le resulte mucho más intuitivo.

Hecha esta aclaración, hagamos el seguimiento de la ejecución del programa:

\begin{itemize}
\item Se inicia la variable $suma$ en 0.
\item Se entra al ciclo en cuál la $i$ arranca en 1 y termina en 100.
	\begin{itemize}
	\item En la primera iteración $suma=0+1$, luego, $suma = 1$
	\item En la segunda iteración $suma=1+2$, luego, $suma = 3$
	\item En la tercera iteración $suma=3+3$, luego, $suma = 6$
	\item En la cuarta iteración $suma=6+4$, luego, $suma = 10$
	\item En la quinta iteración $suma=10+5$, luego, $suma = 15$
	\item ...
	\item En la penúltima iteración $suma=4851+99$, luego, $suma = 4950$
	\item En la última iteración $suma=4950+100$, luego, $suma = 50501$
	\end{itemize}
\item Cuando termina el ciclo se imprime el resultado $5050$
\end{itemize}

Como usted puede ver, el análisis previo es exactamente lo que hace el programa. Al principio puede resultar un poco difícil, pero usted tiene que ganar confianza como programador para poder enfrentar cada vez retos más difíciles y apasionantes.

\begin{quote}
\emph{La visualización (...) remarcaba precisamente aquello que el estudiante tiene que aprender a ignorar, reforzaba precisamente lo que el estudiante tiene que desaprender}.
Profesor Edsger Dijsktra, en su ensayo sobre la crueldad de enseñar realmente ciencias de la computación.
\end{quote}

\subsubsection{Solución matemática}

Aunque nuestro programa resultó funcionar muy bien, es claro que Gauss no pudo contar un computador en su época. Así que veamos cómo lo hizo.

Gauss descubrió una fórmula que permite hallar fácilmente la sumatoria de los $n$ primeros números:

\begin{equation}
1+2+3+\cdots+n = \frac{n(n+1)}{2}
\end{equation}

Si reemplazamos la $n$ por 100 tenemos lo siguiente:

\begin{equation}
1+2+3+\cdots+100 = \frac{100(100+1)}{2}
\end{equation}

Que efectivamente es 5050. ¿Pero cómo intuyó esto Gauss? La respuesta la encontramos si organizamos los números como se ve en la figura \ref{figgauss}.

\begin{figure}[h!]
\centering
\begin{tabular}{| c | c | c | c | c | c | c |}
\hline
1 & 2 & 3 & $\cdots$ & (n-2) & (n+1) & n \\ \hline
n & (n-1) & (n-2) & $\cdots$ & 3 & 2 & 1 \\ \hline
\end{tabular}
\caption{Suma de los primeros $n$ números}
\label{figgauss}
\end{figure}

Es decir, tomamos la serie de forma ascendente y luego de forma descendente. Ahora ¿qué pasa si sumamos verticalmente? En la primera obtenemos $n+1$, en la segunda $2+(n-1)=n+1$, y si seguimos, veremos que en todas obtenemos $n+1$. Como teníamos $n$ números, tenemos $n$ veces $n+1$ que es lo mismo que $n \cdot (n+1)$. Finalmente, la respuesta la dividimos entre dos, porque tomamos la serie dos veces, llegando a la respuesta \cite{graham1994concrete}.

La idea de sumar dos veces la serie resultó bastante útil, y nuestro programa se podría reducir a una línea -imprimir la operación. Las soluciones matemáticas suelen ser mucho más elegantes, rápidas y poderosas que las soluciones a código puro y duro, sin embargo no todo puede resolverse en el computador con una fórmula tan sencilla. Generalmente, se deben mezclar ambos enfoques para resolver problemas más complejos.



\subsubsection{Del papel al computador}

Es necesario aclarar, que para pasar la fórmula anterior al computador hay que tener especial cuidado en el orden en el que la máquina ejecuta las operaciones. A saber:

	\begin{enumerate}
	\item Se resuelve lo que esté dentro de paréntesis \textbf{()}.
	\item Se resuelven las potencias \textbf{**}.
	\item Se resuelven las multiplicaciones, divisiones y residuos  \textbf{* / \%}. 
	\item Se resuelven las sumas y restas  \textbf{+} y \textbf{-}.
	\end{enumerate}
	
Así que, para que el resultado sea el esperado, es necesario utilizar signos de agrupación como se ve en el código \ref{cod-gauss-formula}. \\


\lstinputlisting[language=Python,caption=Las sumas de Gauss.,label=cod-gauss-formula]{./py/nociones/gauss-formula.py}

\subsection{Los números primos}

Los números primos son muy importantes en la computación, ya que la criptografía actual esta basada principalmente en ellos. Sin embargo, los primos han fascinado a la gente desde tiempos inmemoriales. Por ejemplo, los griegos demostraron que existen infinitos números primeros \cite{discretemath}. Para entender qué son estos números, debemos definir el concepto de \emph{divisivilidad}.

\subsubsection{Divisibilidad}

Sean $a$ y $b$ dos enteros cualquiera. Decimos que $a$ es divisible entre $b$ si al realizar la división $a/b$ el residuo es cero.

Por ejemplo, 4 es divisible entre 2, porque al realizar la división 4/2 el residuo es cero; 5 no es divisible entre 2, porque al realizar la división 5/2 el residuo es uno.

Para hallar el residuo de una división en el computador, utilizamos el operador módulo (\%), como se observa en el código \ref{cod-mod}. \\

\lstinputlisting[language=Python,caption=Hallando el módulo o residuo de una división.,label=cod-mod]{./py/nociones/codmod.py}



\subsubsection{Números primos}

Un número entero $p$ mayor que uno, es  primo si solamente es divisible por $1$ y por $p$.  

Por ejemplo, 2 es primo porque sólo lo divisible entre 1 y 2, pero 4 no es primo, porque es divisible entre 1, 2 y 4. Los siguientes son los primeros 100 números primos:

2 3 5 7 11 13 17 19 23 29 31 37 41 43 47 53 59 61 67 71 73 79 83 89 97 101 103 107 109 113 127 131 137 139 149 151 157 163 167 173 179 181 191 193 197 199 211 223 227 229 233 239 241 251 257 263 269 271 277 281 283 293 307 311 313 317 331 337 347 349 353 359 367 373 379 383 389 397 401 409 419 421 431 433 439 443 449 457 461 463 467 479 487 491 499 503 509 521 523 541 

Conociendo los conceptos matemáticos, nuestra misión será crear un programa de computador que determine la cantidad de divisores de un número y posteriormente indique si es primo o no (en el próximo capítulo veremos cómo podemos hacer está última tarea más rápidamente).

\subsubsection{Solución}

La idea es simple: nos dan un número $n$, contamos cuántos divisores tiene, si al final tiene sólo dos divisores (1 y $n$) entonces es primo.

Si recordamos, todos los números son divisibles entre uno y entre el número mismo. Así que estas dos verificaciones son triviales y podemos omitirlas. Tenemos que centrarnos en determinar si algún otro número es divisor. Pero la pregunta es, ¿cuántos números diferentes debemos chequear? hay infinitos números, y nuestro programa no terminaría nunca de revisarlos todos. De hecho, en algún punto se bloquearía, porque la memoria de la máquina es finita. 

Luego, tenemos que determinar con cuidado qué números necesitamos verificar. Eso es algo fácil si lo pensamos así: ningún número es divisible por uno mayor que él. Por ejemplo, 3 no es divisible entre 5 (el 3 en el 5 está cero veces y sobran 3), dicho de otra forma, es imposible dividir el 3 en 5 partes enteras.

Dicho esto, bastará con \textbf{contar} qué números mayores que uno, pero menores que $n$, son divisores del número. Para contar en un programa de computador, se utiliza un mecanismo similar al del ejercicio anterior: definimos una variable, la iniciamos en algún valor (usualmente cero) y cada vez que se cumpla cierta condición le sumamos +1. Es como tener un papel en el que realizamos una raya cada vez que queremos contar algo.

Para nuestro caso, esa variable la llamaremos $divisores$, la iniciaremos en 2 (recuerde que todos los números tienen de entrada al uno y a él mismo como divisores) y cada vez que encontremos un divisor ejecutaremos la instrucción:
\begin{equation}
divisores = divisores + 1
\end{equation}

En el código \ref{cod-primos} puede ver el programa completo. Sólo falta agregar, que al final es necesario verificar que el número no sea uno, ya que por convención el uno no es primo. \\

\newpage

 \lstinputlisting[language=Python,caption=Divisibilidad y números primos.,label=cod-primos]{./py/nociones/primos.py}



\subsection{Las letras}


Supongamos que van a ingresarnos $n$ palabras. La idea es contar cuántas palabras consecutivas son iguales. Por ejemplo:

\begin{itemize}
	\item Si $n$ es 5, y letras, ingresadas en orden son:
	\item A
	\item A
	\item B
	\item A
	\item A
	\item La respuesta es dos. Porque se ingresaron dos As consecutivas.
\end{itemize}



\subsubsection{Solución}

Para este problema sólo basta tener en cuenta si la nueva letra es diferente al anterior, en tal caso se incrementa el número de grupos.

Para programarlo, debemos crear un espacio en memoria en el que podamos hacer seguimiento a la letra \emph{anterior}. Al terminar de leer cada letra y mirar si se forma un grupo, debe actualizarse el anterior, para efectos de tener los datos preparados para la siguiente iteración.

La única parte que es menos intuitiva, es la forma en que haremos que al poner al principio la variable \emph{anterior}. Todo esto se puede ver en el código \ref{cod-magnets}. \\

 \lstinputlisting[language=Python,caption=Solución del problema letras.,label=cod-magnets]{./py/nociones/letras.py}
 
 
\newpage
\section{Problemas}
En este capítulo hemos aprendido lo básico de la programación y de las soluciones imperativa: cómo sumar elementos, cómo contar elementos y vigilar un elemento anterior. \\

Mediante los problemas propuestos usted podrá auto-evaluarse y poner a prueba sus conocimientos. Para saber si su solución está correcta, piense en algunas entradas y las salidas respectivas a cada una de ellas, luego ejecute su programa y valide si el resultado fue correcto. Si no es así, piense ¿qué puede estar fallando?

\vfill

\begin{Exercise}[title={Calentamiento}]
Realice un programa en Python para resolver los siguientes problemas:
\begin{enumerate}
\item Se va a recibir un número, se garantiza que será 0 o 1. En caso de recibir un 0 se debe imprimir 1, en caso de recibir 1 se debe imprimir 0. No se puede utilizar la instrucción \emph{if}.

\item ¿Cómo realizar una multiplicación sin usar el signo por (*)? Encuentre dos soluciones diferentes para éste problema.

\item ¿Cómo realizar una potencia sin usar el signo de potenciación (**)?
\end{enumerate}
\end{Exercise}
\begin{Answer}

\lstinputlisting[language=Python,caption=Solución de los problemas de calentamiento.,label=cod-calentamiento]{./py/nociones/calentamiento.py}
\newpage
\end{Answer}
\vfill
\begin{Exercise}[title={La Copia de Seguridad}]
Bill quiere guardar sus documentos que pesan $x$ gigas en unos discos que tienen $q$ gigas de capacidad. Por ejemplo, si tiene que guardar 6 gigas en discos de 4 gigas, necesitará 2 discos (aunque el segundo lo llene sólo parcialmente).

¿Podrá usted ayudar a Bill construyendo un programa para que él sepa cuántos discos debe comprar? Lea por teclado las variables $x$ y $q$, para calcular la respuesta.

Nota: recuerde que los códigos deben funcionar para cualquier entrada, no sólo para las que aquí se proponen. Asuma de momento que siempre le van a dar datos válidos.

\end{Exercise}

\newpage
\begin{Answer}
 \lstinputlisting[language=Python,caption=Solución del problema La Copia de Seguridad.,label=cod-seguridad]{./py/nociones/copiaseguridad.py}
 
\end{Answer}

\begin{Exercise}[title={El Precio del Videojuego}]
	Joe Cortana tiene una tienda de empeños para videojugadores. Todos los días entran a su tienda cientos de vendedores que quieren cambiar sus antiguos videojuegos por dinero. 
	
	Joe es muy cuidadoso y ha estudiado muy bien como se comporta el mercado. Ha descubierto que en las mañanas llegan los mejores juegos, de esos clásicos que son muy apetecidos por los coleccionistas. A medida que pasan las horas, cada vez recibe juegos más malos, terminando en ET (considerado el peor juego de la historia).
	
	Joe Cortana tiene disponibles $x$ dólares todos los días. Su misión es decirle, dada una lista de $n$ videojuegos con sus precios (todo esto se debe leer por teclado), cuántos títulos alcanza a comprar, de forma que esté adquiriendo siempre lo mejor que vaya entrando a su tienda.	
	
	Por ejemplo: si tiene 10 dólares, y van a ofrecerle 3 juegos que valen respectivamente 6, 4 y 3 dólares, él compra dos juegos.
	
	\textbf{Explicación}: 
	\begin{itemize}
		\item El primero vale 6, y lo compra, quedándole 4 dólares disponibles.
		
		\item El segundo vale 4 dólares, de forma que lo compra y no le queda nada.
		
		\item El tercero vale sólo un dólar, pero ya no tiene dinero. 
		
		\item En definitiva, alcanza a comprar 2 juegos.
	\end{itemize}
	
	Nota: recuerde que los códigos deben funcionar para cualquier entrada, no sólo para las que aquí se proponen. Asuma de momento que siempre le van a dar datos válidos.
	
\end{Exercise}
\begin{Answer}
\lstinputlisting[language=Python,caption=Solución el Precio del Videojuego.,label=cod-vjprecio]{./py/nociones/vjprecio.py}
\newpage
\end{Answer}


\newpage

\chapter{Dividiendo los programas}

La mayoría de textos introductorios a la programación (por lo menos en español) esperan hasta el final para enseñar al estudiante cómo descomponer sus programas en pequeñas partes más fáciles de manejar. Una pena, ya que durante todo el curso se mal acostumbra al estudiante a tener todo dentro de un sólo algoritmo, no se enseña el uso correcto de los \emph{return} y se hace mucho más difícil resolver los problemas.

Dicho esto, veamos qué son los procedimientos.

\section{Procedimientos}

\begin{figure}[h]
\centering
\begin{tikzpicture}
  \node[punkt] (procedimiento) {Procedimiento};
  \node[above left=0.2cm and 1.3cm of procedimiento] (entradaA) {Entrada A};
  \node[below left=0.2cm and 1.3cm of procedimiento] (entradaB) {Entrada B};
  \node[right=1.3cm of procedimiento] (salida) {Salida};
  
  \draw [pil] (entradaA.east) -- node[] {} (procedimiento.west);
  \draw [pil] (entradaB.east) -- node[] {} (procedimiento.west);
  \draw [pil] (procedimiento) -- node[] {} (salida.west);
\end{tikzpicture}
\caption{Procedimiento con dos entradas y una salida}
\label{figproc}
\end{figure}

Los procedimientos o funciones (¡vamos que el tiempo de Pascal ya pasó!\footnote{Pascal es un lenguaje que era muy utilizado hace algunos años en el que se hacía una diferencia explícita entre procedimientos y funciones. Los lenguajes de programación modernos no hacen este tipo de distinciones.}) son formas de reutilizar los programas ya realizados para que no tengamos que andar reescribiéndolos. La idea es que el código siempre sea el mismo, lo que van a cambiar son las entradas. Es decir, sumar dos números siempre se hace igual, ¿qué cambia? los números que sumemos, pero el proceso siempre es el mismo.

En Python definimos un procedimiento así: \\

\begin{tabular}{l l}
def & \textbf{nombre\_procedimiento} (\textbf{entradas}): \\
 & \textbf{instrucciones} \\
 &	return \textbf{salida} \\
\end{tabular} \\

La palabra \emph{def} indica que se está definiendo un nombre para ese fragmento de código. Las entradas pueden ser únicas, múltiples o inclusive ninguna (en caso de presentarse varias entradas, éstas deben estar separadas por comas). La palabra \textbf{return} indica la salida del procedimiento, así que una vez el computador llegue a un \textbf{return}, deja de ejecutar el resto del código del procedimiento, comportamiento que nos resultará sumamente útil más adelante. 

\emph{Importante:} no todos los procedimientos retornan valores, por ejemplo, la función \emph{print} es un procedimiento que no retorna ningún resultado, sólo muestra valores en la pantalla. Inicialmente, nuestros procedimientos tendrán salidas, pero más adelante habrá casos en los cuáles no sea necesario retornar nada.


Por otra parte, es importante recordar que \textbf{las tabulaciones} son fundamentales en Python, ya que el código que está tabulado (como se ve en la sintaxis) es el que hará parte del procedimiento.


\begin{figure}[h]
\centering
\begin{tikzpicture}
  \node[punkt] (procedimiento) {Sumar};
  \node[above left=0.2cm and 1.3cm of procedimiento] (entradaA) {Número A};
  \node[below left=0.2cm and 1.3cm of procedimiento] (entradaB) {Número B};
  \node[right=1.3cm of procedimiento] (salida) {Resultado};
  
  \draw [pil] (entradaA.east) -- node[] {} (procedimiento.west);
  \draw [pil] (entradaB.east) -- node[] {} (procedimiento.west);
  \draw [pil] (procedimiento) -- node[] {} (salida.west);
\end{tikzpicture}
\caption{Procedimiento que recibe dos números y los suma}
\label{figprocsuma}
\end{figure}

Por ejemplo, la figura \ref{figprocsuma} muestra gráficamente un procedimiento que suma dos números y arroja el resultado (este ejemplo es sólo para ilustrar al lector sobre el uso de procedimientos, más adelante en este capítulo veremos su verdadera utilidad, de momento lo mantendremos así de simple). Dicho procedimiento, está implementado en el código \ref{codsuma1}. \\

\newpage

\lstinputlisting[language=Python,caption=Procedimiento para sumar dos números en Python,label=codsuma1]{./py/cap3/suma.py}


\subsection{Invocar procedimientos}

Para ejecutar el procedimiento que hemos definido, bastará con escribir su nombre y entre paréntesis enviar los valores que se le asignan a cada una de las entradas. Si el procedimiento retorna un valor este puede imprimirse directamente o guardarse en una variable -de acuerdo a lo que necesitemos. El código \ref{codinv1} ilustra cómo invocar el procedimiento que hicimos anteriormente. \\

\lstinputlisting[language=Python,caption=Llamado al procedimiento,label=codinv1]{./py/cap3/suma-inv1.py}

El Python, el código completo debe incluir el procedimiento, como se ve en el código \ref{codinv2}. \\s

\lstinputlisting[language=Python,caption=Programa completo,label=codinv2]{./py/cap3/suma-inv2.py}

\subsection{Ejecución del programa}

Revisemos cómo se ejecuta el programa. Primero, notemos que el código está ``dividido'' en dos partes:

\begin{enumerate}
\item La sección dónde declaramos los procedimientos. Ésta siempre debe aparecer al principio del programa. En este caso son las líneas 1 y del código \ref{codinv2}

\item El resto del programa. Esta parte se conoce como \textbf{algoritmo principal}. Note que es la sección que \emph{no está contenida dentro de un def}.
\end{enumerate}

Cuando se inicia el programa, la ejecución comienza por el \textbf{algoritmo principal} y avanza línea a línea. Si se realiza una llamada a un procedimiento, éste se ejecuta, una vez termine, se continúa desde punto en el que se realizó la llamada.

En este orden de ideas la ejecución del programa \ref{codinv2} es la siguiente:

\begin{enumerate}
\item Se inicia la ejecución desde el \textbf{algoritmo principal} en la línea 4.
\item Se llama al procedimiento sumar con los valores 3 y 5, como entrada.
\item Se ejecuta la única línea que tiene el procedimiento (línea 2), retornando la suma de los dos valores. Termina el procedimiento.
\item Se continúa la ejecución en la línea 4, asignando el valor que retornó el procedimiento a la variable $resultado$.
\item Se ejecuta la línea 5 y se imprime el número que estaba almacenado en la variable resultado.
\end{enumerate}

\newpage
\subsection{Regresando con los números primos}

¿Recuerda los números primos? En está ocasión vamos a revisar el programa que habíamos realizado en el capítulo anterior, intentando eliminar algunas validaciones que son innecesarias y descartando algunos números que sabemos con anterioridad, no son primos. El objetivo es optimizar el programa (en un capítulo posterior veremos cómo se mide la eficiencia de un algoritmo) y aprender a utilizar retornos para ahorrarnos tabulaciones (en algunos casos útiles).

Tengamos en cuenta lo siguiente:

\begin{itemize}
	\item Ni el uno ni el cero son primo.
	\item Ningún número par es primo, excepto el dos. 
	\item Si ninguno de los casos anteriores se cumple, sólo debemos verificar desde 3 hasta $\sqrt{n}$. Si en la verificación encontramos que algún número es divisor de $n$, sabemos que $n$ no es primo.
\end{itemize}

Primero revimos el código que implementa todo esto sin usar procedimientos, teniendo en cuenta que imprimiremos TRUE si es primo y FALSE si no lo es. Para hallar la raíz cuadrada en Python se usa la función \emph{sqrt(número)} en el código \ref{codprimos1}. \\

\lstinputlisting[language=Python,caption=Sólución sin procedimientos,label=codprimos1]{./py/cap3/primos1.py}

\newpage

Notemos una cosa de la última parte del código: tenemos que asignar una variable asumiendo que sí es primo, luego revisar los valores y si encontramos un divisor convertimos esa variable en falso, pero el programa sigue ejecutándose hasta la $\sqrt{n}$, sin importar que ya sepamos que no es primo. \\

Dicho esto, miremos cómo quedaría una solución utilizando procedimientos y retornos de forma inteligente (Ver código \ref{codprimos2}). \\

\lstinputlisting[language=Python,caption=Sólución usando procedimientos,label=codprimos2]{./py/cap3/primos2.py}

Con esta solución nos ahorramos la tabulación del último \emph{else}, ya que si se llega a un retorno el resto del procedimiento no se ejecuta. De igual forma la última parte es más clara y no necesita que computemos los valores hasta $\sqrt{n}$, porque al encontrar un divisor del número, el procedimiento retorna \emph{False} como respuesta. Si el programa nunca encuentra un divisor, alcanza el último retorno y devuelve \emph{True}.

Si quisiéramos usar el procedimiento, bastaría con llamarlo las veces que sea necesario, como se ve en el código \ref{codprimoscall}.

\newpage

\lstinputlisting[language=Python,caption=Llamada al procedimiento para saber si un número es primo,label=codprimoscall]{./py/cap3/primoscall.py}


\section{Problemas}
\begin{Exercise}[title={Calentamiento}]	
	\begin{enumerate}
		\item Realice un procedimiento en Python que reciba dos números ($a$, $b$) y los multiplque sin usar el signo por (*).
		
		\item Cree un procedimiento que reciba dos números ($a$, $b$) y realice la operación $a^b$ sin usar el signo de potenciación (**) y \emph{usando el procedimiento realizado en el paso anterior}.
		
		\item Cree un algoritmo principal que lea dos números, los envíe al procedimiento anterior e imprima la respuesta.
	\end{enumerate}
\end{Exercise}
\begin{Answer}
	
	\lstinputlisting[language=Python,caption=Solución de los problemas de calentamiento de procedimientos.,label=cod-calentamiento-proc]{./py/cap3/calentamiento.py}
	\newpage
\end{Answer}
\newpage
\begin{Exercise}[title={Patrullas de policía}]	
	En una estación de policía se tienen unas patrullas para atender los incidentes
	que ocurren en el sector. De cada patrulla sabemos si está disponible o no, cuántos
	agentes tiene asignados, si es motorizada y si los agentes tienen bolillos o armas de
	fuego. Cada cierto tiempo se reciben llamados de emergencia, de estos conocemos
	cuantos agentes se necesitan para controlar la situación, si se requieren patrullas con
	vehículos y si es posible enfrentarse con sospechosos armados.
	
	\begin{enumerate}
		\item Construya un procedimiento que reciba todos los datos necesarios para
		determinar si una patrulla puede atender una emergencia.
		
		\begin{itemize}
			\item Primero realice el procedimiento verificando si una patrulla CUMPLE con los requisitos, uno a uno valídelos.
			
			\item Programe de nuevo el procedimiento pero ahora hágalo DESCARTANDO las patrullas que no cumplen.
			
		\end{itemize}
		
		\item Construya un programa que solicite a un usuario los datos de una
		emergencia, luego pida los datos de 10 patrullas, con cada una
		debe llamar el procedimiento creado en el punto anterior e indicar si la patrulla puede o no atender la emergencia.
	\end{enumerate}
\end{Exercise}
\begin{Answer}
	
	\lstinputlisting[language=Python,caption=Solución del problema Patrullas de Policía.,label=cod-policia]{./py/cap3/policia.py}
	\newpage
\end{Answer}



\chapter{Programas más complejos}

En este capítulo veremos un par de herramientas adicionales, para finalmente poner en práctica todo lo aprendido en la programación de un juego en Python. La idea es también construir una estrategia que le permita al ordenador jugar contra una persona.

\section{Otras formas de repetir}


Utilizando un \emph{for} podemos iterar un número de veces determinado (de acuerdo al rango que definamos), sin embargo en ocasiones no sabemos de antemano cuántas veces debemos repetir, sólo sabemos que debemos hacerlo mientras se cumpla una condición en el estado de la máquina. En esos casos utilizamos el ciclo \textbf{while}, que traduce \emph{mientras} en español. El \textbf{while} tiene la siguiente sintaxis:

\begin{tabular}{l l}
	while & \textbf{condición}: \\
	& \textbf{instrucciones a repetir} \\
\end{tabular} \\

Por ejemplo, podemos tener el mismo comportamiento que un \emph{for}, como se ve en código \ref{codwhilefor} (aunque es mejor utilizar un \emph{for} para estos casos, por obvias razones). 

\newpage

\lstinputlisting[language=Python,caption=While vs For,label=codwhilefor]{./py/cap3/whilefor.py}

A continuación veremos un caso en el que sí se debe usar el \emph{while}.

\subsection{El algoritmo de Euclides}

El algoritmo de Euclides permite computar el \textbf{Máximo Común Divisor} (MCD) entre dos enteros positivos. El MCD está definido como el entero positivo más grande que divide a ambos números \cite{discretemath}. \\

Sean $a$ y $b$ dos enteros positivos, para hallar su MCD esto es lo que debemos realizar: 

\begin{enumerate}
	\item Si el residuo de la división entre $a$ y $b$ es cero, entonces $b$ es el MCD.
	\item Si el residuo de la división entre $a$ y $b$ es un entero $r$ diferente de 0, entonces el MCD entre $a$ y $b$ es el MCD entre $b$ y $r$.
\end{enumerate}



Por ejemplo, para hallar el MCD entre 18 y 12:

\begin{itemize}
	\item El residuo entre 18 y 12 es 6, luego toca aplicar la segunda propiedad.
	\item Al aplicarla, operamos con 12 y 6, cuyo residuo es 0. Luego, el MCD(18,12) es 6. 	
\end{itemize}

Ahora, vamos a programar el algoritmo. Tendremos en cuenta las propiedades, pero haremos un pequeño ajuste para que nos sea más fácil programarlo: iteraremos una vez de más, de forma que el residuo quede cargado en la variable $b$, como se puede ver en la tabla \ref{euclidej}. Luego, la condición para iterar es que $b$ sea diferente de cero, cuando llegue a cero es porque hemos encontrado en MCD (que estará en la variable $a$).

\newpage

\begin{figure}[h!]
	\centering
\begin{tabular}{| p{1cm} | p{1cm} | p{1cm} |}\hline
	\textbf{a} & 	\textbf{b} & 	\textbf{r} \\ \hline
	18 & 12 & 6 \\ 
	12 & 6 & 0 \\ 
	6 & 0 & FIN \\ \hline
\end{tabular}

\caption{Aplicando el algoritmo de Euclides para 18 y 12}
\label{euclidej}
\end{figure}

Para que los valores sobre los que operamos cambien en cada iteración, basta con intercambiar el valor que hay en $a$ por el de $b$ y el de $b$ por el del residuo que acabamos de calcular. Finalmente, el código \ref{codeuclid} expone lo que acabamos de describir. \\

\lstinputlisting[language=Python,caption=El algoritmo de Euclides en Python,label=codeuclid]{./py/cap3/euclid.py}

¿Por qué no usamos un \emph{for}? Porque no podemos saber de antemano cuántas veces vamos a repetir, dependiendo del valor que arroje el residuo debemos seguir repitiendo los pasos del algoritmo, o por el contrario sabemos si hemos encontrado la respuesta correcta.

\subsection{Repitiendo por siempre}

Una de las formas más comunes de utilizar el \emph{while} es poniendo un \textbf{True} en la condición, de forma que se repita por siempre. El \textbf{While True} requiere que se utilice un retorno (en caso de estar en un procedimiento) o la instrucción \textbf{break} (que traduce romper), para indicarle al programa cuándo salirse del ciclo y continuar ejecutando el resto del programa. Tiene la siguiente sintaxis: \\

\begin{tabular}{l l}
	while & True:  \\
	& \textbf{instrucciones a repetir}  \\
	& if \textbf{condición de salida}:  \\
	&    \hspace{1cm} break \\
\end{tabular} \\

Veámos cómo funciona esto con un ejemplo. Suponga que necestiamos realizar un programa que obligue a la persona a digitar un número positivo, es posible realizarlo de dos formas: usando un while o usando un while true, como se puede observar en el código \ref{codwhiletrue}. \\

\lstinputlisting[language=Python,caption=While True VS While,label=codwhiletrue]{./py/cap3/whiletrue.py}

Aunque ambos cumplen con la misma función, con el \emph{While True} se evita la asignación de un valor temporal para entrar en el ciclo, el cuál nada tiene que ver con el objetivo real del programa.

\section{Depurando los programas}

La mayoría de las veces, los programas que construimos tienen errores lógicos o \emph{bugs}. Para eliminarlos y corregir el programa es necesario identificarlos, algo que en ocasiones es una tarea nada trivial. \\

El proceso de arreglar programas con errores se conoce como \emph{depuración} \cite{evansIntro}. Las siguientes son recomendaciones útiles que usted puede seguir para corregir sus programas:

\begin{enumerate}
	\item No le pida ayuda a su profesor o a un programador. Puede sonar extraño, pero cuando usted se acostumbra a que otros corrijan sus programas se hace un daño terrible: usted aprende más corrigiendo un programa que no funciona, que programando un código que falla. 
	
	\item Aprenda a utilizar el \emph{depurador} que la mayoría de entornos de programación contienen, por lo menos aprenda a poner \emph{prints} en el código para identificar dónde está el error. Esto es, use \emph{prints} para imprimir los valores que tienen las variables que representan el estado de la máquina, revise si estos valores son correctos, si no, piense qué podría estar ocasionando el error.
	
	\item Identifique el procedimiento que está produciendo el error. Si tiene todo el programa en un solo \emph{mega-algoritmo}, piense mejor en empezar de cero.
\end{enumerate}

Algunas sugerencias que hace David Evans en su libro Introduction to Computing \cite{evansIntro}, son:

\begin{enumerate}
	\item Asegúrese de entender el comportamiento para el que está destinado su procedimiento. Piense en algunas entradas representativas y las salidas esperadas para esas entradas.
	
	\item Haga experimentos, observe el comportamiento actual de su procedimiento. Pruebe el programa con entradas pequeñas. ¿Cuál es la relación entre las salidas actuales y las esperadas? ¿Funciona correctamente para algunas entradas pero no para otras? 

	\item Haga cambios en su procedimiento y vuelva a probarlo. Si usted no está seguro de qué hacer, haga cambios en pequeños pasos y cuidadosamente observe el impacto de cada cambio.
\end{enumerate}

Dicho todo lo anterior, empecemos a ver algunas cosas aplicaciones interesantes para todo lo aprendido.

\newpage

\section{Aplicando lo aprendido}

Vamos a empezar programando un juego corto. El primer paso será dividir el problema en varias partes, ya que es más fácil solucionar pequeños pedazos del juego que intentar resolverlo todo al mismo tiempo. Al final, ensamblaremos los fragmentos de código resultantes, de forma que podamos construir un programa completamente funcional.

\subsection{Thai 21}

Thai 21 es una variante del juego Nim. Distintas versiones de este juego han sido practicadas a lo largo de muchos siglos, pero nadie sabe realmente de donde proviene el nombre \cite{evansIntro}.

El juego consiste en una pila que contiene 21 rocas. Dos jugadores se enfrentan tomando una, dos o tres en cada turno. Al final gana quién tome la última de las piedras. \\

Como dijimos anteriormente, empezaremos por separar el juego en varias partes, esta es la propuesta:

\begin{enumerate}
	\item Definir cómo representar los datos: responder a esta pregunta ¿cómo representar cada uno de los elementos del juego en Python?
	\item Lectura de datos: tenemos que pedirle a los jugadores cuántas rocas tomar, pero deben cumplirse dos condiciones:
	\begin{itemize}
		\item Que haya suficientes rocas.
		\item Que intente tomar una, dos o tres rocas.
	\end{itemize}
	
	\item Definir un estado inicial: determinar qué jugador empieza y configurar la pila de las rocas.
	
	\item El ciclo del juego (gameloop): pedir los datos (usando el código que realicemos en el paso 2), quitar las rocas de la pila, ver si alguien gano o si se cambia de jugador.
\end{enumerate}

\subsubsection{Representación de los datos}
Vamos a tener dos elementos en el juego: las rocas y los jugadores. Los representaremos así en el programa:

\begin{itemize}
	\item Para las rocas podemos utilizar un entero que nos indique cuántas piedras siguen en el juego.
	
	\item Para los jugadores, simplemente necesitamos saber si es el turno del jugador 1 o del jugador 2, para lo cuál usaremos un \emph{bool}, que represente con True al jugador 1 y con False al jugador 2.
\end{itemize}


\subsubsection{Lectura de los datos}
Definiremos un procedimiento, que luego pueda ser llamado por el algoritmo principal, en el cuál sólo retornaremos el número de rocas que elija el jugador si y sólo si, dicho número es un valor válido para el estado actual del juego.

El procedimiento necesita recibir como entrada el número de rocas disponibles, para determinar que el jugador no está intentando tomar más de las que hay en la pila. 

Dicho procedimiento lo llamaremos \emph{pedir\textunderscore jugada}, como se ve en el código \ref{codthailectura}. \\

\lstinputlisting[language=Python,caption=Lectura de datos para Thai 21,label=codthailectura]{./py/thai21/lectura.py}

\subsubsection{Estado Inicial}
Lo pondremos en el algoritmo principal. La idea es indicarle al computador que empieza el jugador 1 que hay 21 rocas en la pila, como se ve en el código \ref{codthaiinic}. \\

\lstinputlisting[language=Python,caption=Inicialización del programa para Thai21,label=codthaiinic]{./py/thai21/inic.py}

\subsubsection{Gameloop}
Cómo no sabemos cuántos turnos son, el gameloop será un \emph{While True}, que terminará cuando no queden rocas. 

Los pasos a realizar serán lo que especificamos anteriormente: pedir los datos (usando el procedimiento que ya definimos), quitar las rocas de la pila (una resta), ver si alguien ganó (si no quedan rocas) o si se cambia de jugador (como es jugador actual está representado por un \emph{bool}, bastará con hacer una negación: si está en verdadero pasará a falso, o viceversa). El gameloop se puede ver en el código \ref{codthaigameloop}. \\

\lstinputlisting[language=Python,caption=Gameloop del Thai21 en Python,label=codthaigameloop]{./py/thai21/gameloop.py}

\subsubsection{Programa final}

Ya que cada parte está lista, podemos ver el programa completo en el código \ref{codthai}. En dicho código se añadió un print para mostrar cuántas rocas quedan en la pila. \\

\lstinputlisting[language=Python,caption=Thai21 en Python,label=codthai]{./py/thai21/thai.py}

\newpage

Para terminar este capítulo, construiremos una estrategia de victoria para que el computador juegue contra una persona. La idea es la siguiente: ¿es posible crear una estrategia que permita al computador ganar cualquier partida sin importar qué movimientos realice el otro jugador? Asuma que siempre el computador hará la primera jugada. 

\textbf{Piense unos minutos cómo podría ser dicha estrategia, antes de continuar leyendo}.

\newpage

\begin{figure}[h!]
	\centering
	\includegraphics[width=3cm]{./Images/thai21.png}
	\caption{Posición de victoria para el jugador 1 (rojo).}
	\label{figthaiA}
\end{figure}

\subsection{Estrategia de victoria}
La idea es llegar a un punto en dónde, sin importar qué haga, el segundo jugador no pueda ganar. Por ejemplo, si quedan 4 piedras, no importa si el jugador 2 toma una, dos o tres piedras, en cualquier caso, el jugador 1 puede ganar (ver figura \ref{figthaiA}).

\begin{itemize}
\item Si el jugador 2 toma una piedra, quedan tres piedras. El jugador 1 toma las tres piedras y gana.
\item Si el jugador 2 toma dos piedra, quedan dos piedras. El jugador 1 toma las dos piedras y gana.
\item Si el jugador 2 toma tres piedra, queda una piedras. El jugador 1 toma la única piedra y gana.
\end{itemize}

Dicho esto, tenemos que obligar al jugador 2 a que caiga en cuatro, ¿desde cuáles se puede llegar a cuatro? Si es el turno del jugador 1, puede hacerlo desde cinco, seis y siete. Luego, el jugador 2 debe quedar en ocho, ya que sin importar qué haga, quedaremos en alguno de dichos números, como se puede observar en la figura \ref{figthaiB}. 


\begin{figure}[h!]
	\centering
	\includegraphics[width=6cm]{./Images/thai21b.png}	
	\caption{Otra posición clave en el juego.}
	\label{figthaiB}	
\end{figure}

Siguiendo la misma idea, podemos devolvernos. A 8 se llega desde 9, 10 y 11. Luego el jugador 2 debe quedar en 12, pero antes en 16 y antes en 20. \\

Si usted observa bien los números clave son múltiplos de 4 (20, 16, 12, 8 y 4), de forma que el número de rocas a tomar, en cada turno debe ser:

\begin{equation}
JugadaPC = Rocas \% 4
\end{equation}

Así, si al empezar hay 21 rocas: $21\%4=1$. Tomamos una roca y el otro queda en una posición desde la cuál sin importa qué haga, podemos obligarlo, con la misma fórmula, a que caiga en 16, luego en 12, más tarde en 8 y finalmente en 4.

Para adaptar el código que tenemos hay que modificar el procedimiento que pide la jugada (ya que uno de los jugadores va a ser el computador). Vamos a recibir otro parámetro que será el jugador actual (obviamente habrá que enviarlo desde el algoritmo principal).

El resultado final puede verse en el código \ref{codthaifull}.
\newpage
\lstinputlisting[language=Python,caption=Thai21 humano VS computador,label=codthaifull]{./py/thai21/thaifull.py}
\newpage

\section{Problemas}
\begin{Exercise}[title={Calentamiento}]	
	\begin{enumerate}
		\item Entre a \url{http://codecombat.com/} para practicar sus habilidades como programador (y aprender cosas nuevas).
		
		\item Modifique el último código de Thai21 para que empiece un jugador al azar (no siempre el computador). Es posible que en ocasiones no pueda aplicar la estrategia de victoria. \\
		
		Cuando necesite generar valores al azar:
		
		\begin{enumerate}
			\item Ponga en la primera línea del programa: \textbf{import random}. Esto le permitirá usar la librería de Python para generar números \emph{Pseudo-aleatorios}.
			
			\item Utilice la función:  \textbf{random.randint(a,b)}, para generar números al azar (el número generado, llamémoslo $X$ estará en el rango $a \leq X \leq b$). 
			
			\item Ejemplo:
			\lstinputlisting[language=Python,caption=Números aleatorios en Python,label=codrandomn]{./py/thai21/randomn.py}
		\end{enumerate}
		
		Otra función que puede serle útil es \textbf{min(a, b)} que recibe dos números y le retorna el menor de los dos (para evitar escribir el if y el else, como en el segundo capítulo).
	\end{enumerate}
\end{Exercise}
\begin{Answer}	
	\lstinputlisting[language=Python,caption=Thai21 definitivo.,label=codthai21def]{./py/thai21/thaicalentamiento.py}
	\newpage
\end{Answer}

\newpage

\begin{Exercise}[title={El juego del granjero}]	
	Hace mucho tiempo, en un mundo alterno, un granjero fue al mercado del pueblo más cercano y compró: un lobo, una cabra y una col.
	
	Para volver a su casa tenía que cruzar un río. El granjero disponía de una barca para llegar a la otra orilla, pero en la barca solo cabían él y una de sus compras. Por supuesto, tenía ciertas restricciones:
	\begin{itemize}
		\item Si el lobo se quedaba solo con la cabra se la comía.
		\item Si la cabra se quedaba sola con la col se la comía.
		\item El granjero siempre debía ir en el bote (para manejarlo).
	\end{itemize}

	Sabemos que al final del día el granjero logro cruzar con todas sus compras, la pregunta es ¿cómo lo hizo?
	
	Elabore un programa en Python que permita a un jugador escoger quién irá en cada
	viaje con el granjero y que valide de acuerdo a las reglas si en algún momento algo va mal o si se llega al final del juego.
	
	Imagine cómo podría pintar el escenario del juego usando prints.
	 
\end{Exercise}
\begin{Answer}	
	\lstinputlisting[language=Python,caption=El juego del granjero,label=codlccol]{./py/soluciones/lobocabracol.py}
	\newpage
\end{Answer}

\chapter{Estructuras de datos básicas}

Las estructuras de datos son abstracciones que permiten trabajar datos que tienen algún orden o jerarquía.


\section{Ciclos anidados}
\section{Strings}
\section{Listas}
\section{Listas de listas}
\section{Usando las estructuras}
\subsection{Recorridos}
\subsection{Aliasing y mutabilidad}
\subsection{Funciones útiles}
\section{Aplicaciones}
\subsection{El primer parser}
\subsection{El juego del ahorcado}
\section{Problemas}

\chapter{Algo de CS}
\section{Noción de costo computacional}
\section{Recursividad}

\chapter{Un poco de estructuras de datos}
\section{Diccionarios}
\section{Grafos}
\subsection{Recorridos}
\section{Aplicación}
\subsection{Construyendo un Web Crawler}
\section{Problemas}

\chapter{Programación Orientada a Objetos}
\subsection{Clases y objetos}
\subsection{Herencia y polimorfismo}
\subsection{Creando un videojuego}


%\subsection{La secuencia de Fibonacci}

%\begin{minipage}{.70\textwidth} 
%Filius Bonacci era un monje y matemático italiano del siglo XII. Publicó un libro llamado %\emph{Liber Abbaci}, en el que trataba sobre la utilización de los números indo-arábigos, %que 
%\end{minipage}
%\begin{minipage}{.30\textwidth}
%  \centering
%  \includegraphics[height=5cm]{./Images/Fibonacci.jpg}
%\end{minipage}


\backmatter
% bibliography, glossary and index would go here.
\chapter{Respuestas}
\shipoutAnswer


\bibliography{./bibtex/EvansIntro,./bibtex/DiscMath,./bibtex/ConcMath}


\end{document}

% UNUSED 
%\makeatletter
%\renewcommand\chaptermark[1]{
%	\markboth{\textsc{
%			\ifnum\c@secnumdepth>\m@ne\if@mainmatter
%			\@chapapp\ \thechapter. \ \fi\fi #1}}{}%
%}
%\makeatother