% !TeX spellcheck = de_DE
\documentclass[letterpaper,12pt]{book}

\usepackage{ccfonts}
\usepackage[T1]{fontenc}
\usepackage{textcomp}


\usepackage{fancyhdr}
\pagestyle{fancy}
\fancyhead{}
\fancyhead[RE]{\textsc{\nouppercase{\leftmark}}}
\fancyhead[LO]{\textsc{\nouppercase{\rightmark}}}

\usepackage[explicit]{titlesec}


% SECTIONS STYLE
\titleformat{\section}
{\normalfont\LARGE\scshape}{\thesection.}{1em}{\textsc{#1}}
\titleformat{\subsection}
{\normalfont\Large\scshape}{\thesubsection.}{1em}{\textsc{#1}}
\titleformat{\subsubsection}
{\normalfont\large\scshape}{\thesubsubsection.}{1em}{\textsc{#1}}

% CHAP STYLE
\def\hrulefillthick{\leavevmode\leaders\hrule height3pt\hfill\kern0pt}
\titleformat{\chapter}[display]
{\normalfont\normalsize\scshape}
{\centering\hrulefillthick\hspace*{.5cm}\chaptertitlename\  \thechapter\hspace*{.5cm}\hrulefillthick}
{5pt}
{\titlerule\centering\huge#1}
[\titlerule]

\titleformat{name=\chapter,numberless}[display]
{\normalfont\normalsize\scshape}
{\centering\hrulefillthick\hspace*{0cm}}
{5pt}
{\titlerule\centering\huge#1}
[\titlerule]




\usepackage[utf8]{inputenc}
% no quoting para usar flechas en TIKZ!!!
\usepackage[spanish,es-nodecimaldot,,es-noquoting]{babel}


\usepackage{hyperref}
\hypersetup{
	colorlinks=true,
	menucolor=black,
	linkcolor=black,
	citecolor=black,
	urlcolor=blue}

\usepackage{apacite}
\bibliographystyle{apacite}

\usepackage{graphicx}
\usepackage{tabularx}
\usepackage{listings}
\usepackage{textcomp}
\usepackage{color}
\usepackage{pgfplots}
\usepackage{qtree}
\usepackage{url} 
\usepackage{amsmath}
\usepackage{pdfpages}
\usepackage[font=small,labelfont=bf]{caption}

\usepackage{wrapfig}
\setlength{\columnsep}{20pt}

% No indent
\usepackage[parfill]{parskip}

%qtree captions
\usepackage{newfloat}
\DeclareFloatingEnvironment[fileext=lod]{diagram}

%exercises
\usepackage[answerdelayed,lastexercise]{exercise}

% code
\definecolor{codegreen}{rgb}{0,0.6,0}
\definecolor{codered}{rgb}{0.6,0,0}
\definecolor{codegray}{rgb}{0.5,0.5,0.5}
\definecolor{codepurple}{rgb}{0.58,0,0.82}
\definecolor{backcolour}{rgb}{1,1,1} 
\lstdefinestyle{mystyle}{
    backgroundcolor=\color{backcolour},   
    commentstyle=\color{codegreen},
    keywordstyle=\color{blue},
    numberstyle=\tiny\color{codegray},
    stringstyle=\color{codered},
    basicstyle=\footnotesize,
    breakatwhitespace=false,         
    breaklines=true,                 
    captionpos=b,                    
    keepspaces=true,                 
    numbers=left,                    
    numbersep=5pt,                  
    showspaces=false,                
    showstringspaces=false,
    showtabs=false,                  
    tabsize=2,
    basicstyle=\small\ttfamily,
    frame=tlbr,framesep=4pt,framerule=0.1pt
}
\lstset{style=mystyle}
\lstset{upquote=true}



%\setcounter{secnumdepth}{3}
%\setcounter{tocdepth}{3}

\renewcommand{\lstlistingname}{Código} 


%TITLE
\newcommand*{\plogo}{\fbox{$\mathcal{PL}$}} % Generic publisher logo
%----------------------------------------------------------------------------------------
%	TITLE PAGE
%----------------------------------------------------------------------------------------

\newcommand*{\titleGP}{\begingroup % Create the command for including the title page in the document
	\centering % Center all text
	\vspace*{\baselineskip} % White space at the top of the page
	
	\rule{\textwidth}{1.6pt}\vspace*{-\baselineskip}\vspace*{2pt} % Thick horizontal line
	\rule{\textwidth}{0.4pt}\\[\baselineskip] % Thin horizontal line
	
	{\LARGE INTRODUCCIÓN PRÁCTICA\\ A \\[0.3\baselineskip] LAS CIENCIAS DE LA COMPUTACIÓN }\\[0.2\baselineskip] % Title
	
	\rule{\textwidth}{0.4pt}\vspace*{-\baselineskip}\vspace{3.2pt} % Thin horizontal line
	\rule{\textwidth}{1.6pt}\\[\baselineskip] % Thick horizontal line
	
	\scshape % Small caps
	Un viaje por el mundo de la computación \\ % Tagline(s) or further description
	acompañado de exploraciones apasionantes \\[\baselineskip] % Tagline(s) or further description
	Trabajo inicial 2015--2017. \\ Retomado en 2020.\par % Location and year
	
	\vspace*{2\baselineskip} % Whitespace between location/year and editors
	
	Escrito y editado por \\[\baselineskip]
	{\Large DANIEL CAÑIZARES CORRALES \\ \par} % Editor list
	{\itshape Cofundador de The Science of Code \\ Programador\par} % Editor affiliation
	
	\vfill % Whitespace between editor names and publisher logo
	
	
	{\scshape 2020} \\[0.3\baselineskip] % Year published
	{\large The Science of Code}\par % Publisher
	
	\endgroup}
%ENDTITLE



\begin{document}

%\includepdf{./Images/cover.png}
%\newpage
%\thispagestyle{empty}

\frontmatter
\thispagestyle{empty}
\titleGP

%\thispagestyle{empty}
    \null\vspace{\stretch {1}}
        \begin{flushright}
                This is the Dedication.
        \end{flushright}
\vspace{\stretch{2}}\null
\thispagestyle{empty}

\chapter{Licencia}

Esta obra está bajo una licencia Creative Commons de Reconocimiento, No Comercial, Compartir Igual 4.0 Internacional. \\

 Usted puede compartir, copiar y distribuir el material en cualquier medio o formato, pero no explotar comercialmente esta obra. \\

Para más información consulte: \\ \url{http://creativecommons.org/licenses/by-nc-sa/4.0/}

\includegraphics{./Images/ccbyncsa.png}
\newenvironment{abstract}%
    {\cleardoublepage\thispagestyle{empty}\null\vfill\begin{center}%
    \bfseries\abstractname\end{center}}%
    {\vfill\null}
        \begin{abstract}
        This is the abstract.
        \end{abstract}
\thispagestyle{empty}

\chapter{Prefacio}
    La mayoría de libros sobre programación disponibles en nuestro idioma, omiten muchos conceptos matemáticos y computacionales que son importantes para quien desee realizar una carrera como programador. Esta publicación pretende ser una alternativa rigurosa, pero al mismo tiempo práctica y apasionante, que llevará al lector en un viaje por el mundo de las \emph{Ciencias de la Computación} (CS, por sus siglas en inglés).    
    
    Esta metodología no pretende ser la mejor, ni la única, es solamente una alternativa que he desarrollado y mejorado durante algún tiempo. Puede interesarle a muchos estudiantes que no han tenido la oportunidad de acercarse a la programación (entendida más allá del simple desarrollo de software). Igualmente, este libro puede resultar interesante para profesores que quieran experimentar otras formas de enseñar; no puedo garantizarle que todos sus estudiantes aprenderán a programar, pero sí estoy seguro de que esté método es mucho más efectivo que el tradicional, sobre todo en la transmisión de buenas prácticas de programación y en el desarrollo de pensamiento computacional.
    
    Este trabajo fue inspirado por clases como \emph{CS-101 de Udacity} y el libro del profesor David Evans \emph{Introduction to Computing}. 
    
    La forma de trabajo que propongo confía en que usted se ponga manos a la obra ya que en vez de mostrar mil formas de escribir una instrucción, considero que es más apasionante sentarse a programar cosas de verdad.
    
  Finalmente, quiero invitarlo a que visite \url{http://thescienceofcode.com}, una comunidad abierta y dedicada a la enseñanza de las ciencias de la computación como arte liberal.

    
    \newpage
    \thispagestyle{empty}
        \textbf{Nota para profesores}: Empezar directamente con este libro puede ser difícil para un estudiante que no ha tenido ningún tipo de exposición previa a las Ciencias de la Computación, por eso recomiendo a los profesores que expongan a sus alumnos a programas como \emph{Code.org}, que explican de forma muy didáctica la mayoría de conceptos básicos de CS.
      
    
  
    
 
\tableofcontents

\newpage
\thispagestyle{empty}

\mainmatter

\chapter{Introducción}

\begin{quote}
\emph{Ya sea que quieras desvelar los secretos del universo, o quieras hacer una carrera [en ciencias de la computación], la programación básica de computadores es una habilidad esencial por aprender en el siglo XXI.} \\
Stephen Hawking.
\end{quote}

Los esfuerzos humanos por crear máquinas para realizar cómputos de manera automática datan de muchos siglos atrás, pero no fue sino hasta la \emph{Segunda Guerra Mundial} que se contó con el dinero y los componentes necesarios para construir máquinas programables \cite[p.~108]{evansIntro}. En un comienzo su uso fue meramente bélico, las tareas más comunes estaban relacionadas con cálculos de trayectorias para cohetes balísticos. Sin embargo, estás máquinas marcaron un punto de inflexión en la historia de la humanidad: por primera vez un mismo aparato servía para casi cualquier propósito (siempre y cuándo éste fuese \emph{computable}, discusión que abordaremos en el último capítulo de este libro).

Parece increíble, pero los computadores que tenemos en la actualidad están basados en el mismo modelo que planteó Alan Turing en la década de 1930. A grandes rasgos, solamente tenemos equipos más rápidos y con más memoria. Esa mayor capacidad de cómputo nos permite contar con programas cada vez más sofisticados, pero también con herramientas que hacen más fácil nuestra tarea como programadores.
 \newpage

\begin{wrapfigure}{r}{0.5\textwidth}
	\begin{center}
	\includegraphics[width=0.5\textwidth]{./Images/eniac.png}
	\end{center}	
	\caption{El ENIAC, uno de los primeros computadores de la historia. Foto bajo  Dominio Público.}
\end{wrapfigure}

En este sentido, la programación de los primeros computadores era bastante difícil y se realizaba básicamente conectando y desconectando cables. Conforme avanzó el tiempo y aumentó la capacidad de cómputo, aparecieron los \textbf{compiladores}. Básicamente lo que hacen es tomar un programa de computador escrito en un lenguaje que es más fácil de entender para los humanos (\emph{lenguaje de alto nivel}), lo traducen a un lenguaje que sea más cercano al que entiende la máquina (\emph{lenguaje de bajo nivel}), produciendo lo que conocemos como un \emph{ejecutable}. Al final del día, un programa termina siendo una secuencia de bits (unos y ceros) que representan una serie de instrucciones que puede ejecutar la máquina. Los primeros compiladores fueron desarrollados por la Almirante Grace Hopper en la década de 1950.

Existe otro tipo de programas que realizan una tarea similar, pero a diferencia de los compiladores que toman el código y lo traducen completamente, los \textbf{intérpretes} realizan lo que podríamos llamar una \emph{traducción en simultáneo}. 

Cada enfoque tiene sus ventajas y desventajas: el código compilado es más rápido (gracias a la traducción previa que se realiza) pero en el código interpretado es más ágil el proceso de realizar cambios y ver cómo estos afectan la ejecución del programa.

El mecanismo utilizado para traducir los códigos a instrucciones que entiende la máquina es una forma en la cuál podemos clasificar los lenguajes de programación, sin embargo no es la única: la forma en que resolvemos problemas es una clasificación recurrente con la que nos encontraremos. A saber, existen dos grandes corrientes: la \textbf{programación imperativa} y la \textbf{programación funcional}. El enfoque de la primera es escribir instrucciones para modificar el estado de la memoria del computador, hasta eventualmente llegar a un estado que contenga la solución del problema. Por su parte, la programación funcional está relacionada con conceptos mucho más cercanos a las matemáticas, que en general tienen que ver con la aplicación de funciones que no manejan estado y que pueden \emph{componerse} para llegar a la solución. Obviamente ambos conceptos son mucho más profundos y los iremos explorando a medida que avancemos en los tópicos propuestos.

En la realidad nos encontraremos con programas que mezclan lenguajes compilados e interpretados. Por lo general, las partes más críticas de los sistemas son escritas en lenguajes compilados como \emph{C++} y son extendidas por lenguajes interpretados como \emph{Python} o \emph{Lua}. De igual forma, se deberían mezclar y aplicar técnicas propias de cada enfoque (imperativo y funcional), explotando al máximo las ventajas de ambos.

En este libro usaremos \emph{Python}, un lenguaje simple pero poderoso: interpretado, imperativo y con ciertos toques de programación funcional. Sin embargo, cabe anotar, que la mayoría de conceptos aquí expuestos pueden utilizarse en otros lenguajes, y que lo más importante no es la tecnología aplicada en un programa específico. La forma de aproximarse y construir soluciones computacionales, es lo que realmente debería importarle al lector.



\chapter{Nociones básicas}

En este capítulo veremos algunos de los conceptos involucrados en la programación de ordenadores a nivel básico. No se pretende que el lector los domine con sólo ver este capítulo, al contrario, la idea es dar un vistazo general sobre lo que utilizaremos y profundizaremos a lo largo del libro, de forma que, teniendo una imagen más completa de la programación los temas a tratar no parezcan obstáculos sin sentido, que sólo tienden a ralentizar el proceso de aprendizaje. 

Lo primero será comprender cómo las instrucciones de un programa operan directamente en la máquina, para esto introduciremos un modelo abstracto que representará nuestro computador.

\section{Máquina RAM}

 La Máquina de Acceso Aleatorio (RAM, por sus siglas en inglés), es un modelo utilizado para realizar análisis de algoritmos \cite[p.~31]{algDsgn}, tema que abordaremos más adelante. De momento introduciremos los conceptos más elementales sobre cómo funciona un computador para poder entender mejor de qué va esto de la programación. 
 
Vamos a imaginar un computador abstracto que se comporta de forma muy similar a cómo lo hacen nuestras máquinas reales, a ese computador lo llamaremos la Máquina RAM. Para mantener las cosas lo más sencillo posible, sólo consideraremos los siguientes componentes:


\begin{figure}[h!]
	\centering
	\includegraphics[width=10cm]{./Images/ram.png}
	\caption{Máquina RAM}
	\label{figram}
\end{figure}




\begin{itemize}
\item La memoria: representa el estado de la máquina en un momento del tiempo. Está dividida en múltiples secciones que llamamos posiciones de memoria, cada posición puede almacenar un número.

\item La entrada: son los datos sobre los cuáles se va a operar, estos pueden ingresar representados por: una pulsación del teclado, un clic, información que viaja por Internet, la señales de un sensor, etc.
 
\item El programa: es una secuencia de instrucciones que le dice a la máquina cómo modificar su memoria de acuerdo a la entrada que recibe y al estado mismo de la memoria. 

\item La salida: es el resultado que se obtiene después de ejecutar total o parcialmente el programa. Decimos \emph{parcialmente}, porque no tenemos que esperar hasta el final para compartir información con el usuario, por ejemplo, durante la ejecución podemos ir mostrando una barra de carga o mostrando mensajes sobre los datos que estamos procesando.
\end{itemize}



De una u otra forma, el programa tendrá que interactuar con todos los componentes de la máquina, luego, los lenguajes de programación deben proveer mecanismos para manipular y representar la memoria, la entrada y la salida de datos. Cabe anotar que, como el programa depende del estado actual de la máquina, también se debería contar con estructuras que permitan repetir o ejecutar ciertas instrucciones de acuerdo a los datos almacenados en memoria. Por ejemplo, se deben ejecutar algunas instrucciones si la máquina entra o en un estado de error o se deben repetir algunas instrucciones mientras que una persona esté presionando una tecla en un videojuego. A propósito veamos con un ejemplo práctico todos estos conceptos.


\section{Modelando un videojuego}


\begin{figure}[h!]
	\centering
	\includegraphics[width=12cm]{./Images/pong.png}
	\caption{El legendario PONG.}
	\label{figpong}
\end{figure}

Supongamos un videojuego sencillo como \emph{Pong} (ver figura \ref{figpong}). El juego es una especie de \emph{tenis} en el que dos jugadores, cada uno en control de una barra que representa una raqueta, deben golpear una pelota e intentar que el otro jugador no alcance a responder el golpe, para marcar un punto. 

Intentemos comprender cómo representar este juego en nuestra \emph{Máquina RAM}. No lo haremos exhaustivo pero un bosquejo, que permita entender a nivel general las mecánicas de la programación. Empecemos de lo más fácil a lo más complejo:

\begin{itemize}

	
	\item La entrada: en este caso la entrada serían las teclas que presionan los jugadores para mover arriba y abajo cada una de las raquetas.
	
	\item La salida: son los gráficos que vemos en la pantalla y que representan lo que está pasando en estado de la máquina mientras simula el videojuego.
	
	\item La memoria: para poder representar el estado del juego en la máquina es necesario tener espacios de memoria donde guardemos:
	
	\begin{itemize}
		\item La puntuación del jugador 1.
		\item La puntuación del jugador 2.
		\item La posición en el eje Y de la raqueta del jugador 1.
		\item La posición en el eje Y de la raqueta del jugador 2.
		\item La posición en el eje X de la pelota.
		\item La posición en el eje Y de la pelota.
		\item La velocidad de la pelota.
		\item La dirección en el eje X de la pelota, es decir, si va hacia la derecha o la izquierda.
		\item La dirección en el eje Y de la pelota, es decir, si va hacia arriba o abajo.
	\end{itemize}

	Con todos estos datos, es posible saber el estado del juego y así, el computador puede pintar en pantalla qué es exactamente lo que está ocurriendo.
	
	\item El programa: para que nuestra máquina simule realmente el videojuego tendríamos que tener en cuenta darle instrucciones para cada uno de los siguientes casos: 
	
	\begin{itemize}
		\item Pintar en la pantalla las raquetas, la pelota y la puntuación.
		
		\item \textbf{Si} el jugador 1 presiona la tecla arriba, mover la posición de su raqueta hacia arriba; \textbf{Si} presiona la tecla abajo, mover su raqueta hacia abajo. De igual forma habría que configurar un par de teclas para que el jugador 2 pueda moverse.
		
		\item Mover la posición de la pelota de acuerdo a su velocidad y dirección.
		
		\item \textbf{Si} la pelota colisiona con una raqueta, cambiar su dirección.
		
		\item \textbf{Si} la pelota toca la esquina izquierda, sumar más uno a la puntuación del jugador 1; \textbf{si} la pelota toca la esquina derecha, sumar a la puntuación del jugador 2.
				
		\item \textbf{Repetir} desde el primer paso, hasta que los jugadores se salgan del juego.
	\end{itemize}
\end{itemize}

Nótese que usamos palabras importantes (\textbf{Si}, \textbf{Repetir}) para definir cuando ejecutar o no una instrucción, o cuantas veces repetir una serie de pasos en nuestro código.

Hecho todo esto, y si la Máquina RAM entendiera español, más o menos podría simular \emph{Pong} para nosotros. Sin embargo, nuestros computadores no son tan inteligentes, luego debemos ser mucho más explícitos en las instrucciones que utilizamos, tema que comenzaremos a explorar en el próximo capítulo.

Para concluir, como podemos ver el programa termina siendo una serie de instrucciones para que el computador sepa que hacer en un momento determinado. Al principio puede parecer complejo entender los límites de lo que podemos programar o incluso más difícil aún, saber cómo expresar nuestras ideas en instrucciones que entienda la máquina. Similar a aprender un nuevo idioma, con poco de práctica y algo de imaginación, estas barreras irán desapareciendo a medida que avancemos.

\section{Problemas}
\begin{Exercise}[title={Acercamiento inicial}]	
	
Defina con sus propias palabras, pero tratando de ser muy claro y ordenado, las entradas, las salidas, la memoria y el programa para que la Máquina RAM pueda simular las siguientes situaciones:

	\begin{enumerate}
		\item Una calculadora que permita sumar o restar dos números.
		
		\item El videojuego \emph{Flappy Bird}. Es recomendable buscarlo en internet para tener claras las mecánicas del juego.
	\end{enumerate}
\end{Exercise}
\begin{Answer}	
	
Estos problemas no tienen una única solución, la idea es comenzar a interiorizar las mecánicas sobre las cuáles se escriben programas de computados (sin llegar a ser demasiado estrictos). El lector debería comparar sus respuestas con las soluciones propuestas y pensar en qué partes faltó claridad en su programa.
		
\begin{enumerate}
\item Calculadora:
	\begin{itemize}
		\item Entrada: números ingresados por teclado además de las teclas +, - y ENTER.
		\item Salida: resultado de la operación. También podríamos ir mostrando los datos ingresados por el usuario.
		\item Memoria: deberíamos poder almacenar el \emph{numero1}, el \emph{numero2}, la \emph{operación} a realizar y el resultado. Para la operación podríamos ser más específicos, ya que en la memoria sólo podemos almacenar números, el 0 representará restar y el 1 sumar.
		\item Programa:
		\begin{itemize}
			\item Almacenar el primer número digitado por el usuario en la memoria como  \emph{numero1}.					
			\item \textbf{Si} el usuario digita el signo -, almacenar en la memoria en \emph{operación} un 0.
			\item \textbf{Si} el usuario digita el signo +, almacenar en la memoria en \emph{operación} un 1.
			\item Almacenar el segundo número digitado por el usuario en la memoria como  \emph{numero2}.
			\item \textbf{Si} la \emph{operación} es 0, mostrar en la pantalla el resultado de restar: \emph{numero1 - numero2}.					
			\item \textbf{Si} la \emph{operación} es 1, mostrar en la pantalla el resultado de sumar: \emph{numero1 + numero2}.
			\item \textbf{Repetir} desde el primer paso hasta que el usuario se salga del programa.
		\end{itemize}
		
	\end{itemize}
	
\item Flappy birds:
	\begin{itemize}
		\item Entrada: el jugador tiene una única tecla ue presiona para que el pajaro aletee y vuele hacia arriba.
		\item Salida: en la pantalla debemos pintar el estado del juego, mostrando el jugador y los obstáculos.
		\item Memoria: deberíamos poder almacenar la \emph{puntación del jugador}, su \emph{posición}, la \emph{velocidad en el eje X} del jugador, la \emph{velocidad en el eje Y} y para mantener las cosas simples, sólo diremos que guardaremos la posición y tamaño de algunos obstáculos que irán apareciendo en la pantalla.
		\item Programa:
		\begin{itemize}
			\item Pintar en la pantalla el pájaro y los obstáculos visibles. Calcular los obstáculos visibles puede ser un poco complejo, pero por facilidad en la explicación no entraremos de momento en detalles		
			\item \textbf{Si} el jugador presiona la tecla de aletear, se debe incrementar la \emph{velocidad en el eje Y} de forma que el pájaro suba un poco.					
			\item Actualizar la posición del jugador de acuerdo a su velocidad en cada eje. A la velocidad en Y se le debe restar una cantidad, para que el impulso que tenía se vaya perdiendo y vuelva a comenzar a caer eventualmente.
			\item \textbf{Si} el jugador colisiona con un obstáculo pierde.
			\item \textbf{Repetir} desde el primer paso hasta que el jugador pierda.
		\end{itemize}
		
	\end{itemize}
\end{enumerate}
\newpage
\end{Answer}

\newpage

\chapter{Introducción a Python}

Luego de ver a grandes rasgos cómo programar un computador, llegó la hora de explorar lo que hemos aprendido en un lenguaje de programación real. 

Cómo vimos, los principales componentes de un computador son \emph{la entrada}, \emph{la salida}, \emph{la memoria} y \emph{el programa}. Ya que vamos a escribir nuestros programas en Python, surge una pregunta: ¿Cómo interactúa Python con los componentes de nuestro computador? Después de muchos intentos con distintas explicaciones, quizá la mejor forma de responder esta pregunta es mediante ejemplos reales en el lenguaje de programación. Se recomienda encarecidamente al lector, que pruebe todos los programas en su máquina, para que los vea en funcionamiento. De igual forma, nada de esto tendrá sentido si el lector pasa rápidamente sobre este libro y no se toma el tiempo para interiorizar cada instrucción.

\section{Las abstracciones básicas}

Comencemos por dar un intuición de qué es una \emph{abstracción}:

\emph{Una abstracción es un mecanismo que permite manipular una máquina sin necesidad de conocer a fondo su funcionamiento. Por ejemplo, el pedal de un carro permite que usted acelere sin necesidad de conocer el funcionamiento del motor}. 

En la programación de ordenadores, las abstracciones son un concepto sumamente útil porque nos permiten desarrollar nuestros programas sin necesidad de conocer todos los detalles de nuestras máquinas, así por ejemplo, para desarrollar un videojuego no necesitamos conocer exactamente cómo funciona una tarjeta gráfica para pintar dibujos en la pantalla, ni tampoco saber los pormenores de una tarjeta de sonido para reproducir una canción: existen ya abstracciones, así como los pedales de un carro, que de forma mucho más intuitiva nos permiten interactuar con dichos componentes.

Aunque en un computador todo termina siendo unos y ceros, Python tiene algunas abstracciones básicas (otras más avanzadas, que veremos en capítulos posteriores) que harán todo más fácil.

\subsection{Abstracciones de los datos}

En Python contamos con diferentes \textbf{tipos de datos} para determinar las acciones que podemos realizar con los valores almacenados en la memoria del computador. La ventaja es que no  necesitamos conocer a profundidad qué hace el computador con ellos, ni cómo los almacena, simplemente los utilizamos y operamos a conveniencia. Por ejemplo, si necesitáramos guardar un texto u operar un par de números, Python tiene abstracciones ya definidas para ellos. En la tabla \ref{tabla_tipos} podemos ver los principales tipos que existen.

\begin{table}[ht]
\centering
\begin{tabular}{ l c c c }
	
	\hline
	Tipo de dato & Python & Ejemplos de valores posibles \\
	\hline
	Números enteros & int  & 0, -1, 1, 40, 52,  \\ 
	Números con decimales & float & 0, 0.1, -0.2, 22.5 \\ 
	Valores lógicos & bool & True, False \\
	Texto & str & 'Esto es un texto!' \\
	\hline
	
\end{tabular}
\caption{Los principales tipos de datos básicos en Python.}
\label{tabla_tipos}
\end{table}

A continuación por diseccionar un poco cada grupo, empezando por los números (incluyendo a ambos, enteros y decimales).

\subsubsection{Números}

En Python, los enteros se llaman \textbf{int} (de \emph{integer}, en inglés) y los que tienen decimales se conocen como \textbf{float}. Permiten efectuar operaciones aritméticas, las más comunes son:

\begin{itemize}
\item Sumas (+): $5 + 2$ evalúa $7$.
\item Restas (-): $5 - 2$ evalúa $3$.
\item Multiplicaciones (*): $5 * 2$ evalúa $10$.
\item Divisiones (/): $5 / 2$ evalúa $2.5$.
\item Divisiones enteras (//): $5 // 2$ evalúa $2.5$.
\item Residuos (\%): $5 \% 2$ evalúa $1$.
\item Potencias (**): $5 ** 2$ evalúa $25$.
\end{itemize}

Las dos operaciones que se salen un poco del conocimiento general son las divisiones enteras y los residuos. Aclaremos para qué sirven: 

\begin{itemize}
\item Si luego de realizar una división queremos eliminar los decimales usamos el operador división entera (//). En el ejemplo: \emph{5 / 2} da como resultado \emph{2.5}; \emph{5 // 2} da como resultado \emph{2}.

\item Cuando hablamos de residuos, es el valor que sobra luego de realizar una división. En el ejemplo: Si dividimos 5 entre 2 el residuo es 1, luego \emph{5 \% 2} da como resultado \emph{1}.
\end{itemize}


\subsubsection{Valores lógicos}

Los valores lógicos se conocen en Python como \textbf{bool} (booleanos) y sólo tienen dos valores posibles verdadero (\textbf{True}) o falso (\textbf{False}). Podemos aplicar los operadores:

\newpage
\begin{itemize}
\item Y (\textbf{and}): Evalúa que dos valores sean verdaderos.
	\begin{itemize}
	\item $True$ and $True$ evalúa $True$.
	\item $True$ and $False$ evalúa $False$.
	\item $False$ and $True$ evalúa $False$ .
	\item $False$ and $False$ evalúa $False$ .
	\end{itemize}
\item  O(\textbf{or}): Evalúa que al menos uno de los valores sea verdadero.
	\begin{itemize}
	\item $True$ and $True$ evalúa $True$.
	\item $True$ and $False$ evalúa $True$.
	\item $False$ and $True$ evalúa $True$ .
	\item $False$ and $False$ evalúa $False$ .
	\end{itemize}
\item Negación (\textbf{not}): cambia el valor al contrario posible.
	\begin{itemize}
	\item not $True$ evalúa $False$.
	\item not $False$ $True$.
	\end{itemize}
\end{itemize}

Al igual que con una expresión aritmética, donde podemos mezclar sumas y restas (por decir un par de operaciones), con los valores lógicos podemos utilizar varios operadores para componer proposiciones más complejas, como por ejemplo, analizar si tres valores son verdaderos.

\subsubsection{Texto} 

Se conocen como \textbf{str} (strings) o secuencias de caracteres. Empiezan por una comilla doble o sencilla y terminan por el mismo símbolo, esto, con el fin de distinguir los textos arbitrarios definidos por el programador de otras instrucciones propias del lenguaje de programación (que en breve comenzaremos a poner en práctica).

Para los textos, en los programas podemos analizarlos y manipularlos utilizando diferentes operaciones y técnicas, tópico que revisaremos en el capítulo Qué es un Parser (FALTA: Referencia al capítulo). De momento sólo los utilizaremos para mostrar información útil en la pantalla, como veremos en breve.

\subsection{Abstracciones de la memoria}

Python no sólo facilita la representación de los datos en memoria, sino también su almacenamiento y manipulación. Para este fin introduciremos el concepto de \textbf{variables}.

\subsubsection{Las variables}

Permiten dar un nombre arbitrario y un valor a partes de la memoria utilizando el signo igual (=). Es importante aclarar que \emph{Python detecta automáticamente el tipo dato que se le asigna a una variable}, luego simplemente podemos realizar las asignaciones como se ve en el código \ref{cod-asignacion}. \\

\lstinputlisting[language=Python,caption=Asignaciones en Python,label=cod-asignacion]{./py/python/asignacion.py}

Existen tres restricciones que debemos tener en cuenta antes de nombrar una variable: usar únicamente letras, números o guiones bajos, no empezar por un número y no dejar espacios en blanco. Si esto no se cumple el código presentará un error.

Por otro lado, la forma más común de nombrar variables en Python, aunque no es obligatoria es recomendada: consiste en escribir todo en minúsculas y usar nombres \emph{descriptivos}, si el nombre son varias palabras se separan por guiones bajos. Si esto no se cumple el código aún funcionará, pero el programa será, probablemente, menos claro que si seguimos las recomendaciones. \\

Anotemos también que es posible realizar comentarios a nuestro código añadiendo el caracter \#, donde podemos escribir aclaraciones del programa, sin afectar su comportamiento.

En el fragmento \ref{cod-constantes} podemos ver ejemplos de nombramientos correctos, no recomendados e incorrectos. \\

\lstinputlisting[language=Python,caption=Ejemplos para nombres de variables,label=cod-asignacion]{./py/python/nombres.py}

Es también importante anotar, que en programas pequeños, como los ejemplos que veremos en breve, es posible que usemos nombres más cortos (como una letra, similar a cómo se hace en las expresiones matemáticas), para facilitar la manipulación de las variables; en programa más grandes, se privilegia en general el uso de nombres más largos y descriptivos.

\subsubsection{Las constantes}

Cuando estamos programando, es común encontrarnos con valores generales que no cambian durante la ejecución del código. Dichos valores los llamamos constantes, y aunque en Python no existe una distinción especial entre \emph{variables} y \emph{constantes}, se acostumbra a nombrarlos con MAYÚSCULAS\_SOSTENIDAS, de esta forma podemos distinguir fácilmente que valores NO deberíamos cambiar bajo ninguna circunstancia. \\

\lstinputlisting[language=Python,caption=Asignaciones en Python,label=cod-constantes]{./py/python/constantes.py}

Hasta este punto, si ejecutáramos alguno de los códigos anteriores en el computador no veríamos nada porque aún no hemos definido instrucciones para la salida de datos.

\subsection{Abstracciones de entrada y salida}

Las siguientes instrucciones nos permiten interactuar con el usuario de una forma básica.

\subsubsection{Salida de datos}

La forma básica de mostrar información en la pantalla es usando la instrucción \emph{print}. Basta con escribirla y dentro de paréntesis colocar el valor, la operación o variable que se desea mostrar. En caso de necesitar imprimir múltiples valores, se pueden poner separados por comas, como se ve en el fragmento \ref{cod-salida} \\

\lstinputlisting[language=Python,caption=Salida de datos en Python,label=cod-salida]{./py/python/salida.py}

\subsubsection{Entrada de datos}

En cuanto a la entrada de datos básica, tenemos la instrucción \emph{input()}; entre los paréntesis se puede colocar un mensaje para que le aparezca al usuario. Luego de ingresar los caracteres por teclado y presionar la tecla \emph{enter}, los datos quedan almacenados en la variable a la cuál se haya asignado la instrucción. Sin embargo, hay que anotar que la instrucción \emph{input()} nos devuelve un \emph{string}(o texto), luego, si queremos hacer operaciones númericas, debemos convertir ese texto en un número. Dicha conversión se conoce con el nombre de \emph{casting}. De momento, usaremos alguna de las dos opciones que se ve en el fragmento \ref{cod-casting}. \\

\lstinputlisting[language=Python,caption=Algunos castings en Python,label=cod-casting]{./py/python/casting.py}

 Por ejemplo, el código \ref{cod-suma} muestra como solicitar dos números por teclado para luego imprimir el valor que resulta al sumarlos \footnote{Si estamos esperando un número y el usuario ingresa una palabra, el programa terminará. Existen mecanismos de validación, pero por motivos pedagógicos, de momento omitiremos validaciones adicionales y nos centraremos en el proceso de construcción de nuestras soluciones.}. 
 \newpage
 \textbf{Importante:} Qué pasa si no convertimos las entradas en enteros? Se recomienda al lector que ensaye en su computador. Si quedan dudas, en la sección problemas de este capítulo podrá encontrar la respuesta. \\

\lstinputlisting[language=Python,caption=Lectura y suma de dos números en Python,label=cod-suma]{./py/python/suma.py}

\subsection{Abstracciones para el programa}

Existen dos mecanismos básicos para controlar el flujo de nuestro programa, a saber:

\begin{itemize}
\item Condicionar las instrucciones: es decir, indicar en qué casos se debería o no ejecutar una parte de nuestro código.
\item Repetir las instrucciones: para determinar fácilmente que bloques de nuestro código se deben ejecutar más de una vez.
\end{itemize}

Exploremos cómo realizar ambos mecanismos en Python.

\subsubsection{Condicionar las instrucciones}

Para este fin utilizamos la expresión \emph{if}, que traduce \emph{si} en español. Dicha expresión, debe ir acompañada de una \emph{condición}, que no es más que la evaluación, \emph{verdadera} o \emph{falsa}, de una proposición utilizando los operadores lógicos: es igual (==) \footnote{Nótese que se usa doble igual para distinguir la comparación de la asignación.}, mayor ($>$), menor($<$), mayor o igual ($>=$), menor o igual ($<=$) o diferente (!=). Por ejemplo, el código \ref{cod-if} sólo ejecuta la instrucción que imprime \emph{Es cero}, si  el valor ingresado es igual a 0. Las instrucciones que están condicionadas \textbf{deben ir tabuladas}, como se observa en el código de ejemplo, y pueden ser una o más instrucciones de cualquier tipo. \\

\lstinputlisting[language=Python,caption=Un \emph{if} en Python,label=cod-if]{./py/python/if.py}


La instrucción \emph{if} puede ir acompañada de un \emph{else}, que traduce \emph{si no} en español, y que se utiliza para indicar qué instrucciones ejecutar cuando la condición no se cumple. Esta situación se puede observar en el código \ref{cod-ifelse}, en el cuál se reciben dos números y se imprime cuál de ellos es el mayor (en caso de ser iguales, no importará cuál mostremos). Nótese que el texto entre comillas se muestra tal cuál en patalla, mientras que el valor después de la coma se reemplaza por el valor que tenga la variable en el momento de la ejecución. \\

\lstinputlisting[language=Python,caption=Un \emph{if-else} en Python,label=cod-ifelse]{./py/python/ifelse.py}

Si se necesitan condicionar más caminos, se puede utilizar la expresión \emph{elif}, que es una abreviación de \emph{else if}. Por ejemplo, si quisiéramos saber cuál es el mayor de tres números (llámense $a$, $b$ y $c$), habría tres opciones: 

\begin{itemize}
\item \emph{Si} $a$ es mayor que $b$ y que $c$, en tal caso, $a$ sería el mayor.

\item \emph{Si no}, se debe descartar entre $b$ y $c$ cuál es mayor, luego, \emph{si} $b$ es mayor que $c$, $b$ es el mayor.

\item \emph{Si no}, ninguna de las anteriores condiciones se cumple, es porque el mayor era $c$. 
\end{itemize}


Lo expresado anteriormente se observa en el código \ref{cod-ifelifelse}. Es importante resaltar en este punto, que si la condición es más compleja que evaluar un único valor, podemos usar los operadores \emph{and}, \emph{or} y \emph{not} para componer una expresión que sirva para representar el caso que necesitemos.   \\

\lstinputlisting[language=Python,caption=El mayor de tres números,label=cod-ifelifelse]{./py/python/ifelifelse.py}


\subsubsection{Repetición de instrucciones}

En cuanto a la repetición de instrucciones, se utiliza la instrucción \emph{for} \footnote{Entre otras que veremos en capítulos posteriores.}, que traduce \emph{para} en español. Veamos como funciona: \\

\lstinputlisting[language=Python,caption=Repetición de instrucciones en Python,label=cod-for]{./py/python/for.py}

El código anterior imprime los números desde el cero hasta el nueve (nótese que el límite superior \textbf{no} se incluye). Su lectura en español nos ayudará a entenderlo: \emph{para i en el rango de 0 a 10, imprima i}. La $i$ es una variable cualquiera, por lo cuál podemos utilizar otro nombre; es común usar $i$ por su relación con los índices y subíndices en matemáticas. Los dos números dentro del paréntesis de \emph{range} indican desde que valor inicia la repetición y en cuál termina. Al igual que con la estructura \emph{if}, las instrucciones que se repiten deben ir tabuladas.

Si quisieramos mostrar los números en un rango digitado por el usuario, donde \emph{a} es el límite inferior y \emph{b} es el límite superior, podríamos contruir un programa como el que se ve en el fragmento \ref{cod-for-rango}.

\lstinputlisting[language=Python,caption=El mayor de tres números,label=cod-for-rango]{./py/python/for-rango.py}

Es posible mezclar no sólo las instrucciones de entrada y salida, si no también todas las demás que hemos visto (y las que veremos), por ejemplo, tener programas como el del código \ref{cod-forif} que recibe dos números, imprime todos los enteros que hay entre ellos, indicando si son negativos o positivos. \\


\lstinputlisting[language=Python,caption=Repeticiones y decisiones en Python,label=cod-forif]{./py/python/forif.py}



\section{Problemas}



\newpage
%\chapter{Programación imperativa}
 
Las restricciones propias de los computadores y el tipo de soluciones que se desean encontrar al codificar un programa, 

\section{¿Qué es un algoritmo?}

\newpage


\section{Paradigmas de programación}

En la programación de ordenadores, existen a grandes rasgos dos corrientes que determinan la forma en que nos aproximamos a un problema \footnote{Básicamente el resto de paradigmas que puedan encontrarse se derivan a nivel teórico de alguno de estos dos.}. De igual forma, esos enfoques (o \emph{paradigmas}) nos brindan ciertas técnicas que guían la forma en cómo podemos construir nuestras soluciones.

\begin{enumerate}
\item \textbf{Programación imperativa}: el enfoque está en escribir instrucciones para modificar el estado de la memoria del computador, hasta eventualmente llegar a un estado que contenga la solución del problema.

\item \textbf{Programación funcional}: está relacionada con conceptos mucho más cercanos a las matemáticas, que en general tienen que ver con la aplicación de funciones que no manejan estado y que pueden \emph{componerse} para llegar a la solución.
\end{enumerate}

Cabe resaltar que ambos paradigmas suelen mezclarse, en mayor o menor medida, para construir soluciones computacionales. De momento nos centraremos en explorar la \emph{programación imperativa}.

\section{Pensamiento imperativo}

Cómo dijimos anteriormente, en la \emph{programación imperativa} modificamos constantemente la memoria del computador para intentar llevarla hasta un estado en el que obtengamos la respuesta a nuestro problema. Esto plantea algunas dificultades ya que \emph{una posición de memoria cambiará muchas veces durante la ejecución del programa}, luego será necesario abstraernos un poco más, de forma que podamos \emph{comprender el significado de nuestras instrucciones en el tiempo}. Es decir, que en nuestros programas, una variable cambiará de valor muchas veces durante la ejecución, de forma que debemos estar familiarizados con algunas herramientas que nos permitan comprender que es lo que realmente tenemos almacenado en esa variable a medida que el programa es ejecutado.

Mediante unos problemas, exploraremos las tres técnicas que debe dominar todo programador imperativo para manipular de forma básica las abstracciones básicas de la memoria del computador, a saber: cómo sumar múltiples datos, cómo contar elementos y cómo vigilar valores anteriores.

\subsection{Las sumas de Gauss}

\begin{wrapfigure}{r}{0.3\textwidth}
	\begin{center}
		\includegraphics[width=0.3\textwidth]{./Images/gauss.jpg}
	\end{center}	
	\caption{Karl Gauss.}
\end{wrapfigure}


Se cuenta que cuando Karl Gauss estaba en la escuela primaria, su maestra asignó a la clase la tarea de sumar los enteros del 1 al 100 (por ejemplo, 1 + 2 + 3 + · · · + 100) para mantenerlos ocupados \cite[p.~59]{evansIntro}. Gracias a sus habilidades, Gauss pudo desarrollar una fórmula para calcular esta suma de una forma mucho más ágil. Exploremos dos formas de cómo solucionar este problema usando un computador.

\subsubsection{Solución imperativa}
Cómo vimos en el capítulo anterior, en Python es posible repetir instrucciones de forma sencilla, utilizando la instrucción \emph{for}. Ahora, la pregunta es ¿cómo repetimos las instrucciones, de manera que al finalizar, el estado de la máquina tenga la respuesta a este problema?

La respuesta es corta, pero quizá no muy intuitiva. Antes de continuar, es recomendable que se tome un par de minutos y escriba un programa de computador, con las instrucciones que usted cree, darán la respuesta. No importa si al final el código tiene errores, poner a trabajar las neuronas es más importante que llegar a la solución al primer intento. Una vez tenga su aproximación, continúe leyendo.
\\

Ya que no tenemos una hoja de papel para hacer un montón de operaciones, sólo posiciones de memoria que guardan bits (representadas en Python por el concepto de variables), necesitamos pensar muy bien como expresar las instrucciones. Hagámonos la siguiente pregunta, ¿será necesario almacenar en una variable diferente cada operación realizada? La respuesta es no. Imagine que usted tiene solo un pequeño trozo de papel, ¿cómo lo utilizaría para sumar los números del 1 al 100? Usted podría realizar operaciones y a medida que avanza ir borrando lo que ya no necesita guardarlas todas. Por ejemplo podría:

\begin{itemize}
\item Sumar 1 + 2 = 3. Como ya sabe la respuesta, puede borrar la operación y sólo conservar en su mente el número 3.
\item Luego, hay que sumar 3 + 3 = 6. De igual forma, sólo es necesario conservar el 6 presente, el resto se puede borrar.
\item La siguiente suma en el trozo de papel sería 6 + 4 = 10. Siguiendo el mismo órden de ideas, pasamos a la siguiente suma.
\item 10 + 5 = 15
\item 15 + 6 = 21
\item (...)
\item Y repetir, en el mismo pedazo de papel, hasta sumar los 100 números. 
\end{itemize}

En el computador la idea es la misma. No necesitamos 100 variables, usaremos una sola variable que iremos cambiando a medida que avance el tiempo, de forma que se acumule la suma. Ahora, si observamos cuidadosamente el proceso anterior, podemos encontrar que estamos sumando el valor que vamos acumulando, con cada uno de los 100 números. Luego, si damos el nombre de $suma$ a la variable en la que acumulamos el resultado, y le damos el nombre de $i$ a cada uno de los 100 número, la operación en el computador sería: 

\begin{equation}
suma = suma + i
\end{equation}

Puede revisar el código \ref{cod-for} para mayor claridad sobre el tratamiento que tendremos con la variable $i$.

Si metemos la expresión dentro de un \emph{for}, ¿funcionará? de hecho no. Notemos que el computador acumula haciendo $suma$ igual a lo que tenía más $i$. Y entonces, ¿qué tiene $suma$ la primera vez? Nada, pero hay que decírselo al computador iniciándola en cero. Al final, nuestro programa será como el que se observa en el código \ref{cod-gauss}. \\

\lstinputlisting[language=Python,caption=La suma de Gauss en Python,label=cod-gauss]{./py/imperativa/gauss.py}

\subsubsection{Ejecución en la máquina RAM}

Vamos a ver cómo se modifica la memoria de la máquina hasta dar con la respuesta. La principal razón para no repetir este tipo de cosas es que \textbf{el programador debe aprender a leer el código por lo que significa y no por la forma en que la máquina lo ejecuta}; de otra manera nada ganaríamos con tener procesadores cada vez más rápidos, si a la hora de programar tenemos que repasar como ejecuta el computador las miles de iteraciones de nuestros programas. Por otro lado, el lector debe tener completamente claro lo que acabamos de programar, para que el comportamiento del resto de programas que veremos le resulte mucho más intuitivo.

Hecha esta aclaración, hagamos el seguimiento de la ejecución del programa:

\begin{itemize}
\item Se inicia la variable $suma$ en 0.
\item Se entra al ciclo en cuál la $i$ arranca en 1 y termina en 100.
	\begin{itemize}
	\item En la primera iteración $suma=0+1$, luego, $suma = 1$
	\item En la segunda iteración $suma=1+2$, luego, $suma = 3$
	\item En la tercera iteración $suma=3+3$, luego, $suma = 6$
	\item En la cuarta iteración $suma=6+4$, luego, $suma = 10$
	\item En la quinta iteración $suma=10+5$, luego, $suma = 15$
	\item ...
	\item En la penúltima iteración $suma=4851+99$, luego, $suma = 4950$
	\item En la última iteración $suma=4950+100$, luego, $suma = 50501$
	\end{itemize}
\item Cuando termina el ciclo se imprime el resultado $5050$
\end{itemize}

Como usted puede ver, el análisis previo es exactamente lo que hace el programa. Al principio puede resultar un poco difícil, pero usted tiene que ganar confianza como programador para poder enfrentar cada vez retos más difíciles y apasionantes.

\begin{quote}
\emph{La visualización (...) remarcaba precisamente aquello que el estudiante tiene que aprender a ignorar, reforzaba precisamente lo que el estudiante tiene que desaprender}.
Profesor Edsger Dijsktra, en su ensayo sobre la crueldad de enseñar realmente ciencias de la computación.
\end{quote}

\subsubsection{Solución matemática}

Aunque nuestro programa resultó funcionar muy bien, es claro que Gauss no pudo contar un computador en su época. Así que veamos cómo lo hizo.

Gauss descubrió una fórmula que permite hallar fácilmente la sumatoria de los $n$ primeros números:

\begin{equation}
1+2+3+\cdots+n = \frac{n(n+1)}{2}
\end{equation}

Si reemplazamos la $n$ por 100 tenemos lo siguiente:

\begin{equation}
1+2+3+\cdots+100 = \frac{100(100+1)}{2}
\end{equation}

Que efectivamente es 5050. ¿Pero cómo intuyó esto Gauss? La respuesta la encontramos si organizamos los números como se ve en la figura \ref{figgauss}.

\begin{figure}[h!]
\centering
\begin{tabular}{| c | c | c | c | c | c | c |}
\hline
1 & 2 & 3 & $\cdots$ & (n-2) & (n+1) & n \\ \hline
n & (n-1) & (n-2) & $\cdots$ & 3 & 2 & 1 \\ \hline
\end{tabular}
\caption{Suma de los primeros $n$ números}
\label{figgauss}
\end{figure}

Es decir, tomamos la serie de forma ascendente y luego de forma descendente. Ahora ¿qué pasa si sumamos verticalmente? En la primera obtenemos $n+1$, en la segunda $2+(n-1)=n+1$, y si seguimos, veremos que en todas obtenemos $n+1$. Como teníamos $n$ números, tenemos $n$ veces $n+1$ que es lo mismo que $n \cdot (n+1)$. Finalmente, la respuesta la dividimos entre dos, porque tomamos la serie dos veces, llegando a la respuesta \cite{graham1994concrete}.

La idea de sumar dos veces la serie resultó bastante útil, y nuestro programa se podría reducir a una línea -imprimir la operación. Las soluciones matemáticas suelen ser mucho más elegantes, rápidas y poderosas que las soluciones a código puro y duro, sin embargo no todo puede resolverse en el computador con una fórmula tan sencilla. Generalmente, se deben mezclar ambos enfoques para resolver problemas más complejos.



\subsubsection{Del papel al computador}

Es necesario aclarar, que para pasar la fórmula anterior al computador hay que tener especial cuidado en el orden en el que la máquina ejecuta las operaciones. A saber:

	\begin{enumerate}
	\item Se resuelve lo que esté dentro de paréntesis \textbf{()}.
	\item Se resuelven las potencias \textbf{**}.
	\item Se resuelven las multiplicaciones, divisiones y residuos  \textbf{* / \%}. 
	\item Se resuelven las sumas y restas  \textbf{+} y \textbf{-}.
	\end{enumerate}
	
Así que, para que el resultado sea el esperado, es necesario utilizar signos de agrupación como se ve en el código \ref{cod-gauss-formula}. \\


\lstinputlisting[language=Python,caption=Las sumas de Gauss.,label=cod-gauss-formula]{./py/imperativa/gauss-formula.py}

\subsection{Los números primos}

Los números primos son muy importantes en la computación, ya que la criptografía actual esta basada principalmente en ellos. Sin embargo, los primos han fascinado a la gente desde tiempos inmemoriales. Por ejemplo, los griegos demostraron que existen infinitos números primeros \cite{discretemath}. Para entender qué son estos números, debemos definir el concepto de \emph{divisivilidad}.

\subsubsection{Divisibilidad}

Sean $a$ y $b$ dos enteros cualquiera. Decimos que $a$ es divisible entre $b$ si al realizar la división $a/b$ el residuo es cero.

Por ejemplo, 4 es divisible entre 2, porque al realizar la división 4/2 el residuo es cero; 5 no es divisible entre 2, porque al realizar la división 5/2 el residuo es uno.

Para hallar el residuo de una división en el computador, utilizamos el operador módulo (\%), como se observa en el código \ref{cod-mod}. \\

\lstinputlisting[language=Python,caption=Hallando el módulo o residuo de una división.,label=cod-mod]{./py/imperativa/codmod.py}



\subsubsection{Números primos}

Un número entero $p$ mayor que uno, es  primo si solamente es divisible por $1$ y por $p$.  

Por ejemplo, 2 es primo porque sólo lo divisible entre 1 y 2, pero 4 no es primo, porque es divisible entre 1, 2 y 4. Los siguientes son los primeros 100 números primos:

2 3 5 7 11 13 17 19 23 29 31 37 41 43 47 53 59 61 67 71 73 79 83 89 97 101 103 107 109 113 127 131 137 139 149 151 157 163 167 173 179 181 191 193 197 199 211 223 227 229 233 239 241 251 257 263 269 271 277 281 283 293 307 311 313 317 331 337 347 349 353 359 367 373 379 383 389 397 401 409 419 421 431 433 439 443 449 457 461 463 467 479 487 491 499 503 509 521 523 541 

Conociendo los conceptos matemáticos, nuestra misión será crear un programa de computador que determine la cantidad de divisores de un número y posteriormente indique si es primo o no (en el próximo capítulo veremos cómo podemos hacer está última tarea más rápidamente).

\subsubsection{Solución}

La idea es simple: nos dan un número $n$, contamos cuántos divisores tiene, si al final tiene sólo dos divisores (1 y $n$) entonces es primo.

Si recordamos, todos los números son divisibles entre uno y entre el número mismo. Así que estas dos verificaciones son triviales y podemos omitirlas. Tenemos que centrarnos en determinar si algún otro número es divisor. Pero la pregunta es, ¿cuántos números diferentes debemos chequear? hay infinitos números, y nuestro programa no terminaría nunca de revisarlos todos. De hecho, en algún punto se bloquearía, porque la memoria de la máquina es finita. 

Luego, tenemos que determinar con cuidado qué números necesitamos verificar. Eso es algo fácil si lo pensamos así: ningún número es divisible por uno mayor que él. Por ejemplo, 3 no es divisible entre 5 (el 3 en el 5 está cero veces y sobran 3), dicho de otra forma, es imposible dividir el 3 en 5 partes enteras.

Dicho esto, bastará con \textbf{contar} qué números mayores que uno, pero menores que $n$, son divisores del número. Para contar en un programa de computador, se utiliza un mecanismo similar al del ejercicio anterior: definimos una variable, la iniciamos en algún valor (usualmente cero) y cada vez que se cumpla cierta condición le sumamos +1. Es como tener un papel en el que realizamos una raya cada vez que queremos contar algo.

Para nuestro caso, esa variable la llamaremos $divisores$, la iniciaremos en 2 (recuerde que todos los números tienen de entrada al uno y a él mismo como divisores) y cada vez que encontremos un divisor ejecutaremos la instrucción:
\begin{equation}
divisores = divisores + 1
\end{equation}

En el código \ref{cod-primos} puede ver el programa completo. Sólo falta agregar, que al final es necesario verificar que el número no sea uno, ya que por convención el uno no es primo. \\

\newpage

 \lstinputlisting[language=Python,caption=Divisibilidad y números primos.,label=cod-primos]{./py/imperativa/primos.py}



\subsection{Las letras}


Supongamos que van a ingresarnos $n$ palabras. La idea es contar cuántas palabras consecutivas son iguales. Por ejemplo:

\begin{itemize}
	\item Si $n$ es 5, y letras, ingresadas en orden son:
	\item A
	\item A
	\item B
	\item A
	\item A
	\item La respuesta es dos. Porque se ingresaron dos As consecutivas.
\end{itemize}



\subsubsection{Solución}

Para este problema sólo basta tener en cuenta si la nueva letra es diferente al anterior, en tal caso se incrementa el número de grupos.

Para programarlo, debemos crear un espacio en memoria en el que podamos hacer seguimiento a la letra \emph{anterior}. Al terminar de leer cada letra y mirar si se forma un grupo, debe actualizarse el anterior, para efectos de tener los datos preparados para la siguiente iteración.

La única parte que es menos intuitiva, es la forma en que haremos que al poner al principio la variable \emph{anterior}. Todo esto se puede ver en el código \ref{cod-magnets}. \\

 \lstinputlisting[language=Python,caption=Solución del problema letras.,label=cod-magnets]{./py/imperativa/letras.py}
 
 
\newpage
\section{Problemas}
En este capítulo hemos aprendido lo básico de la programación y de las soluciones imperativa: cómo sumar elementos, cómo contar elementos y vigilar un elemento anterior. \\

Mediante los problemas propuestos usted podrá auto-evaluarse y poner a prueba sus conocimientos. Para saber si su solución está correcta, piense en algunas entradas y las salidas respectivas a cada una de ellas, luego ejecute su programa y valide si el resultado fue correcto. Si no es así, piense ¿qué puede estar fallando?

\vfill

\begin{Exercise}[title={Calentamiento}]
Realice un programa en Python para resolver los siguientes problemas:
\begin{enumerate}
\item Se va a recibir un número, se garantiza que será 0 o 1. En caso de recibir un 0 se debe imprimir 1, en caso de recibir 1 se debe imprimir 0. No se puede utilizar la instrucción \emph{if}.

\item ¿Cómo realizar una multiplicación sin usar el signo por (*)? Encuentre dos soluciones diferentes para éste problema.

\item ¿Cómo realizar una potencia sin usar el signo de potenciación (**)?
\end{enumerate}
\end{Exercise}
\begin{Answer}

\lstinputlisting[language=Python,caption=Solución de los problemas de calentamiento.,label=cod-calentamiento]{./py/imperativa/calentamiento.py}
\newpage
\end{Answer}
\vfill
\begin{Exercise}[title={La Copia de Seguridad}]
Bill quiere guardar sus documentos que pesan $x$ gigas en unos discos que tienen $q$ gigas de capacidad. Por ejemplo, si tiene que guardar 6 gigas en discos de 4 gigas, necesitará 2 discos (aunque el segundo lo llene sólo parcialmente).

¿Podrá usted ayudar a Bill construyendo un programa para que él sepa cuántos discos debe comprar? Lea por teclado las variables $x$ y $q$, para calcular la respuesta.

Nota: recuerde que los códigos deben funcionar para cualquier entrada, no sólo para las que aquí se proponen. Asuma de momento que siempre le van a dar datos válidos.

\end{Exercise}

\newpage
\begin{Answer}
 \lstinputlisting[language=Python,caption=Solución del problema La Copia de Seguridad.,label=cod-seguridad]{./py/imperativa/copiaseguridad.py}
 
\end{Answer}

\begin{Exercise}[title={El Precio del Videojuego}]
	Joe Cortana tiene una tienda de empeños para videojugadores. Todos los días entran a su tienda cientos de vendedores que quieren cambiar sus antiguos videojuegos por dinero. 
	
	Joe es muy cuidadoso y ha estudiado muy bien como se comporta el mercado. Ha descubierto que en las mañanas llegan los mejores juegos, de esos clásicos que son muy apetecidos por los coleccionistas. A medida que pasan las horas, cada vez recibe juegos más malos, terminando en ET (considerado el peor juego de la historia).
	
	Joe Cortana tiene disponibles $x$ dólares todos los días. Su misión es decirle, dada una lista de $n$ videojuegos con sus precios (todo esto se debe leer por teclado), cuántos títulos alcanza a comprar, de forma que esté adquiriendo siempre lo mejor que vaya entrando a su tienda.	
	
	Por ejemplo: si tiene 10 dólares, y van a ofrecerle 3 juegos que valen respectivamente 6, 4 y 3 dólares, él compra dos juegos.
	
	\textbf{Explicación}: 
	\begin{itemize}
		\item El primero vale 6, y lo compra, quedándole 4 dólares disponibles.
		
		\item El segundo vale 4 dólares, de forma que lo compra y no le queda nada.
		
		\item El tercero vale sólo un dólar, pero ya no tiene dinero. 
		
		\item En definitiva, alcanza a comprar 2 juegos.
	\end{itemize}
	
	Nota: recuerde que los códigos deben funcionar para cualquier entrada, no sólo para las que aquí se proponen. Asuma de momento que siempre le van a dar datos válidos.
	
\end{Exercise}
\begin{Answer}
\lstinputlisting[language=Python,caption=Solución el Precio del Videojuego.,label=cod-vjprecio]{./py/imperativa/vjprecio.py}
\newpage
\end{Answer}


\newpage

%\chapter{Programas más complejos}

En este capítulo veremos un par de herramientas adicionales, para finalmente poner en práctica todo lo aprendido en la programación de un juego en Python. La idea es también construir una estrategia que le permita al ordenador jugar contra una persona.

\section{Otras formas de repetir}


Utilizando un \emph{for} podemos iterar un número de veces determinado (de acuerdo al rango que definamos), sin embargo en ocasiones no sabemos de antemano cuántas veces debemos repetir, sólo sabemos que debemos hacerlo mientras se cumpla una condición en el estado de la máquina. En esos casos utilizamos el ciclo \textbf{while}, que traduce \emph{mientras} en español. El \textbf{while} tiene la siguiente sintaxis:

\begin{tabular}{l l}
	while & \textbf{condición}: \\
	& \textbf{instrucciones a repetir} \\
\end{tabular} \\

Por ejemplo, podemos tener el mismo comportamiento que un \emph{for}, como se ve en código \ref{codwhilefor} (aunque es mejor utilizar un \emph{for} para estos casos, por obvias razones). 

\newpage

\lstinputlisting[language=Python,caption=While vs For,label=codwhilefor]{./py/cap3/whilefor.py}

A continuación veremos un caso en el que sí se debe usar el \emph{while}.

\subsection{El algoritmo de Euclides}

El algoritmo de Euclides permite computar el \textbf{Máximo Común Divisor} (MCD) entre dos enteros positivos. El MCD está definido como el entero positivo más grande que divide a ambos números \cite{discretemath}. \\

Sean $a$ y $b$ dos enteros positivos, para hallar su MCD esto es lo que debemos realizar: 

\begin{enumerate}
	\item Si el residuo de la división entre $a$ y $b$ es cero, entonces $b$ es el MCD.
	\item Si el residuo de la división entre $a$ y $b$ es un entero $r$ diferente de 0, entonces el MCD entre $a$ y $b$ es el MCD entre $b$ y $r$.
\end{enumerate}



Por ejemplo, para hallar el MCD entre 18 y 12:

\begin{itemize}
	\item El residuo entre 18 y 12 es 6, luego toca aplicar la segunda propiedad.
	\item Al aplicarla, operamos con 12 y 6, cuyo residuo es 0. Luego, el MCD(18,12) es 6. 	
\end{itemize}

Ahora, vamos a programar el algoritmo. Tendremos en cuenta las propiedades, pero haremos un pequeño ajuste para que nos sea más fácil programarlo: iteraremos una vez de más, de forma que el residuo quede cargado en la variable $b$, como se puede ver en la tabla \ref{euclidej}. Luego, la condición para iterar es que $b$ sea diferente de cero, cuando llegue a cero es porque hemos encontrado en MCD (que estará en la variable $a$).

\newpage

\begin{figure}[h!]
	\centering
\begin{tabular}{| p{1cm} | p{1cm} | p{1cm} |}\hline
	\textbf{a} & 	\textbf{b} & 	\textbf{r} \\ \hline
	18 & 12 & 6 \\ 
	12 & 6 & 0 \\ 
	6 & 0 & FIN \\ \hline
\end{tabular}

\caption{Aplicando el algoritmo de Euclides para 18 y 12}
\label{euclidej}
\end{figure}

Para que los valores sobre los que operamos cambien en cada iteración, basta con intercambiar el valor que hay en $a$ por el de $b$ y el de $b$ por el del residuo que acabamos de calcular. Finalmente, el código \ref{codeuclid} expone lo que acabamos de describir. \\

\lstinputlisting[language=Python,caption=El algoritmo de Euclides en Python,label=codeuclid]{./py/cap3/euclid.py}

¿Por qué no usamos un \emph{for}? Porque no podemos saber de antemano cuántas veces vamos a repetir, dependiendo del valor que arroje el residuo debemos seguir repitiendo los pasos del algoritmo, o por el contrario sabemos si hemos encontrado la respuesta correcta.

\subsection{Repitiendo por siempre}

Una de las formas más comunes de utilizar el \emph{while} es poniendo un \textbf{True} en la condición, de forma que se repita por siempre. El \textbf{While True} requiere que se utilice un retorno (en caso de estar en un procedimiento) o la instrucción \textbf{break} (que traduce romper), para indicarle al programa cuándo salirse del ciclo y continuar ejecutando el resto del programa. Tiene la siguiente sintaxis: \\

\begin{tabular}{l l}
	while & True:  \\
	& \textbf{instrucciones a repetir}  \\
	& if \textbf{condición de salida}:  \\
	&    \hspace{1cm} break \\
\end{tabular} \\

Veámos cómo funciona esto con un ejemplo. Suponga que necestiamos realizar un programa que obligue a la persona a digitar un número positivo, es posible realizarlo de dos formas: usando un while o usando un while true, como se puede observar en el código \ref{codwhiletrue}. \\

\lstinputlisting[language=Python,caption=While True VS While,label=codwhiletrue]{./py/cap3/whiletrue.py}

Aunque ambos cumplen con la misma función, con el \emph{While True} se evita la asignación de un valor temporal para entrar en el ciclo, el cuál nada tiene que ver con el objetivo real del programa.

\section{Depurando los programas}

La mayoría de las veces, los programas que construimos tienen errores lógicos o \emph{bugs}. Para eliminarlos y corregir el programa es necesario identificarlos, algo que en ocasiones es una tarea nada trivial. \\

El proceso de arreglar programas con errores se conoce como \emph{depuración} \cite{evansIntro}. Las siguientes son recomendaciones útiles que usted puede seguir para corregir sus programas:

\begin{enumerate}
	\item No le pida ayuda a su profesor o a un programador. Puede sonar extraño, pero cuando usted se acostumbra a que otros corrijan sus programas se hace un daño terrible: usted aprende más corrigiendo un programa que no funciona, que programando un código que falla. 
	
	\item Aprenda a utilizar el \emph{depurador} que la mayoría de entornos de programación contienen, por lo menos aprenda a poner \emph{prints} en el código para identificar dónde está el error. Esto es, use \emph{prints} para imprimir los valores que tienen las variables que representan el estado de la máquina, revise si estos valores son correctos, si no, piense qué podría estar ocasionando el error.
	
	\item Identifique el procedimiento que está produciendo el error. Si tiene todo el programa en un solo \emph{mega-algoritmo}, piense mejor en empezar de cero.
\end{enumerate}

Algunas sugerencias que hace David Evans en su libro Introduction to Computing \cite{evansIntro}, son:

\begin{enumerate}
	\item Asegúrese de entender el comportamiento para el que está destinado su procedimiento. Piense en algunas entradas representativas y las salidas esperadas para esas entradas.
	
	\item Haga experimentos, observe el comportamiento actual de su procedimiento. Pruebe el programa con entradas pequeñas. ¿Cuál es la relación entre las salidas actuales y las esperadas? ¿Funciona correctamente para algunas entradas pero no para otras? 

	\item Haga cambios en su procedimiento y vuelva a probarlo. Si usted no está seguro de qué hacer, haga cambios en pequeños pasos y cuidadosamente observe el impacto de cada cambio.
\end{enumerate}

Dicho todo lo anterior, empecemos a ver algunas cosas aplicaciones interesantes para todo lo aprendido.

\newpage

\section{Aplicando lo aprendido}

Vamos a empezar programando un juego corto. El primer paso será dividir el problema en varias partes, ya que es más fácil solucionar pequeños pedazos del juego que intentar resolverlo todo al mismo tiempo. Al final, ensamblaremos los fragmentos de código resultantes, de forma que podamos construir un programa completamente funcional.

\subsection{Thai 21}

Thai 21 es una variante del juego Nim. Distintas versiones de este juego han sido practicadas a lo largo de muchos siglos, pero nadie sabe realmente de donde proviene el nombre \cite{evansIntro}.

El juego consiste en una pila que contiene 21 rocas. Dos jugadores se enfrentan tomando una, dos o tres en cada turno. Al final gana quién tome la última de las piedras. \\

Como dijimos anteriormente, empezaremos por separar el juego en varias partes, esta es la propuesta:

\begin{enumerate}
	\item Definir cómo representar los datos: responder a esta pregunta ¿cómo representar cada uno de los elementos del juego en Python?
	\item Lectura de datos: tenemos que pedirle a los jugadores cuántas rocas tomar, pero deben cumplirse dos condiciones:
	\begin{itemize}
		\item Que haya suficientes rocas.
		\item Que intente tomar una, dos o tres rocas.
	\end{itemize}
	
	\item Definir un estado inicial: determinar qué jugador empieza y configurar la pila de las rocas.
	
	\item El ciclo del juego (gameloop): pedir los datos (usando el código que realicemos en el paso 2), quitar las rocas de la pila, ver si alguien gano o si se cambia de jugador.
\end{enumerate}

\subsubsection{Representación de los datos}
Vamos a tener dos elementos en el juego: las rocas y los jugadores. Los representaremos así en el programa:

\begin{itemize}
	\item Para las rocas podemos utilizar un entero que nos indique cuántas piedras siguen en el juego.
	
	\item Para los jugadores, simplemente necesitamos saber si es el turno del jugador 1 o del jugador 2, para lo cuál usaremos un \emph{bool}, que represente con True al jugador 1 y con False al jugador 2.
\end{itemize}


\subsubsection{Lectura de los datos}
Definiremos un procedimiento, que luego pueda ser llamado por el algoritmo principal, en el cuál sólo retornaremos el número de rocas que elija el jugador si y sólo si, dicho número es un valor válido para el estado actual del juego.

El procedimiento necesita recibir como entrada el número de rocas disponibles, para determinar que el jugador no está intentando tomar más de las que hay en la pila. 

Dicho procedimiento lo llamaremos \emph{pedir\textunderscore jugada}, como se ve en el código \ref{codthailectura}. \\

\lstinputlisting[language=Python,caption=Lectura de datos para Thai 21,label=codthailectura]{./py/thai21/lectura.py}

\subsubsection{Estado Inicial}
Lo pondremos en el algoritmo principal. La idea es indicarle al computador que empieza el jugador 1 que hay 21 rocas en la pila, como se ve en el código \ref{codthaiinic}. \\

\lstinputlisting[language=Python,caption=Inicialización del programa para Thai21,label=codthaiinic]{./py/thai21/inic.py}

\subsubsection{Gameloop}
Cómo no sabemos cuántos turnos son, el gameloop será un \emph{While True}, que terminará cuando no queden rocas. 

Los pasos a realizar serán lo que especificamos anteriormente: pedir los datos (usando el procedimiento que ya definimos), quitar las rocas de la pila (una resta), ver si alguien ganó (si no quedan rocas) o si se cambia de jugador (como es jugador actual está representado por un \emph{bool}, bastará con hacer una negación: si está en verdadero pasará a falso, o viceversa). El gameloop se puede ver en el código \ref{codthaigameloop}. \\

\lstinputlisting[language=Python,caption=Gameloop del Thai21 en Python,label=codthaigameloop]{./py/thai21/gameloop.py}

\subsubsection{Programa final}

Ya que cada parte está lista, podemos ver el programa completo en el código \ref{codthai}. En dicho código se añadió un print para mostrar cuántas rocas quedan en la pila. \\

\lstinputlisting[language=Python,caption=Thai21 en Python,label=codthai]{./py/thai21/thai.py}

\newpage

Para terminar este capítulo, construiremos una estrategia de victoria para que el computador juegue contra una persona. La idea es la siguiente: ¿es posible crear una estrategia que permita al computador ganar cualquier partida sin importar qué movimientos realice el otro jugador? Asuma que siempre el computador hará la primera jugada. 

\textbf{Piense unos minutos cómo podría ser dicha estrategia, antes de continuar leyendo}.

\newpage

\begin{figure}[h!]
	\centering
	\includegraphics[width=3cm]{./Images/thai21.png}
	\caption{Posición de victoria para el jugador 1 (rojo).}
	\label{figthaiA}
\end{figure}

\subsection{Estrategia de victoria}
La idea es llegar a un punto en dónde, sin importar qué haga, el segundo jugador no pueda ganar. Por ejemplo, si quedan 4 piedras, no importa si el jugador 2 toma una, dos o tres piedras, en cualquier caso, el jugador 1 puede ganar (ver figura \ref{figthaiA}).

\begin{itemize}
\item Si el jugador 2 toma una piedra, quedan tres piedras. El jugador 1 toma las tres piedras y gana.
\item Si el jugador 2 toma dos piedra, quedan dos piedras. El jugador 1 toma las dos piedras y gana.
\item Si el jugador 2 toma tres piedra, queda una piedras. El jugador 1 toma la única piedra y gana.
\end{itemize}

Dicho esto, tenemos que obligar al jugador 2 a que caiga en cuatro, ¿desde cuáles se puede llegar a cuatro? Si es el turno del jugador 1, puede hacerlo desde cinco, seis y siete. Luego, el jugador 2 debe quedar en ocho, ya que sin importar qué haga, quedaremos en alguno de dichos números, como se puede observar en la figura \ref{figthaiB}. 


\begin{figure}[h!]
	\centering
	\includegraphics[width=6cm]{./Images/thai21b.png}	
	\caption{Otra posición clave en el juego.}
	\label{figthaiB}	
\end{figure}

Siguiendo la misma idea, podemos devolvernos. A 8 se llega desde 9, 10 y 11. Luego el jugador 2 debe quedar en 12, pero antes en 16 y antes en 20. \\

Si usted observa bien los números clave son múltiplos de 4 (20, 16, 12, 8 y 4), de forma que el número de rocas a tomar, en cada turno debe ser:

\begin{equation}
JugadaPC = Rocas \% 4
\end{equation}

Así, si al empezar hay 21 rocas: $21\%4=1$. Tomamos una roca y el otro queda en una posición desde la cuál sin importa qué haga, podemos obligarlo, con la misma fórmula, a que caiga en 16, luego en 12, más tarde en 8 y finalmente en 4.

Para adaptar el código que tenemos hay que modificar el procedimiento que pide la jugada (ya que uno de los jugadores va a ser el computador). Vamos a recibir otro parámetro que será el jugador actual (obviamente habrá que enviarlo desde el algoritmo principal).

El resultado final puede verse en el código \ref{codthaifull}.
\newpage
\lstinputlisting[language=Python,caption=Thai21 humano VS computador,label=codthaifull]{./py/thai21/thaifull.py}
\newpage

\section{Problemas}
\begin{Exercise}[title={Calentamiento}]	
	\begin{enumerate}
		\item Entre a \url{http://codecombat.com/} para practicar sus habilidades como programador (y aprender cosas nuevas).
		
		\item Modifique el último código de Thai21 para que empiece un jugador al azar (no siempre el computador). Es posible que en ocasiones no pueda aplicar la estrategia de victoria. \\
		
		Cuando necesite generar valores al azar:
		
		\begin{enumerate}
			\item Ponga en la primera línea del programa: \textbf{import random}. Esto le permitirá usar la librería de Python para generar números \emph{Pseudo-aleatorios}.
			
			\item Utilice la función:  \textbf{random.randint(a,b)}, para generar números al azar (el número generado, llamémoslo $X$ estará en el rango $a \leq X \leq b$). 
			
			\item Ejemplo:
			\lstinputlisting[language=Python,caption=Números aleatorios en Python,label=codrandomn]{./py/thai21/randomn.py}
		\end{enumerate}
		
		Otra función que puede serle útil es \textbf{min(a, b)} que recibe dos números y le retorna el menor de los dos (para evitar escribir el if y el else, como en el segundo capítulo).
	\end{enumerate}
\end{Exercise}
\begin{Answer}	
	\lstinputlisting[language=Python,caption=Thai21 definitivo.,label=codthai21def]{./py/thai21/thaicalentamiento.py}
	\newpage
\end{Answer}

\newpage

\begin{Exercise}[title={El juego del granjero}]	
	Hace mucho tiempo, en un mundo alterno, un granjero fue al mercado del pueblo más cercano y compró: un lobo, una cabra y una col.
	
	Para volver a su casa tenía que cruzar un río. El granjero disponía de una barca para llegar a la otra orilla, pero en la barca solo cabían él y una de sus compras. Por supuesto, tenía ciertas restricciones:
	\begin{itemize}
		\item Si el lobo se quedaba solo con la cabra se la comía.
		\item Si la cabra se quedaba sola con la col se la comía.
		\item El granjero siempre debía ir en el bote (para manejarlo).
	\end{itemize}

	Sabemos que al final del día el granjero logro cruzar con todas sus compras, la pregunta es ¿cómo lo hizo?
	
	Elabore un programa en Python que permita a un jugador escoger quién irá en cada
	viaje con el granjero y que valide de acuerdo a las reglas si en algún momento algo va mal o si se llega al final del juego.
	
	Imagine cómo podría pintar el escenario del juego usando prints.
	 
\end{Exercise}
\begin{Answer}	
	\lstinputlisting[language=Python,caption=El juego del granjero,label=codlccol]{./py/soluciones/lobocabracol.py}
	\newpage
\end{Answer}

%\chapter{Estructuras de datos básicas}

Las estructuras de datos son abstracciones que permiten trabajar datos que tienen algún orden o jerarquía.


\section{Ciclos anidados}
\section{Strings}
\section{Listas}
\section{Listas de listas}
\section{Usando las estructuras}
\subsection{Recorridos}
\subsection{Aliasing y mutabilidad}
\subsection{Funciones útiles}
\section{Aplicaciones}
\subsection{El primer parser}
\subsection{El juego del ahorcado}
\section{Problemas}

\chapter{Algo de CS}
\section{Noción de costo computacional}
\section{Recursividad}

\chapter{Un poco de estructuras de datos}
\section{Diccionarios}
\section{Grafos}
\subsection{Recorridos}
\section{Aplicación}
\subsection{Construyendo un Web Crawler}
\section{Problemas}

\chapter{Programación Orientada a Objetos}
\subsection{Clases y objetos}
\subsection{Herencia y polimorfismo}
\subsection{Creando un videojuego}

\chapter{Programando un motor 3D}

En este capítulo nos aventuraremos a desarrollar un pequeño motor tridimensional. No construiremos una pieza optimizada de alta tecnología pero aprenderemos algunos de los conceptos que permitieron en el año 1992 a una empresa llamada \emph{ID Software} crear uno de los juegos más emblemáticos de la historia, \textbf{Wolfenstein 3D}.

¿Preparados para poner su cerebro a prueba?

\begin{figure}[h!]
	\centering
	\includegraphics[width=10cm]{./Images/wolf3d.jpg}
	\caption{Wolfenstein uno de los primeros juegos en usar motores 3D de forma efectiva. \emph{La imagen es propiedad de sus respectivos dueños, usada únicamente con fines educativos}.}
	\label{wolf3d}
\end{figure}

\section{Contexto histórico}

\begin{quote}
	\emph{Tenemos esta oportunidad de hacer algo totalmente nuevo aquí, algo rápido y con texturas. Si podemos hacer que los gráficos se vean fantásticos y rápidos, y hacer el sonido fresco y fuerte, y hacer el juego explosivamente divertido, entonces tendremos algo ganador.} \\
	John Romero, en Masters of Doom \cite[p.~92]{doom}.
\end{quote}


\begin{figure}[h!]
	\centering
	\includegraphics[width=10cm]{./Images/keen.png}
	\caption{Keen Commander, anterior a Wolfenstein, también de ID Software, uno de los primeros juegos de plataformas en aprovechar toda la potencia de los ordenadores de la época. \emph{La imagen es propiedad de sus respectivos dueños, usada únicamente con fines educativos}.}
	\label{keen}
\end{figure}

Corría el año de 1992, la mayoría de computadores apenas si tenían capacidad para ejecutar de forma fluida algunos juegos en 2D. \emph{}.

%\begin{figure}[h!]
%	\centering
%	\includegraphics[width=10cm]{./Images/wolf3d-juego.jpg}
%	\caption{Gráficos dentro del juego Wolfenstein 3D. \emph{La imagen es propiedad de sus respectivos dueños, usada únicamente con fines educativos}.}
%	\label{wolf3d-juego}
%\end{figure}

\newpage

\section{Simulando gráficos en 3D}


Dada la poca potencia de los computadores de la época, tener verdaderos gráficos en 3D era una tarea imposible, entonces, ¿cómo hicieron en ID para crear Wolfenstein 3D? La respuesta es que lo que realmente estamos viendo es \emph{una ilusión}.

\begin{figure}[h!]
	\centering
	\includegraphics[width=10cm]{./Images/wolf3d-mapa.png}
	\caption{El mapa siempre fue una cuadrícula bidimensional. El jugador es el punto rojo y el triángulo amarillo refleja el campo de visión.}
	\label{wolf3d-mapa}
\end{figure}

Como se puede apreciar en la figura \ref{wolf3d-mapa}, toda la lógica del juego ocurría en una cuadrícula; la magia era mostrar los laberintos y enemigos cómo si fueran parte de un entorno tridimensional, pero la verdad, nunca dejaron de ser 2D.

Esto traía muchas ventajas a nivel de rendimiento, sin embargo venía con algunas limitaciones, como por ejemplo que al ser realmente un mapa 2D no podían tenerse cuartos unos encima de otros. Aún así, esta estrategia no era suficiente en sí misma, los procesadores de aquella época era muy limitados y para mantener una tasa de refresco lo suficientemente alta, fueron necesarias otras simplificaciones y optimizaciones, entre ellas: que los muros fueran todos en ángulos de $90^{\circ}$, que los mapas siguieran una estricta alineación a una grilla, que el techo y el piso fueran colores planos sin texturas de ningún tipo.

Dicho esto, exploraremos un algoritmo simplificado para realizar el pintado de los escenarios.

\newpage
\subsection{Raycasting}

\begin{wrapfigure}{r}{0.4\textwidth}
	\begin{center}
		\includegraphics[width=0.25\textwidth]{./Images/rays.png}
	\end{center}	
	\caption{Rayos lanzados en el campo de visión del jugador para calcular la distancia de los muros. \newline}
	\label{rays}
	
	\begin{center}
		\includegraphics[width=0.4\textwidth]{./Images/raytracing.png}
	\end{center}	
	\caption{Pintado de los muros de acuerdo a la distancia. Al final se genera la ilusión de un espacio en 3D. \newline}
	\label{raytracing}
	
	\begin{center}
		\includegraphics[width=0.4\textwidth]{./Images/wolf3d-juego.jpg}
	\end{center}
	\caption{Gráficos dentro del Wolfenstein 3D.}
	\label{wolf3d-mapa}

\end{wrapfigure}

El \textbf{Raycasting} es una técnica que consiste en lanzar rayos para determinar cuál es la primer superficie que estos interceptan. El motor de Wolfenstein basaba su renderizado en está técnica. La idea general es la siguiente:

\begin{enumerate}
	\item De acuerdo al campo de visión del jugador empezando desde la izquierda y hasta recorrer todo el cono, se lanzarían unos \emph{rayos} verificando a qué distancia se encuentran los muros. Ver figura \ref{rays}.
	
	\item A continuación, esas distancias se utilizan para pintar en la pantalla, de izquierda a derecha unas líneas verticales representando los muros. Entre mayor sea la distancia, más pequeña debe ser la línea (para dar la sensación de perspectiva). Ver figura \ref{raytracing}.
\end{enumerate}

Con está técnica es posible pintar cualquier mapa de Wolfenstein. Ahora, el algoritmo utilizado por ID tiene muchas optimizaciones y detalles adicionales para conseguir una velocidad de renderizado adecuada, así como permitir detalles que incluyen texturas en los muros, lámparas, objetos, enemigos, entre otros. Sin embargo, para nuestro empeño nos valdremos de esta versión simplificada.



\newpage
\clearpage 
\subsection{Algo de matemáticas}

Para poder implementar nuestra solución será necesario revisar un par de conceptos matemáticos. 

\subsubsection{Vectores}

Un vector es un objeto que tiene una magnitud y una dirección \cite{introVectors} \footnote{En español solemos decir que tienen módulo, dirección y sentido, pero nos ceñiremos a la definición anglosajona.}, gráficamente lo representamos con una flecha.
 
 \begin{figure}[h!]
 	\centering
 	\includegraphics[width=6cm]{./Images/vector.png}
 	\caption{Representación gráfica de un vector.}
 	\label{vector}
 \end{figure}

Con los vectores podemos representar por ejemplo la velocidad en la que se mueve un personaje: la magnitud indicaría la rapidez y la dirección indicaría hacia dónde se está moviendo. En el computador esto lo podríamos representar de la siguiente forma:

\begin{enumerate}
\item Un ángulo ($\theta$) indicando \emph{la dirección}.

\item Un número ($v$) indicando \emph{la rapidez}.
\end{enumerate}


\subsubsection{Vector unitario}

Los vectores unitarios son aquellos cuya magnitud es 1, luego podemos utilizarlo cuando necesitamos saber sólamente la dirección en la que apunta el vector.

\begin{figure}[h!]
	\centering
	\includegraphics[width=6cm]{./Images/unitaria.png}
	\caption{Circunferencia unitaria.}
	\label{unitaria}
\end{figure}


Cuando tratamos problemas de dos dimensiones, nos referimos a la grafica \ref{unitaria}. De donde podemos sacar, dado un ángulo $\theta$, las componentes en X e Y del vector unitario que representa dicho ángulo.

\begin{equation}
\begin{aligned}
\label{unitvector}
ComponenteX = sin \theta \\
ComponenteY = cos \theta
\end{aligned}
\end{equation}

En otras palabras, dado un ángulo, al aplicar el \emph{sin} y el \emph{cos}, podemos obtener un vector unitario que apunta en la misma dirección. En breve, veremos cómo utilizar este concepto para darle movimiento al jugador y para disparar los \emph{raycast}.

\subsubsection{Radianes}

Los radianes es una unidad utilizada para representar ángulos. En los lenguajes de programación es más común que se utilicen \emph{radianes} en vez de \emph{grados} para la manipulación y operación de ángulos, dicho esto, las funciones como \emph{sin} y \emph{cos} en Python esperan recibir ángulo en \emph{radianes}. 

\begin{figure}[h!]
	\centering
	\includegraphics[width=\textwidth]{./Images/radianes.png}
	\caption{¿Gráficamente qué son los radianes?}
	\label{rads}
\end{figure}

En la gráfica \ref{rads} vemos una intuición sobre lo que son los radianes. De la misma gráfica, es posible notar que $\pi$ radianes equivale a $180^{\circ}
$, luego $2\pi$ son $360^{\circ}$. Siguiendo la misma lógica, $\frac{\pi}{2}$ son  $90^{\circ}$.


Vistos todos los conceptos matemáticos necesarios, es hora de modelar realmente las fórmulas que nos permitirán construír nuestro \textbf{motor 3D}.

\subsection{Modelando el motor}

Para realizar el modelo matemático que dará forma a nuestro motor, iremos contruyendo las soluciones formalmente y al final de cada sección escribiremos el código correspondiente para cada caso.

Antes de continuar, cabe anotar que durante toda la ejecución tendremos acceso a los siguientes datos:

\begin{itemize}
	\item Posición del jugador: $jugador_x$ y $jugador_y$, dos números decimales.
	\item Dirección en la que está mirando el jugador, almacenada como un número decimal que representa ángulo en radianes: $jugador_{rot}$.
	\item La velocidad de movimiento y de rotación del jugador, almacenados como dos números decimales: $velocidad$,  $velocidad_{rot}$. 
	\item El campo de visión del jugador en radianes $FOV$ (Por sus siglas en inglés Field of View - Campo de Visión).
	\item La resolución de la pantalla. En especial el ancho $pantalla_{ancho}$.
\end{itemize}

Luego los cálculos que realicemos, estarán basados en dichos valores. Lo anterior podríamos codificarlo en Python como se ve en el código \ref{cod-motor-vars}.

\lstinputlisting[language=Python,caption=Variables generales del programa con algunos valores posibles.,label=cod-motor-vars]{./py/motor-3d/variables.py}

\subsubsection{El movimiento del jugador}

Cuando se presione la tecla arriba el jugador se moverá hacia adelante en la dirección en la que él está mirando. Luego, deberíamos obtener un vector unitario a partir de la variable $jugador_{rot}$ , multiplicarlo por la $velocidad$ a la que se desplaza el jugador y sumarlo a su posición actual. Aplicaríamos este concepto a cada componente ($x$ e $y$) como se puede ver en las ecuaciones \ref{movforw}. Para que camine hacia atrás, en vez de sumar, restamos. 

\begin{equation}
\begin{aligned}
\label{movforw}
jugador_x = jugador_x \pm velocidad * sin(jugador_{rot})  \\
jugador_y = jugador_y \pm velocidad * cos(jugador_{rot})
\end{aligned}
\end{equation}

Finalmente, para la rotación, bastaría con sumar o restar \emph{la velocidad de rotación} a \emph{la rotación actual del jugador}.

\begin{equation}
\begin{aligned}
\label{movforw}
jugador_{rot} = jugador_{rot} \pm velocidad_{rot}
\end{aligned}
\end{equation}

\lstinputlisting[language=Python,caption=Movimiento del personaje de acuerdo a las teclas presionadas.,label=cod-motor-mov]{./py/motor-3d/movimiento.py}



\subsubsection{El raycast}

En nuestro caso, para poder realizar los \emph{Raycast} necesitamos saber en qué dirección dispararlo (para lo cuál usaríamos un vector unitario) y qué tan lejos hacerlo (de momento usaremos una variable llamada $distancia$ ya veremos cómo calcularla). Una simplificación que podemos hacer, es que como sabemos que el mapa está perfectamente alineado a una cuadrícula podemos omitir los decimales, para lo cuál usaremos la función \emph{floor} (que aproxima un número decimal a su entero más cercano por debajo). Todo esto lo podemos ver en la ecuación \ref{eqray}.

\begin{equation}
\begin{aligned}
\label{eqray}
rayo_x = floor(x + sin \theta * distancia)\\
rayo_y = floor(y + cos \theta * distancia)\\
\end{aligned}
\end{equation}

Hablando de la $distancia$, como no la sabemos de antemano, lo que haremos es comenzar a calcularla, iniciando desde 0, e incrementándola poco a poco hasta un valor máximo o hasta que toquemos una pared (Ver imagen \ref{raycastit}). La idea de poner un límite máximo es optimizar un poco las cosas, si ya el rayo está muy grande y no hemos encontrado una pared, no queremos seguir buscando infinitamente, porque el juego se bloquearía.

\begin{figure}[h!]
	\centering
	\includegraphics[width=6cm]{./Images/raycast.png}
	\caption{Incrementando iterativamente la distancia del rayo hasta encontrar una pared o llegar a un límite máximo.}
	\label{raycastit}
\end{figure}


\subsubsection{Raycasts en el campo de visión}

El último cálculo que nos resta, es el de hallar los angulos de inicio y fin, desde los cuáles lanzaremos los rayos. Llamémoslos $FOV_{inicio}$ y $FOV_{fin}$ (FOV por las siglas en inglés de Field of View - Campo de Visión).

\begin{figure}[h!]
	\centering
	\includegraphics[width=6cm]{./Images/fov.png}
	\caption{El campo de visión del jugador.}
	\label{fovgraph}
\end{figure}

De la gráfica \ref{fovgraph}, podemos ver que el inicio del FOV se encuentra restando la mitad del FOV total del ángulo al que está mirando el jugador. Para el fin de FOV, sumamos la mitad del total del FOV al ángulo. Todo esto se ve en la ecuación \ref{foveq}. 

\begin{equation}
\begin{aligned}
\label{foveq}
FOV_{inicio} = \theta - \frac{FOV}{2}\\
FOV_{fin} = \theta + \frac{FOV}{2} \\
\end{aligned}
\end{equation}

A continuación, debemos tener en cuenta que el FOV es en definitiva lo que vamos a pintar en el ancho de la pantalla. Luego, debemos hallar la proporción entre el FOV y el ancho de la pantalla para saber cuánto espacio debería haber entre cada rayo, de forma que al final se cubra toda la pantalla.

\begin{equation}
\begin{aligned}
\label{foveq}
Rayo_{espaciado} = \frac{FOV}{Pantalla_{ancho}}
\end{aligned}
\end{equation}

Finalmente, deberíamos tirar un rayo en el equivalente a cada pixel horizontal de la pantalla empezando desde la $FOV_{inicio}$. Luego, debemos realizar la siguiente operación para $pixel$, desde 0 hasta el ancho de la pantalla:

\begin{equation}
\begin{aligned}
\label{foveq}
Rayo_{angulo} = FOV_{inicio} + pixel * Rayo_{espaciado} 
\end{aligned}
\end{equation}

Reemplazando:

\begin{equation}
\begin{aligned}
\label{foveq}
Rayo_{angulo} = \left(\theta - \frac{FOV}{2} + pixel * \frac{FOV}{Pantalla_{ancho}} \right)
\end{aligned}
\end{equation}



%\subsection{La secuencia de Fibonacci}

%\begin{minipage}{.70\textwidth} 
%Filius Bonacci era un monje y matemático italiano del siglo XII. Publicó un libro llamado %\emph{Liber Abbaci}, en el que trataba sobre la utilización de los números indo-arábigos, %que 
%\end{minipage}
%\begin{minipage}{.30\textwidth}
%  \centering
%  \includegraphics[height=5cm]{./Images/Fibonacci.jpg}
%\end{minipage}


\backmatter
% bibliography, glossary and index would go here.
\chapter{Respuestas}
\shipoutAnswer


\bibliography{./bibtex/EvansIntro,./bibtex/DiscMath,./bibtex/ConcMath,./bibtex/AlgManual,./bibtex/doom,./bibtex/IntroVectors}


\end{document}

% UNUSED 
%\makeatletter
%\renewcommand\chaptermark[1]{
%	\markboth{\textsc{
%			\ifnum\c@secnumdepth>\m@ne\if@mainmatter
%			\@chapapp\ \thechapter. \ \fi\fi #1}}{}%
%}
%\makeatother