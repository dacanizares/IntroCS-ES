\chapter{Programación imperativa}
 
En este capítulo nos introduciremos de una forma más profunda a la programación de ordenadores, haciendo especial énfasis en algunos conceptos y técnicas que todo programador debería dominar.

\section{¿Qué es un algoritmo?}

Cómo hemos visto, para programar un computador es necesario listar una secuencia de instrucciones que la máquina debería ejecutar. De aquí se desprenden dos conceptos importantes, a saber:

\begin{itemize}
	\item \emph{Proceso}: es una secuencia de pasos.
	\item \emph{Procedimiento}: Es la descripción de un proceso. 
\end{itemize}

Podríamos decir entonces que un listado de pasos es un \emph{procedimiento} y que el acto de seguirlos es el \emph{proceso} \cite[p.~2]{evansIntro}. Como los computadores no son máquinas, los \emph{procedimientos} que escribamos deben ser muy claros y las instrucciones no deberían dar lugar a ambigüedades. Luego los \emph{procedimientos} que ejecuta un computador deberían poder seguirse de forma mecánica, es decir, sin necesidad de realizar pensamientos adicionales. \textbf{Un algoritmo} es un procedimiento mecánico que eventualmente finalizará (no debería ejecutarse infinitamente).

Ahora, para solucionar problemas en nuestros computadores debemos aprender a construir \emph{algoritmos}. Es importante resaltar que cuando hablamos de programación, queremos encontrar un \emph{algoritmo} que solucione todas las instancias de un problema, y no sólamente una \cite[p.~53]{evansIntro}, es decir, no queremos un programa que sepa sumar $2+2$, si no uno que sepa sumar cualquier par de numeros $a$ y $b$. 

Dicho lo anterior, debemos pensar en encontrar \emph{soluciones generalizadas} y no simplemente resolver casos específicos de un problema, lo cuál es más complejo. Por suerte contamos con \emph{Paradigmas} que pueden guiar nuestras ideas para construir de una forma más predecible soluciones algorítmicas.

\section{Paradigmas de programación}

En la programación de ordenadores, existen a grandes rasgos dos corrientes que determinan la forma en que nos aproximamos a un problema \footnote{Básicamente el resto de paradigmas que puedan encontrarse se derivan a nivel teórico de alguno de estos dos.}. De igual forma, esos enfoques (o \emph{paradigmas}) nos brindan ciertas técnicas que guían la forma en cómo podemos construir nuestras soluciones.

\begin{enumerate}
\item \textbf{Programación imperativa}: el enfoque está en escribir instrucciones para modificar el estado de la memoria del computador, hasta eventualmente llegar a un estado que contenga la solución del problema.

\item \textbf{Programación funcional}: está relacionada con conceptos mucho más cercanos a las matemáticas, que en general tienen que ver con la aplicación de funciones que no manejan estado y que pueden \emph{componerse} para llegar a la solución.
\end{enumerate}

Cabe resaltar que ambos paradigmas suelen mezclarse, en mayor o menor medida, para construir soluciones computacionales efectivas. De momento nos centraremos en explorar la \emph{programación imperativa}.

\section{Pensamiento imperativo}

Cómo dijimos anteriormente, en la \emph{programación imperativa} modificamos constantemente la memoria del computador para intentar llevarla hasta un estado en el que obtengamos la respuesta a nuestro problema. Esto plantea algunas dificultades ya que \emph{una posición de memoria cambiará muchas veces durante la ejecución del programa}, luego será necesario abstraernos un poco más, de forma que podamos \emph{comprender el significado de nuestras instrucciones en el tiempo}. Es decir, que en nuestros programas, una variable cambiará de valor muchas veces durante la ejecución, de forma que debemos estar familiarizados con algunas herramientas que nos permitan comprender que es lo que realmente tenemos almacenado en esa variable a medida que el programa es ejecutado.

Mediante unos problemas, exploraremos las tres técnicas que debe dominar todo programador imperativo para manipular de forma básica las abstracciones básicas de la memoria del computador, a saber: cómo sumar múltiples datos, cómo contar elementos y cómo vigilar valores anteriores.

\subsection{Las sumas de Gauss}

\begin{wrapfigure}{r}{0.3\textwidth}
	\begin{center}
		\includegraphics[width=0.3\textwidth]{./Images/gauss.jpg}
	\end{center}	
	\caption{Karl Gauss.}
\end{wrapfigure}

Se cuenta que cuando Karl Gauss estaba en la escuela primaria, su maestra asignó a la clase la tarea de sumar los enteros del 1 al 100 (por ejemplo, 1 + 2 + 3 + · · · + 100) para mantenerlos ocupados \cite[p.~59]{evansIntro}. Gracias a sus habilidades, Gauss pudo desarrollar una fórmula para calcular esta suma de una forma mucho más ágil, obteniendo el resultado 5050. Exploremos dos formas de cómo solucionar este problema usando un computador.

\subsubsection{Solución imperativa}

Cómo vimos en el capítulo anterior, en Python es posible repetir instrucciones de forma sencilla, utilizando la instrucción \emph{for}. Ahora, la pregunta es ¿cómo repetimos las instrucciones, de manera que al finalizar, el estado de la máquina tenga la respuesta a este problema?

La respuesta es corta, pero quizá no muy intuitiva. Antes de continuar, es recomendable que se tome un par de minutos y escriba un programa de computador, con las instrucciones que usted cree, darán la respuesta. No importa si al final el código tiene errores, poner a trabajar las neuronas es más importante que llegar a la solución al primer intento. Una vez tenga su aproximación, continúe leyendo.
\\

Ya que no tenemos una hoja de papel para hacer un montón de operaciones, sólo posiciones de memoria que guardan bits (representadas en Python por el concepto de variables), necesitamos pensar muy bien como expresar las instrucciones. Hagámonos la siguiente pregunta, ¿será necesario almacenar en una variable diferente cada operación realizada? La respuesta es no. Imagine que usted tiene solo un pequeño trozo de papel, ¿cómo lo utilizaría para sumar los números del 1 al 100? Usted podría realizar operaciones y a medida que avanza ir borrando lo que ya no necesita guardarlas todas. Por ejemplo podría:

\begin{itemize}
\item Sumar 1 + 2 = 3. Como ya sabe la respuesta, puede borrar la operación y sólo conservar en su mente el número 3.
\item Luego, hay que sumar 3 + 3 = 6. De igual forma, sólo es necesario conservar el 6 presente, el resto se puede borrar.
\item La siguiente suma en el trozo de papel sería 6 + 4 = 10. Siguiendo el mismo órden de ideas, pasamos a la siguiente suma.
\item 10 + 5 = 15
\item 15 + 6 = 21
\item (...)
\item Y repetir, en el mismo pedazo de papel, hasta sumar los 100 números. 
\end{itemize}

En el computador la idea es la misma. No necesitamos 100 variables, usaremos una sola variable que iremos cambiando a medida que avance el tiempo, de forma que se acumule la suma. Ahora, si observamos cuidadosamente el proceso anterior, podemos encontrar que estamos sumando el valor que vamos acumulando, con cada uno de los 100 números. Luego, si damos el nombre de $suma$ a la variable en la que acumulamos el resultado, y le damos el nombre de $i$ a cada uno de los 100 número, la operación en el computador sería: 

\begin{equation}
suma = suma + i
\end{equation}

Puede revisar el código \ref{cod-for} para mayor claridad sobre el tratamiento que tendremos con la variable $i$.

Si metemos la expresión dentro de un \emph{for}, ¿funcionará? de hecho no. Notemos que el computador acumula haciendo $suma$ igual a lo que tenía más $i$. Y entonces, ¿qué tiene $suma$ la primera vez? Nada, pero hay que decírselo al computador iniciándola en cero. Al final, nuestro programa será como el que se observa en el código \ref{cod-gauss}. \\

\lstinputlisting[language=Python,caption=La suma de Gauss en Python,label=cod-gauss]{./py/imperativa/gauss.py}

\subsubsection{Ejecución en la máquina RAM}

Vamos a ver cómo se modifica la memoria de la máquina hasta dar con la respuesta. Existe una razón para no repetir \emph{muy a menudo} este tipo de visualizaciones: \textbf{el programador debe aprender a leer el código por lo que significa}; de otra manera nada ganaríamos con tener procesadores cada vez más rápidos, si a la hora de programar tenemos que repasar como ejecuta el computador las miles de iteraciones de nuestros programas. Si en algún momento usted está confundido con qué hace un código, es valido hacerle seguimiento a cómo cambian las variables, pero debería hacerse con valores relativamente pequeños y manteniendo un pensamiento holístico (es decir, no simplemente repitiendo los pasos de forma mecánica) que permita entender qué es lo que realmente está haciendo el programa.

Hecha esta aclaración, hagamos el seguimiento de la ejecución del programa para un valor más pequeño como 5:

\begin{itemize}
\item Se inicia la variable $suma$ en 0.
\item Se entra al ciclo en cuál la $i$ arranca en 1 y terminará en 5.
	\begin{itemize}
	\item En la primera iteración $suma=0+1$, luego, $suma = 1$
	\item En la segunda iteración $suma=1+2$, luego, $suma = 3$
	\item En la tercera iteración $suma=3+3$, luego, $suma = 6$
	\item En la cuarta iteración $suma=6+4$, luego, $suma = 10$
	\item En la quinta iteración $suma=10+5$, luego, $suma = 15$
	\end{itemize}
\item Cuando termina el ciclo se imprime el resultado $15$
\end{itemize}

Como usted puede ver, el análisis previo es exactamente lo que hace el programa. Al principio puede resultar un poco difícil, pero usted irá ganando progresivamente más confianza como programador para poder enfrentar cada vez retos más difíciles y apasionantes, lo importante es no dejar de practicar.

\subsubsection{Solución matemática}

Aunque nuestro programa resultó funcionar muy bien, es claro que Gauss no pudo contar un computador en su época. Así que veamos cómo lo hizo. \emph{Nota:} si bien está breve explicación matemática puede resultar interesante, es importante aclarar que como tal está solución no entraría dentro de las llamadas \emph{soluciones imperativas}, ya que simplemente comprenderemos de dónde viene una fórmula matemática y cómo podemos escribirla en nuestra máquina.

Gauss descubrió una fórmula que permite hallar fácilmente la sumatoria de los $n$ primeros números:

\begin{equation}
1+2+3+\cdots+n = \frac{n(n+1)}{2}
\end{equation}

Si reemplazamos la $n$ por 100 tenemos lo siguiente:

\begin{equation}
1+2+3+\cdots+100 = \frac{100(100+1)}{2}
\end{equation}

Que efectivamente es 5050. ¿Pero cómo intuyó esto Gauss? La respuesta la encontramos si organizamos los números como se ve en la figura \ref{figgauss}.

\begin{figure}[h!]
\centering
\begin{tabular}{| c | c | c | c | c | c | c |}
\hline
1 & 2 & 3 & $\cdots$ & (n-2) & (n+1) & n \\ \hline
n & (n-1) & (n-2) & $\cdots$ & 3 & 2 & 1 \\ \hline
\end{tabular}
\caption{Suma de los primeros $n$ números}
\label{figgauss}
\end{figure}

Es decir, tomamos la serie de forma ascendente y luego de forma descendente. Ahora ¿qué pasa si sumamos verticalmente? En la primera obtenemos $n+1$, en la segunda $2+(n-1)=n+1$, y si seguimos, veremos que en todas obtenemos $n+1$. Como teníamos $n$ números, tenemos $n$ veces $n+1$ que es lo mismo que $n \cdot (n+1)$. Finalmente, la respuesta la dividimos entre dos, porque tomamos la serie dos veces, llegando a la respuesta \cite{graham1994concrete}.

La idea de sumar dos veces la serie resultó bastante útil, y nuestro programa se podría reducir a una línea -imprimir la operación. Las soluciones matemáticas suelen ser mucho más elegantes, rápidas y poderosas que las soluciones a código puro y duro, sin embargo no todo puede resolverse en el computador con una fórmula tan sencilla. Generalmente, se deben mezclar ambos enfoques para resolver problemas más complejos.



\subsubsection{Del papel al computador}

Es necesario aclarar, que para pasar la fórmula anterior al computador hay que tener especial cuidado en el orden en el que la máquina ejecuta las operaciones. A saber:

	\begin{enumerate}
	\item Se resuelve lo que esté dentro de paréntesis \textbf{()}.
	\item Se resuelven las potencias \textbf{**}.
	\item Se resuelven las multiplicaciones, divisiones y residuos  \textbf{* / \%}. 
	\item Se resuelven las sumas y restas  \textbf{+} y \textbf{-}.
	\end{enumerate}
	
Así que, para que el resultado sea el esperado, es necesario utilizar signos de agrupación como se ve en el código \ref{cod-gauss-formula}. \\

\lstinputlisting[language=Python,caption=Las sumas de Gauss.,label=cod-gauss-formula]{./py/imperativa/gauss-formula.py}

Como pudimos ver, en ocasiones existen fórmulas matemáticas que nos permiten resolver los problemas de formas más directas (y rápidas), pero no siempre será el caso. Al final del día, lo más importante para un programador es tener muchas herramientas computacionales y matemáticas, entre más de esas herramientas \textbf{domine y sepa aplicar}, más problemas podrá resolver y mejores soluciones podrá plantear para cada caso.

\subsection{Los números primos}

Los números primos son muy importantes en la computación, ya que la criptografía actual esta basada principalmente en ellos. Sin embargo, los primos han fascinado a la gente desde tiempos inmemoriales. Por ejemplo, los griegos demostraron que existen infinitos números primeros \cite{discretemath}. Para entender qué son estos números, debemos definir el concepto de \emph{divisivilidad}.

\subsubsection{Divisibilidad}

Sean $a$ y $b$ dos enteros cualquiera. Decimos que $a$ es divisible entre $b$ si al realizar la división $a/b$ el residuo es cero.

Por ejemplo, 4 es divisible entre 2, porque al realizar la división 4/2 el residuo es cero; 5 no es divisible entre 2, porque al realizar la división 5/2 el residuo es uno.

Para hallar el residuo de una división en el computador, utilizamos el operador módulo (\%), como se observa en el código \ref{cod-mod}. \\

\lstinputlisting[language=Python,caption=Hallando el módulo o residuo de una división.,label=cod-mod]{./py/imperativa/codmod.py}



\subsubsection{Números primos}

Un número entero $p$ mayor que uno, es  primo si solamente es divisible por $1$ y por $p$.  

Por ejemplo, 2 es primo porque sólo lo divisible entre 1 y 2, pero 4 no es primo, porque es divisible entre 1, 2 y 4. Los siguientes son los primeros 100 números primos:

2 3 5 7 11 13 17 19 23 29 31 37 41 43 47 53 59 61 67 71 73 79 83 89 97 101 103 107 109 113 127 131 137 139 149 151 157 163 167 173 179 181 191 193 197 199 211 223 227 229 233 239 241 251 257 263 269 271 277 281 283 293 307 311 313 317 331 337 347 349 353 359 367 373 379 383 389 397 401 409 419 421 431 433 439 443 449 457 461 463 467 479 487 491 499 503 509 521 523 541 

Conociendo los conceptos matemáticos, nuestra misión será crear un programa de computador que determine la cantidad de divisores de un número y posteriormente indique si es primo o no.

\subsubsection{Solución}

La idea es simple: nos dan un número $n$, contamos cuántos divisores tiene, si al final tiene sólo dos divisores (1 y $n$) entonces es primo.

Si recordamos, todos los números son divisibles entre uno y entre el número mismo. Así que estas dos verificaciones son triviales y podemos omitirlas. Tenemos que centrarnos en determinar si algún otro número es divisor. Pero la pregunta es, ¿cuántos números diferentes debemos chequear? hay infinitos números, y nuestro programa no terminaría nunca de revisarlos todos. De hecho, en algún punto se bloquearía, porque la memoria de la máquina es finita. 

Luego, tenemos que determinar con cuidado qué números necesitamos verificar. Eso es algo fácil si lo pensamos así: ningún número es divisible por uno mayor que él. Por ejemplo, 3 no es divisible entre 5 (el 3 en el 5 está cero veces y sobran 3), dicho de otra forma, es imposible dividir el 3 en 5 partes enteras.

Dicho esto, bastará con \textbf{contar} qué números mayores que uno, pero menores que $n$, son divisores del número. Para contar en un programa de computador, se utiliza un mecanismo similar al del ejercicio anterior: definimos una variable, la iniciamos en algún valor (usualmente cero) y cada vez que se cumpla cierta condición le sumamos +1. Es como tener un papel en el que realizamos una raya cada vez que queremos contar algo.

Para nuestro caso, esa variable la llamaremos $divisores$, la iniciaremos en 2 (recuerde que todos los números tienen de entrada al uno y a él mismo como divisores) y cada vez que encontremos un divisor ejecutaremos la instrucción:
\begin{equation}
divisores = divisores + 1
\end{equation}

Sólo falta agregar, que al final es necesario verificar que el número no sea uno, ya que por convención el uno no es primo. En el código \ref{cod-primos} puede verse el programa completo.  \\

\lstinputlisting[language=Python,caption=Divisibilidad y números primos.,label=cod-primos]{./py/imperativa/primos.py}


\subsection{Vigilar valores anteriores}

Luego de haber aprendido cómo contar y cómo hacer sumatorias (junto con algunos conceptos matemáticos), sólo nos resta una técnica por revisar: \emph{vigilar valores anteriores}.

Supongamos que tenemos un programa que va a recibir $n$ números positivos por teclado y que al final debemos imprimir por pantalla el mayor de entre esos valores. ¿Cómo podríamos hacerlo? En el \autoref{chap:python} vimos que es posible con condicionales hallar el mayor de entre varios números, pero a medida que la cantidad de valores crece, las posibilidades también, lo que hace más complejo el código. En nuestro caso es aún peor, ya que no sabemos de antemano cuál será la cantidad de números ($n$), luego tenemos que ser más inteligentes en la forma de aproximarnos a la solución.

Una posible forma de solucionar este problema consiste  en  tener \emph{vigilado} el mayor valor que nos vayan ingresando. A medida que se digitan nuevos números, deberíamos comparar si el nuevo valor es mayor aquel que teníamos \emph{vigilado}. De ser así, se actualiza el dato que teníamos como el mayor. Repitiendo esta misma lógica, al terminar el programa deberíamos tener la respuesta como el valor \emph{vigilado}.

Es decir, suponiendo que $n = 4$ y que los números serán respectivamente $4, 6, 10, 1$, el programa debería comportarse más o menos así:

\begin{enumerate}
\item Como sabemos que nos van a digitar números positivos, el menor posible es 0, luego podemos iniciar vigilando dicho valor (cualquiera otro que nos ingresen será mayor), por lo tanto \textbf{mayor = 0}.
\item Nos ingresan el primer número y lo comparamos contra el que teníamos como mayor, si el nuevo número es efectivamente mayor, actualizamos. En nuestro caso $4 > 0$, luego \textbf{mayor = 4}.
\item Aplicamos la misma lógica con el resto de números. $6 > 4$, luego \textbf{mayor = 6}.
\item $10 > 6$, entonces \textbf{mayor = 10}.
\item $1 \leq 6$, entonces \textbf{mayor = 10} (no actualizamos).
\end{enumerate}

Lo anterior podemos verlo generalizado en el código \ref{cod-mayor}. Nótese que estamos simplemente \emph{vigilando} el mayor valor ingresado y que lo actualizamos a medida que el programa avanza en su ejecución. \\

 \lstinputlisting[language=Python,caption=Hallar el mayor número de entre una serie de valores.,label=cod-mayor]{./py/imperativa/mayor.py}

\newpage

En este capítulo hemos aprendido lo básico de la programación y de las soluciones imperativa: cómo sumar elementos, cómo contar elementos y vigilar un elemento anterior. \\

Mediante los problemas propuestos usted podrá a prueba sus conocimientos. Para saber si su solución está correcta, piense en algunas entradas y las salidas respectivas a cada una de ellas, luego ejecute su programa y valide si el resultado fue correcto. Si no es así, piense ¿qué puede estar fallando?

\section{Problemas}

\begin{Exercise}[title={Calentamiento}]
Realice un programa en Python para resolver los siguientes problemas:
\begin{enumerate}
\item Se va a recibir un número, se garantiza que será 0 o 1. En caso de recibir un 0 se debe imprimir 1, en caso de recibir 1 se debe imprimir 0. No se puede utilizar la instrucción \emph{if}.

\item ¿Cómo realizar una multiplicación sin usar el signo por (*)? Encuentre dos soluciones diferentes para éste problema.

\item ¿Cómo realizar una potencia sin usar el signo de potenciación (**)?

\item FALTA: Respuestas ¿Cómo hallar el menor de entre $n$ números positivos ingresados por teclado?

\item ¿Cómo hallar el menor de entre $n$ números (sin ninguna restricción)  ingresados por teclado?

\item ¿Cómo hallar el mayor de entre $n$ números (sin ninguna restricción)  ingresados por teclado?
\end{enumerate}
\end{Exercise}
\begin{Answer}

\lstinputlisting[language=Python,caption=Solución de los problemas de calentamiento.,label=cod-calentamiento]{./py/imperativa/calentamiento.py}
\end{Answer}

\begin{Exercise}[title={La Copia de Seguridad}]
Bill quiere guardar sus documentos que pesan $x$ gigas en unos discos que tienen $q$ gigas de capacidad. Por ejemplo, si tiene que guardar 6 gigas en discos de 4 gigas, necesitará 2 discos (aunque el segundo lo llene sólo parcialmente).

¿Podrá usted ayudar a Bill construyendo un programa para que él sepa cuántos discos debe comprar? Lea por teclado las variables $x$ y $q$, para calcular la respuesta.

Nota: recuerde que los códigos deben funcionar para cualquier entrada, no sólo para las que aquí se proponen. Asuma de momento que siempre le van a dar datos válidos.

\end{Exercise}

\begin{Answer}
 \lstinputlisting[language=Python,caption=Solución del problema La Copia de Seguridad.,label=cod-seguridad]{./py/imperativa/copiaseguridad.py}
 
\end{Answer}

\begin{Exercise}[title={El Precio del Videojuego}]
	Joe Cortana tiene una tienda de empeños para videojugadores. Todos los días entran a su tienda cientos de vendedores que quieren cambiar sus antiguos videojuegos por dinero. 
	
	Joe es muy cuidadoso y ha estudiado muy bien como se comporta el mercado. Ha descubierto que en las mañanas llegan los mejores juegos, de esos clásicos que son muy apetecidos por los coleccionistas. A medida que pasan las horas, cada vez recibe juegos más malos, terminando en ET (considerado el peor juego de la historia).
	
	Joe Cortana tiene disponibles $x$ dólares todos los días. Su misión es decirle, dada una lista de $n$ videojuegos con sus precios (todo esto se debe leer por teclado), cuántos títulos alcanza a comprar, de forma que esté adquiriendo siempre lo mejor que vaya entrando a su tienda.	
	
	Por ejemplo: si tiene 10 dólares, y van a ofrecerle 3 juegos que valen respectivamente 6, 4 y 3 dólares, él compra dos juegos.
	
	\textbf{Explicación}: 
	\begin{itemize}
		\item El primero vale 6, y lo compra, quedándole 4 dólares disponibles.
		
		\item El segundo vale 4 dólares, de forma que lo compra y no le queda nada.
		
		\item El tercero vale sólo un dólar, pero ya no tiene dinero. 
		
		\item En definitiva, alcanza a comprar 2 juegos.
	\end{itemize}
	
	Nota: recuerde que los códigos deben funcionar para cualquier entrada, no sólo para las que aquí se proponen. Asuma de momento que siempre le van a dar datos válidos.
	
\end{Exercise}
\begin{Answer}
\lstinputlisting[language=Python,caption=Solución el Precio del Videojuego.,label=cod-vjprecio]{./py/imperativa/vjprecio.py}
\newpage
\end{Answer}


\newpage
