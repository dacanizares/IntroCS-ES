\thispagestyle{empty}

\chapter{¿Por qué este libro?}
    La mayoría de libros sobre programación disponibles en nuestro idioma, omiten muchos conceptos matemáticos y computacionales que son importantes para quien desee realizar una carrera como programador. Esta publicación pretende ser una alternativa rigurosa, pero al mismo tiempo práctica y apasionante, que llevará al lector en un viaje por el mundo de las \emph{Ciencias de la Computación} (CS, por sus siglas en inglés Computer Science).    
    
    Esta metodología no pretende ser la mejor, ni la única, es solamente una alternativa. Inspirada por clases como \emph{CS-101 de Udacity} y el libro del profesor David Evans \emph{Introduction to Computing}, puede interesarle a personas que no han tenido la oportunidad de acercarse a la programación (entendida más allá del simple desarrollo de software). Igualmente, este libro puede resultar útil para profesores que quieran experimentar otras formas de enseñar; no es posible  garantizar que todos los estudiantes aprenderán a programar, pero este método tiene un enfoque que puede resultar de utilidad en la transmisión de buenas prácticas de programación y en el desarrollo de pensamiento computacional.
    
    Finalmente, es importante resaltar que la forma de trabajo propuesta confía en que el lector se ponga manos a la obra, ya que en vez de mostrar mil formas de escribir una instrucción,  se plantean exploraciones y problemas reales que implican un mayor trabajo mental, pero que con toda seguridad resultarán más gratificantes y apasionantes de comprender.
     
    \newpage
    \thispagestyle{empty}
        \textbf{Nota}: Si el lector no ha tenido ningún acercamiento previo a la programación, se recomienda revisar la Hora del Código en el siguiente enlace: \url{https://studio.code.org/hoc/1}. Luego de terminar esas cortas lecciones, los contenidos en este libro serán más fácil de comprender.
      
    
  
    
 