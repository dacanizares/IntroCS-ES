\chapter{Introducción a Python}

Luego de ver a grandes rasgos cómo programar un computador, llegó la hora de explorar lo que hemos aprendido en un lenguaje de programación real. 

Cómo vimos, los principales componentes de un computador son \emph{la entrada}, \emph{la salida}, \emph{la memoria} y \emph{el programa}. Ya que vamos a escribir nuestros programas en Python, surge una pregunta: ¿Cómo interactúa Python con los componentes de nuestro computador? Después de muchos intentos con distintas explicaciones, quizá que la mejor forma de responder esta pregunta es mediante ejemplos reales en el lenguaje de programación. Se recomienda encarecidamente al lector, que pruebe todos los programas en su máquina, para que los vea en funcionamiento. De igual forma, nada de esto tendrá sentido si el lector pasa rápidamente sobre este libro y no se toma el tiempo para interiorzar cada instrucción.

\section{Instalando Python}

Los códigos escritos para este libro están basados en la version 3 de Python. Para instalarlo bastará con ir a la página oficial \url{https://www.python.org/} y en la sección descargas buscar las versión correspondiente para nuestro sistema operativo. Luego de completar la instalación se recomienda reiniciar el computador.

Luego de instalar Python, podríamos usar el editor integrado que la herramienta contiene, se llama IDLE y deberías poder encontrarlo en el menú de inicio o en el listado de programas instalados de tu sistema operativo. Sin embargo, la mejor opción sea instalar VS Code, el cuál se puede descargar de \url{https://code.visualstudio.com/}. Luego de haberlo instalado, en la sección de \textbf{Extensiones} debería instalarse el complemento que da soporte a Python, bastará con buscar en la barra que tiene dicha sección y seleccionar la primera opción (Python, extesion de Microsoft).

Para poder ejecutar cualquier código bastará con crear un archivo con extensión \textbf{.py} y presionar la tecla F5 (o en su defecto clic derecho en el archivo, \emph{ejecutar archivo Python en la terminal}).

\section{Programación en Python}

Aunque en un computador todo termina siendo unos y ceros, Python tiene algunas abstracciones básicas (otras un poco más avanzadas, que veremos en el capítulos posteriores) que harán todo más fácil. Una abstracción permite manipular la máquina sin necesidad de conocer a fondo su funcionamiento. Por ejemplo, el pedal de un carro permite que usted acelere sin necesidad de conocer el funcionamiento del motor. 

Si hablamos de la memoria, en Python contamos con diferentes \textbf{tipos de datos} que determinan las acciones podemos realizar con los datos almacenados, sin necesidad de conocer a profundidad qué hace el computador con ellos. A saber:

\begin{itemize}
\item Números: con los números podemos efectuar operaciones aritméticas como sumas (+), restas (-), multiplicaciones (*), divisiones(/), residuos(\%) y potencias (**). En Python, los enteros se llaman \textbf{int} (de \emph{integer}, en inglés) y los que tienen decimales se conocen como \textbf{float}. Por ejemplo: 0, -1, 1, 40, 52, son enteros y 0.1, 0.7, -1.8 son decimales. 

Si luego de realizar una división queremos elimitar los decimales usamos el operador división entera (//). Ejemplo: \emph{5 / 2} da como resultado \emph{2.5}; \emph{5 // 2} da como resultado \emph{2}.

Cuando hablamos de residuos, es el valor que sobra luego de realizar una división. Ejemplo: Si dividimos 5 entre 2 el residuo es 1, luego \emph{5 \% 2} da como resultado \emph{1}.

\item Valores lógicos: en Python se conocen como \textbf{bool} (booleanos) y sólo tienen dos valores posibles: verdadero (\textbf{True}) o falso (\textbf{False}). Podemos aplicar los operadores Y (\textbf{and}), O(\textbf{or}), Negación (\textbf{not}), entre otros, para componer proposiciones más complejas.  Ver código \ref{cod-asignacion}.

\item Texto: se conocen como \textbf{strings} o secuencias de caracteres. Empiezan por una comilla doble o sencilla y terminan por el mismo símbolo, esto, con el fin de distinguir los textos arbitrarios definidos por el programador de las instrucciones propias del programa. Ver código \ref{cod-asignacion}.
\end{itemize}

Python no sólo facilita la representación de los datos en memoria, sino también su almacenamiento y manipulación. Para esto, existe el concepto de \textbf{variables}, que permiten dar un nombre arbitrario \footnote{Sólo hay tres restricciones: usar únicamente letras, números o guiones bajos, no empezar por un número y no dejar espacios en blanco} y un valor a partes de la memoria solamente utilizando el operador de igualdad (=), como se ve en el código \ref{cod-asignacion}. Python detecta automáticamente el tipo dato que se le asigna a una variable. \\

\lstinputlisting[language=Python,caption=Asignaciones en Python,label=cod-asignacion]{./py/nociones/asignacion.py}

Si ejecutamos ese código, el computador no mostrará nada, eso es porque aún no hemos definido instrucciones para la salida de datos. La forma básica de mostrar información en la pantalla es usando la instrucción \emph{print}. Basta con escribirla y dentro de paréntesis colocar el valor, la operación o la variable que se desea mostrar. En Python podemos comentar el código, añadiendo el caracter \#, de forma que podamos escribir aclaraciones del programa, sin afectar su comportamiento. Todo lo anterior se ve en el código \ref{cod-salida}. \\


\lstinputlisting[language=Python,caption=Salida de datos en Python,label=cod-salida]{./py/nociones/salida.py}

Por otro, la entrada de datos básica se hace con la instrucción \emph{input()}; entre los paréntesis se puede colocar un mensaje para que le aparezca al usuario. Luego de ingresar los caracteres por teclado y presionar la tecla \emph{enter}, los datos quedan almacenados en la variable a la cuál se haya asignado la instrucción. Sin embargo, hay que anotar que la instrucción \emph{input()} nos devuelve un \emph{string}(o texto), luego, si queremos hacer operaciones númericas, debemos convertir ese texto en un número usando alguna de las dos siguientes opciones:

\begin{itemize}
 \item Convertir una entrada a entero: \textbf{entero = int(input('Digita un entero'))}
 \item Convertir una entrada a decimal: \textbf{entero = floats(input('Digita un entero'))}
\end{itemize}

 Por ejemplo, el código \ref{cod-suma} muestra como solicitar dos números por teclado para luego imprimir el valor que resulta al sumarlos \footnote{Como lo importante es aprender, por ahora confiaremos en que el usuario no cometerá errores al ingresar los datos. Hay libros complican al estudiante que poniendole a realizar con mil y una validaciones antes de poder hacer cualquier cosa, bajo la excusa de que eso no se puede omitir en un programa comercial. Eso no tiene sentido, al fin y al cabo, cuando se realizan programas grandes, se usan otras herramientas que ayudan a prevenir estos errores. En capítulos posteriores exploraremos cómo hacerlo.}. \textbf{Importante:} Qué pasa si no convertimos las entradas en enteros? Se recomienda al lector que ensaye en su computador. Si quedan dudas en la sección problemas de este capítulo podrá encontrar la respuesta. \\

\newpage

\lstinputlisting[language=Python,caption=Lectura y suma de dos números en Python,label=cod-suma]{./py/nociones/suma.py}

En Python también es posible condicionar la ejecución de ciertas instrucciones utilizando la expresión \emph{if}, que traduce \emph{si} en español. Dicha expresión, debe ir acompañada de una \emph{condición}, que no es más que la evaluación, \emph{verdadera} o \emph{falsa}, de una proposición utilizando los operadores lógicos: es igual (==) \footnote{Nótese que se usa doble igual para distinguir la comparación de la asignación.}, mayor ($>$), menor($<$), mayor o igual ($>=$), menor o igual ($<=$) o diferente (!=). Por ejemplo, el código \ref{cod-if} sólo ejecuta la instrucción que imprime \emph{Es cero}, si  el valor ingresado es igual a 0. Las instrucciones que están condicionadas \textbf{deben ir tabuladas}, como se observa en el código de ejemplo, y pueden ser una o más instrucciones de cualquier tipo. \\

\lstinputlisting[language=Python,caption=Un \emph{if} en Python,label=cod-if]{./py/nociones/if.py}


La instrucción \emph{if} puede ir acompañada de un \emph{else}, que traduce \emph{si no} en español, y que se utiliza para indicar qué instrucciones ejecutar cuando la condición no se cumple. Esta situación se puede observar en el código \ref{cod-ifelse}, en el cuál se reciben dos números y se imprime cuál de ellos es el mayor (en caso de ser iguales, no importará cuál mostremos). Nótese que el texto entre comillas se muestra tal cuál en patalla, mientras que el valor después de la coma se reemplaza por el valor que tenga la variable en el momento de la ejecución. \\

\lstinputlisting[language=Python,caption=Un \emph{if-else} en Python,label=cod-ifelse]{./py/nociones/ifelse.py}

Si se necesitan condicionar más caminos, se puede utilizar la expresión \emph{elif}, que es una abreviación de \emph{else if}. Por ejemplo, si quisiéramos saber cuál es el mayor de tres números (llámense $a$, $b$ y $c$), habría tres opciones: 
\begin{itemize}
\item Que $a$ sea mayor que $b$ y que $c$, en tal caso, $a$ sería el mayor.

\item Si no, se debe descartar entre $b$ y $c$ cuál es mayor, luego, si $b$ es mayor que $c$, $b$ es el mayor.

\item Si ninguna de las anteriores condiciones se cumple, es porque el mayor era $c$. 
\end{itemize}


Lo expresado anteriormente se observa en el código \ref{cod-ifelifelse}.  \\

\lstinputlisting[language=Python,caption=El mayor de tres números,label=cod-ifelifelse]{./py/nociones/ifelifelse.py}

En cuanto a la repetición de instrucciones, se utiliza la instrucción \emph{for}, que traduce \emph{para} en español. Veamos como funciona: \\

\lstinputlisting[language=Python,caption=Repetición de instrucciones en Python,label=cod-for]{./py/nociones/for.py}

El código anterior imprime los números desde el cero hasta el nueve (como puede ver, el límite superior no se incluye). Su lectura en español nos ayudará a entenderlo: \emph{para i en el rango de 0 a 10, imprima i}. La $i$ es una variable cualquiera, por lo cuál podemos utilizar otro nombre; es común usar $i$ por su relación con los índices y subíndices en matemáticas. Los dos números dentro del paréntesis de \emph{range} indican desde que valor inicia la repetición y en cuál termina. Al igual que con la estructura \emph{if}, las instrucciones que se repiten deben ir tabuladas.

Es posible mezclar todas las instrucciones que hemos visto (y las que veremos), por ejemplo, tener programas como el del código \ref{cod-forif} que recibe dos números, imprime todos los enteros que hay entre ellos, indicando si son negativos o positivos.


\lstinputlisting[language=Python,caption=Repeticiones y decisiones en Python,label=cod-forif]{./py/nociones/forif.py}



\section{Problemas}



\newpage