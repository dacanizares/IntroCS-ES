\chapter{Programando un motor 3D}
\label{chap:motor-3d}

En este capítulo nos aventuraremos a desarrollar un pequeño motor tridimensional. No construiremos una pieza optimizada de alta tecnología pero aprenderemos algunos de los conceptos que permitieron en el año 1992 a una empresa llamada \emph{ID Software} \footnote{Las imágenes de juegos de id Software usadas en este capítulo son propiedad de sus respectivos dueños. Imágenes incluidas únicamente con fines educativos} crear uno de los juegos más emblemáticos de la historia, \textbf{Wolfenstein 3D}. ¿Preparados para poner su cerebro a prueba?

Antes de continuar se recomienda revisar el capítulo \autoref{chap:pygame}.

\begin{figure}[h!]
	\centering
	\includegraphics[width=8cm]{./Images/wolf3d.jpg}
	\caption{Wolfenstein uno de los primeros juegos en utilizar gráficos 3D de forma efectiva.}
	\label{wolf3d}
\end{figure}

\newpage

\section{Contexto histórico}

\begin{wrapfigure}{r}{0.4\textwidth}	
	\begin{center}
		\includegraphics[width=0.4\textwidth]{./Images/keen.png}
	\end{center}
	\caption{Keen Commander, de de ID Software, uno de los primeros juegos de plataformas en aprovechar toda la potencia de los ordenadores de la época.}
	\label{keen}	
\end{wrapfigure}

Corría el año de 1992, la mayoría de computadores eran vistos como máquinas inferiores a las consolas: apenas si tenían capacidad para ejecutar de forma fluida algunos juegos en 2D.

Luego de su éxito con \emph{Keen Commander}, John Carmack, John Romero, Tom Hall y Adrian Carmack, fundadores de id Software, decidieron crear un videojuego totalmente diferente, uno con una perspectiva en primera persona y grandes dosis de acción frenética.

John Carmack, que llevaba ya unos cuántos meses interesado en generar gráficos 3D en los limitados ordenadores del momento, fue el encargado de construír el motor que renderizaría los gráficos de aquel juego. Exprimiendo al máximo el poder de esos ordenadores, y valiéndose de inteligentes optimizaciones matemáticas y algorítmicas, John Carmack logró lo que parecía imposible, y en Mayo de 1992, id Software saltó al éxito con uno de los títulos gráficamente más impresionantes de la época. Había nacido: \emph{Wolfenstein 3D}.
 
 
 \begin{quote}
 	\emph{Tenemos esta oportunidad de hacer algo totalmente nuevo aquí (...) Si podemos hacer que los gráficos se vean fantásticos y rápidos, y hacer el sonido fresco y fuerte, y el juego explosivamente divertido, entonces tendremos algo ganador.} \\
 	John Romero, en Masters of Doom \cite[p.~92]{doom}.
 \end{quote}

\newpage

\section{Simulando gráficos en 3D}


Dada la poca potencia de los computadores de la época, tener verdaderos gráficos en 3D era una tarea imposible, entonces, ¿cómo hicieron en id para crear Wolfenstein 3D? La respuesta es más simple de lo que parecen: \emph{todo es una ilusión}.

\begin{figure}[h!]
	\centering
	\includegraphics[width=10cm]{./Images/wolf3d-mapa.png}
	\caption{El mapa siempre fue una cuadrícula bidimensional. El jugador es el punto rojo y el triángulo amarillo refleja el campo de visión.}
	\label{wolf3d-mapa}
\end{figure}

Como se puede apreciar en la figura \ref{wolf3d-mapa}, toda la lógica del juego ocurría en una cuadrícula; la magia era mostrar los laberintos y enemigos cómo si fueran parte de un entorno tridimensional, pero la verdad, nunca dejaron de ser 2D.

Esto traía muchas ventajas a nivel de rendimiento pero venía con algunas limitaciones, como por ejemplo, que al ser realmente un mapa 2D no podían tenerse cuartos unos encima de otros. Sin embargo, esta estrategia no era suficiente en sí misma, los procesadores de aquella época era muy limitados y para mantener una tasa de refresco lo suficientemente alta, fueron necesarias otras simplificaciones y optimizaciones, entre ellas: que los muros fueran todos en ángulos de $90^{\circ}$, que los mapas siguieran una estricta alineación a una grilla, que el techo y el piso fueran colores planos sin texturas de ningún tipo. Al final, pese a todo esto, dados los tiempos que corrían, el resultado fue impresionante. 

A continuación, exploraremos, el engranaje que permitía generar esos increíbles gráficos en esos limitados computadores de principios de los 90's.

\newpage
\subsection{Raycasting}

\begin{wrapfigure}{r}{0.4\textwidth}
	\begin{center}
		\includegraphics[width=0.25\textwidth]{./Images/rays.png}
	\end{center}	
	\caption{Rayos lanzados en el campo de visión del jugador para calcular la distancia de los muros. \newline}
	\label{rays}
	
	\begin{center}
		\includegraphics[width=0.4\textwidth]{./Images/raytracing.png}
	\end{center}	
	\caption{Pintado de los muros de acuerdo a la distancia. Al final se genera la ilusión de un espacio en 3D. \newline}
	\label{raytracing}
	
	\begin{center}
		\includegraphics[width=0.4\textwidth]{./Images/wolf3d-juego.jpg}
	\end{center}
	\caption{Gráficos dentro del Wolfenstein 3D.}
	\label{wolf3d-inside}

\end{wrapfigure}

El \textbf{Raycasting} es una técnica que consiste en lanzar rayos para determinar cuál es la primera superficie que estos interceptan. El motor de Wolfenstein basaba su renderizado en está técnica. La idea general es la siguiente:

\begin{enumerate}
	\item De acuerdo al campo de visión del jugador empezando desde la izquierda y hasta recorrer todo el cono, se lanzarían unos \emph{rayos} verificando a qué distancia se encuentran los muros. Ver figura \ref{rays}.
	
	\item A continuación, esas distancias se utilizan para pintar en la pantalla, de izquierda a derecha unas líneas verticales representando los muros. Entre mayor sea la distancia, más pequeña debe ser la línea (para dar la sensación de perspectiva). Ver figura \ref{raytracing}.
\end{enumerate}

Con está técnica es posible pintar cualquier mapa de Wolfenstein. Ahora, el algoritmo original tiene muchas optimizaciones y detalles adicionales para conseguir una velocidad de renderizado adecuada, así como permitir detalles que incluyen texturas en los muros, lámparas, objetos, enemigos, entre otros. Sin embargo, para nuestro empeño nos valdremos de esta versión simplificada.

\newpage
\clearpage

\newpage 
\subsection{Algo de matemáticas}

Para poder implementar nuestra solución será necesario revisar algunos conceptos matemáticos. 

\subsubsection{Vectores}

Un vector es un objeto que tiene una magnitud y una dirección \cite{introVectors} \footnote{En español solemos decir que tienen módulo, dirección y sentido, pero nos ceñiremos a la definición anglosajona, ya que resulta a todas luces más clara para un programador.}, gráficamente lo representamos como una \emph{flecha}.
 
 \begin{figure}[h!]
 	\centering
 	\includegraphics[width=6cm]{./Images/vector.png}
 	\caption{Representación gráfica de un vector.}
 	\label{vector}
 \end{figure}

Con los vectores podemos representar por ejemplo la velocidad en la que se mueve un personaje: la magnitud indicaría la rapidez y la dirección indicaría hacia dónde se está moviendo. En el computador, una representación admisible requeriría dos variables:

\begin{enumerate}
\item Un ángulo ($\theta$) indicando \emph{la dirección}.

\item Un número ($v$) indicando \emph{la rapidez}.
\end{enumerate}


\subsubsection{Vector unitario}

Los vectores unitarios son aquellos cuya magnitud es 1, luego suelen utilizarse cuando necesitamos saber \textbf{solamente} la dirección en la que apunta el vector.

\begin{figure}[h!]
	\centering
	\includegraphics[width=6cm]{./Images/unitaria.png}
	\caption{Circunferencia unitaria.}
	\label{unitaria}
\end{figure}


Cuando tratamos problemas de dos dimensiones, nos referimos a la grafica \ref{unitaria}. De dicha gráfica podemos sacar, dado un ángulo $\theta$, las componentes en X e Y del vector unitario que representa dicho ángulo.

\begin{equation}
\begin{aligned}
\label{unitvector}
ComponenteX = sin \theta \\
ComponenteY = cos \theta
\end{aligned}
\end{equation}

En otras palabras, dado un ángulo, al aplicar el \emph{sin} y el \emph{cos}, podemos obtener un vector unitario que apunta en la misma dirección. En breve, veremos cómo utilizar este concepto para darle movimiento al jugador y para disparar los \emph{raycasts}.

\subsubsection{Radianes}

Los radianes son una unidad utilizada para representar ángulos. En los lenguajes de programación es más común que se utilicen \emph{radianes} en vez de \emph{grados} para la manipulación y operación de ángulos, dicho esto, las funciones como \emph{sin} y \emph{cos} en Python esperan recibir ángulo en \emph{radianes}. 

\begin{figure}[h!]
	\centering
	\includegraphics[width=\textwidth]{./Images/radianes.png}
	\caption{¿Gráficamente qué son los radianes?}
	\label{rads}
\end{figure}

En la gráfica \ref{rads} vemos una intuición sobre lo que son los radianes. De la misma gráfica, es posible notar que $\pi$ radianes equivale a $180^{\circ}
$, luego $2\pi$ son $360^{\circ}$. Siguiendo la misma lógica, $\frac{\pi}{2}$ son  $90^{\circ}$ y $\frac{\pi}{4}$ son  $45^{\circ}$.


Estos conceptos matemáticos que introducimos brevemente, serán los necesarios para modelar las ecuaciones que nos permitirán construir nuestro \textbf{motor 3D}. A continuación veremos como aplicar lo aprendido, no sin antes revisar el código base sobre el cuál programaremos.


\subsection{Código base en PyGame}

El Siguiente será el código base que usaremos para construír el motor. Nótese que en el código, además de los conceptos introducidos en el \autoref{chap:pygame}, hemos añadido un par de líneas para cargar una fuente y pintar en pantalla los cuadros por segundo (FPS - Frames per second) a los que está cargando el motor, de forma que podamos tener una medida de qué tan rápido (o tan lento) es el renderizado.

\lstinputlisting[language=Python,caption=Código base para el motor.,label=cod-motor-3d-pygame,basicstyle=\fontsize{9}{9}\selectfont\ttfamily]{./py/motor-3d/base-pygame.py}

\newpage
\subsection{Modelando el motor}

Para realizar el modelo matemático que dará forma a nuestro motor, construiremos las soluciones formalmente y al final de cada sección escribiremos el código de lo previamente analizado, de forma que sea posible interiorizar y contrastar los conceptos matemáticos con la implementación real. En el \autoref{chap:enlaces} (Enlaces adicionales) el lector podrá encontrar contenidos adicionales para profundizar este tópico.

Para empezar, cabe anotar que durante toda la ejecución tendremos acceso a los siguientes datos:

\begin{itemize}
	\item Posición del jugador: $jugador_x$ y $jugador_y$, dos números decimales.
	\item Dirección en la que está mirando el jugador, almacenada como un número decimal que representa ángulo en radianes: $jugador_{rot}$.
	\item La velocidad de movimiento y de rotación del jugador, almacenados como dos números decimales: $velocidad$,  $velocidad_{rot}$. 
	\item El campo de visión del jugador en radianes $FOV$ (Por sus siglas en inglés Field of View - Campo de Visión).
	\item La resolución de la pantalla: el alto y ancho de la pantalla, dos enteros.
	\item Una distacia máxima para limitar qué tan lejos mandamos los raycast, un decimal.
	\item El mapa, que definiremos en breve.
\end{itemize}

Luego los cálculos que realicemos, estarán basados en dichos valores. Lo anterior podríamos codificarlo en Python como se ve en el fragmento \ref{cod-motor-vars}.

\newpage
\lstinputlisting[language=Python,caption=Variables generales del programa con algunos valores posibles.,label=cod-motor-vars,basicstyle=\fontsize{9}{9}\selectfont\ttfamily]{./py/motor-3d/variables.py}

\subsubsection{Cargando el mapa}

Definir el cargado del mapa en memoria es algo sencillo. Para tal empeño, una simple función que acceda la variable global \emph{mapa}, y que agregue múltiples líneas representando un nivel, donde los $.$ significan espacio vacío y los $\#$ simbolizan muros.

\lstinputlisting[language=Python,caption=Función para cargar el mapa.,label=cod-motor-vars,basicstyle=\fontsize{9}{9}\selectfont\ttfamily]{./py/motor-3d/mapa.py}


\subsubsection{El movimiento del jugador}

Para el movimiento, necesitamos aplicar los conceptos de vectores que revisamos previamente. La idea es la siguiente: cuando se presione la tecla arriba, el jugador se moverá hacia adelante en la dirección en la que él está mirando. Luego, deberíamos obtener un vector unitario a partir de la variable $jugador_{rot}$ , multiplicarlo por la $velocidad$ a la que se desplaza el jugador y sumarlo a su posición actual. Aplicaríamos este concepto a cada componente ($x$ e $y$) como se puede ver en las ecuaciones \ref{movforw}. Para que camine hacia atrás, en vez de sumar, restamos. 

\begin{equation}
\begin{aligned}
\label{movforw}
jugador_x = jugador_x \pm velocidad * sin(jugador_{rot})  \\
jugador_y = jugador_y \pm velocidad * cos(jugador_{rot})
\end{aligned}
\end{equation}

Finalmente, para la rotación, bastaría con sumar o restar \emph{la velocidad de rotación} a \emph{la rotación actual del jugador}.

\begin{equation}
\begin{aligned}
\label{eq-rot}
jugador_{rot} = jugador_{rot} \pm velocidad_{rot}
\end{aligned}
\end{equation}

\lstinputlisting[language=Python,caption=Movimiento de acuerdo a las teclas presionadas.,label=cod-motor-mov]{./py/motor-3d/movimiento.py}

\newpage

\subsubsection{El raycast}

El cálculo de un \emph{Raycast} implica que sepamos dirección dispararlo (para lo cuál usaríamos un vector unitario) y qué tan lejos hacerlo (de momento introduciremos una variable llamada $distancia$, que ya veremos cómo calcularla). Una simplificación que podemos hacer, gracias a que el mapa está perfectamente alineado a una cuadrícula, es omitir los decimales al calcular las coordenadas del rayo. Para esto usaremos la función \emph{floor} (que aproxima un número decimal a su entero más cercano por debajo). Todo lo anterior lo podemos ver en la ecuación \ref{eqray}.

\begin{equation}
\begin{aligned}
\label{eqray}
rayo_x = floor(x + sin \theta * distancia)\\
rayo_y = floor(y + cos \theta * distancia)\\
\end{aligned}
\end{equation}

Hablando de la $distancia$, la calcularemos de la siguiente forma: la iniciamos en 0 e iterativamene la iremos incrementando poco a poco hasta llegar a un valor máximo o hasta que toquemos una pared (Ver imagen \ref{raycastit}). La idea de poner un límite máximo es optimizar un poco las cosas, si ya el rayo está muy grande y no hemos encontrado una pared, no queremos seguir buscando infinitamente, porque el juego se bloquearía o cargaría extremadamente lento.

\begin{figure}[h!]
	\centering
	\includegraphics[width=6cm]{./Images/raycast.png}
	\caption{Incrementando iterativamente la distancia del rayo hasta encontrar una pared o llegar a un límite máximo.}
	\label{raycastit}
\end{figure}

La lógica del \emph{raycast} puede traducirse a Pytohn utilizando un código como el del fragmento \ref{cod-motor-raycast}. Nótese que para mayor orden, hemos definido las instrucciones como una función que podamos reutilizar. A dicha función le enviamos la coordenada $x$ e $y$ y la $rotacion$ desde la cuál disparar el rayo; al final recibimos la distancia en la cuál hubo colisión, o en su defecto el valor que hallamos configurado como distancia máxima del rayo. En cuanto a \emph{saber cuándo tocamos un muro}, simplemente llamamos una función que recibe las coordenadas del rayo y chequea si están dentro del mapa, de ser así revisa si dicha posición es o no un muro.

\lstinputlisting[language=Python,caption=Función para realizar un raycast.,label=cod-motor-raycast]{./py/motor-3d/raycast.py}

Si miramos detenidamente, encontraremos que el valor de \emph{math.sin(rot)} y \emph{math.cos(rot)}, será el mismo durante un llamado a la función, luego podríamos optimizar ligeramente nuestro código haciendo el cálculo previamente y reutilizando el valor, como se puede ver en el código \ref{cod-motor-raycast-opt}.

\newpage

\lstinputlisting[language=Python,caption=Optimizando ligeramente la función del raycast.,label=cod-motor-raycast-opt]{./py/motor-3d/raycast-opt.py}


\begin{wrapfigure}{r}{0.4\textwidth}	
	\begin{center}
		\includegraphics[width=0.4\textwidth]{./Images/fov.png}
	\end{center}
	\caption{El campo de visión del jugador.}
	\label{fovgraph}	
\end{wrapfigure}

\subsubsection{\mbox{Raycasts para el campo de visión}}

Ahora, necesitamos calcular cómo mandar varios \emph{raycasts}, de forma que podamos dibujar todo el campo de visión (FOV) del jugador. Para este propósito, lo primero será hallar los angulos de inicio y fin desde los cuáles lanzar los rayos. Llamémoslos $FOV_{inicio}$ y $FOV_{fin}$.

De la gráfica \ref{fovgraph}, podemos ver que en la mitad del FOV se encuentra el ángulo en el que está mirando el jugador ($\theta$), luego $FOV_{inicio}$ podríamos calcularlo restando la \textbf{mitad} del FOV total a $\theta$ y $FOV_{fin}$ lo encontraríamos sumamos la mitad del total del FOV $\theta$. Tal y como se ve en la ecuación \ref{foveq}. 

\clearpage

\begin{equation}
\begin{aligned}
\label{foveq}
FOV_{inicio} = jugador_{rot} - \frac{FOV}{2}\\
FOV_{fin} = jugador_{rot} + \frac{FOV}{2} \\
\end{aligned}
\end{equation}

A continuación, debemos tener en cuenta que el FOV es en definitiva lo que vamos a pintar en el ancho de la pantalla. Luego, debemos hallar la proporción entre el FOV y el ancho de la pantalla para saber cuánto espacio debería haber entre cada rayo, de forma que al final se cubra toda la pantalla.

\begin{equation}
\begin{aligned}
\label{foveq}
Rayo_{espaciado} = \frac{FOV}{Pantalla_{ancho}}
\end{aligned}
\end{equation}

Finalmente, deberíamos tirar un rayo en el equivalente a cada pixel horizontal de la pantalla empezando desde la $FOV_{inicio}$. Luego, debemos realizar la siguiente operación para $pixel$, desde 0 hasta el ancho de la pantalla:

\begin{equation}
\begin{aligned}
\label{eq-rayo}
Rayo_{angulo} = FOV_{inicio} + pixel * Rayo_{espaciado} 
\end{aligned}
\end{equation}

Hablando del programa, bastaría con calcular el ángulo de cada uno de los rayos y utilizar la función \textbf{raycast} que definimos en la sección anterior, para así disparar el rayo desde la posición del jugador en el ángulo mencionado anteriormente. Todo esto reflejado en el código \ref{raycasts-fov}.

\lstinputlisting[language=Python,caption=Raycasts para todo el campo de visión del jugador.,label=raycasts-fov]{./py/motor-3d/raycasts-fov.py}

\subsubsection{Pintando la pantalla}


Al dibujar la escena tenemos 3 partes en la pantalla, a saber: el techo, el piso y el muro.

\begin{figure}[h!]
	\centering
	\includegraphics[width=6cm]{./Images/partes-3d.png}
	\caption{Renderizado del campo de visión del jugador.}
	\label{partes-3d}
\end{figure}

El algoritmo de pintado será, más o menos el siguiente: 

\begin{enumerate}
\item Pintamos la mitad superior de la pantalla con color negro, para indicar cuál es el techo.
\item Pintamos la mitad inferior de la pantalla con color café, indicando el piso.
\item Pintamos los muros de acuerdo a la distancia calculada, teniendo presente que la sensación de perspectiva la obtenemos así: entre más cercano el muro, debemos pintar una franja verde más grande; a mayor distancia, más pequeña la franja verde.
\end{enumerate}

En esta parte nos valdremos de una función llamada \textbf{dibujar\_rectangulo} que recibe los siguientes parámetros:

\begin{itemize}
\item La coordenada en X de la pantalla donde se va a pintar el rectángulo.
\item La coordenada en Y de la pantalla donde se va a pintar el rectángulo.
\item El ancho del rectángulo.
\item El alto del rectángulo.
\item El color en RGB (Red-rojo, Green-verde, Blue-azul), representado como una tupla de 3 valores, cada uno de ellos entre 0 y 255. Por ejemplo, los siguientes son algunos colores válidos:
	\begin{itemize}
	\item (0, 0, 0): negro.
	\item (255, 0, 0): rojo.
	\item (0, 255, 0): verde.
	\item (0, 0, 255): azul.
	\item (255, 255, 255): blanco.
	\end{itemize}
\end{itemize}

\subsubsection{El techo y el piso}

\begin{figure}[h!]
	\centering
	\includegraphics[width=10cm]{./Images/techo-piso.png}
	\caption{Pintado de los elementos básicos de la pantalla.}
	\label{techo-piso}
\end{figure}

Para dibujar el techo usaremos el color negro (0, 0, 0) y para el piso un color café (170, 85, 0). 

El techo es la mitad superior de la pantalla, luego debería pintarse como un rectángulo en la posición (0, 0), el tamaño debería ser de todo el ancho de la pantalla y de la mitad del alto de la misma; para el piso, el tamaño debería ser el mismo, pero la posición debería ser (0, mitad del alto de la pantalla). Lo anterior, reflejado en el código \ref{cod-motor-techo-piso}.

\newpage
\lstinputlisting[language=Python,caption=Código para pintar el techo y el piso.,label=cod-motor-techo-piso,basicstyle=\fontsize{9}{9}\selectfont\ttfamily]{./py/motor-3d/techo-piso.py}


\subsubsection{Los muros}

Con respecto a los muros, recordemos que lo más importante es que a mayor distancia, debería verse más pequeño. Es necesario resaltar, que en un motor real la distancia se calcula haciendo una proyección ortogonal al vector en la dirección exacta que está mirando el jugador, de esta forma se evita un efecto llamado \emph{fish-eye effect}. Sin embargo, en nuestro caso usaremos una fórmula simplificada que nos dará un buen efecto, sin necesidad de entrar en detalles adicionales.

\begin{figure}[h!]
	\centering
	\includegraphics[width=10cm]{./Images/techo-piso-muro.png}
	\caption{Cambio en las proporciones de los elementos de la pantalla de acuerdo a la distancia de los muros.}
	\label{techo-piso-muro}
\end{figure}

El cálculo, cómo se puede ver en la figuro \ref{techo-piso-muro}, es determinar el tamaño de la sección de techo y piso, de acuerdo a la distancia calculada contra el muro. Teniendo ese dato, podemos saber qué tan grande pintamos la sección verde de la pantalla.

Lo primero que podemos determinar es que el $techo$ y el $piso$ tienen el mismo tamaño, luego teniendo uno de ellos, sabremos el otro.

Ahora, centrándonos en el techo, podemos ver lo siguiente: el valor máximo del $techo$ es la mitad del alto de la pantalla (en caso de no haber muro, como se ve en la figura \ref{techo-piso}), a dicho valor le restaremos una proporción de la pantalla de acuerdo a la distancia en la que se encuentra el muro.

\begin{equation}
\begin{aligned}
\label{eq-techo}
techo = \frac{Pantalla_{alto}}{2} - ProporcionDistancia
\end{aligned}
\end{equation}

Es posible calcular la proporción usando diferentes métodos, sin embargo, una que funciona bien y no genera demasiados efectos ópticos extraños, es dividir el alto de la pantalla entre la distancia con respecto al muro. 

\begin{equation}
\begin{aligned}
\label{eq-propdist}
ProporcionDistancia = \left(\frac{Pantalla_{alto}}{distancia}\right)
\end{aligned}
\end{equation}

Reemplazando en la ecuación \ref{eq-techo} obtenemos la fórmula completa (\ref{eq-techo-completa}).

\begin{equation}
\begin{aligned}
\label{eq-techo-completa}
techo = \frac{Pantalla_{alto}}{2} - \left(\frac{Pantalla_{alto}}{distancia}\right)
\end{aligned}
\end{equation}


Debido a la proporción usada, es posible que la fórmula en ocasiones arroje valores negativos, para prevenirlo bastará con hallar el máximo entre 0 y el valor obtenido.

Por otro lado, teniendo el valor del techo y sabiendo que el piso es del mismo tamaño, entonces podemos encontrar el tamaño del muro restando de la altura total de la pantalla dichos valores.

\begin{equation}
\begin{aligned}
\label{eq-muros}
& piso = techo \\
& muro = Pantalla_{alto} - piso - techo \\
\end{aligned}
\end{equation}


Para concluir, agregaremos un mayor efecto (\emph{shading}) de profundidad, haciendo el color más oscuro a medida que está más lejos el muro. Utilizando la fórmula \ref{eq-shading} podemos hacerlo fácilmente. Cabe anotar, que también deberíamos garantizar que no vayan a existir valores negativos, de igual forma, hallando el máximo valor entre 0 y el \emph{shading} calculado.

\begin{equation}
\begin{aligned}
\label{eq-shading}
shading = 1 - \frac{distancia}{DISTANCIA_{MAX}}
\end{aligned}
\end{equation}

Todo lo anterior, podemos aplicarlo en el código \ref{motor-3d-dibujar}.

\lstinputlisting[language=Python,caption=Código para pintar los muros con un shading que agudiza el efecto de profundidad.,label=motor-3d-dibujar,basicstyle=\fontsize{9}{9}\selectfont\ttfamily]{./py/motor-3d/dibujar.py}

\section{Código completo}

Aunque parezca magia, con los fragmentos anteriormente analizados y codificados, bastará con unir todo en un programa completo y \emph{voilà}, tendremos nuestro primer motor 3D funcionando. No es óptimo, pero ha sido toda una aventura programarlo. 

Para profundizar en este tópico, recuerde revisar la sección enlaces adicionales \autoref{chap:motor-3d}.

\newpage
\lstinputlisting[language=Python,caption=Código completo de nuestro primer motor 3D.,label=motor-3d-dibujar,basicstyle=\fontsize{9}{9}\selectfont\ttfamily]{./py/motor-3d/motor-3d.py}

