\section{Instalando Python}

Los códigos escritos para este libro están basados en la version 3 de Python. Para instalarlo bastará con ir a la página oficial \url{https://www.python.org/} y en la sección descargas buscar la versión correspondiente para nuestro sistema operativo. Luego de completar la instalación se recomienda reiniciar el computador.

Luego de instalar Python, podríamos usar el editor integrado que la herramienta contiene, se llama IDLE y deberías poder encontrarlo en el menú de inicio o en el listado de programas instalados de tu sistema operativo. Sin embargo, la mejor opción sea instalar VS Code, el cuál se puede descargar de \url{https://code.visualstudio.com/}. Luego de haberlo instalado, en la sección de \textbf{Extensiones} debería instalarse el complemento que da soporte a Python, bastará con buscar en la barra que tiene dicha sección y seleccionar la primera opción (Python, extesion de Microsoft).

Para poder ejecutar cualquier código bastará con crear un archivo con extensión \textbf{.py} y presionar la tecla F5 (o en su defecto clic derecho en el archivo, \emph{ejecutar archivo Python en la terminal} o el botón PLAY). Para más detalles ver imagen \ref{vscode}.

\begin{figure}[h!]
	\centering
	\includegraphics[width=14cm]{./Images/vscode.png}
	\caption{Elementos importanes en VS Code.}
	\label{vscode}
\end{figure}