\chapter{Nociones básicas}

En este capítulo veremos algunos de los conceptos involucrados en la programación de ordenadores a nivel básico. No se pretende que el lector los domine con sólo ver este capítulo, al contrario, la idea es dar un vistazo general sobre lo que utilizaremos y profundizaremos a lo largo del libro, de forma que, teniendo una imagen más completa de la programación los temas a tratar no parezcan obstáculos sin sentido, que sólo tienden a ralentizar el proceso de aprendizaje. 

Lo primero será comprender cómo las instrucciones de un programa operan directamente en la máquina, para esto introduciremos un modelo abstracto que representará nuestro computador.

\section{Máquina RAM}

 La Máquina de Acceso Aleatorio (RAM, por sus siglas en inglés), es un modelo utilizado para realizar análisis de algoritmos \cite[p.~31]{algDsgn}, tema que abordaremos más adelante. De momento introduciremos los conceptos más elementales sobre cómo funciona un computador para poder entender mejor de qué va esto de la programación. 
 
Vamos a imaginar un computador abstracto que se comporta de forma muy similar a cómo lo hacen nuestras máquinas reales, a ese computador lo llamaremos la Máquina RAM. Para mantener las cosas lo más sencillo posible, sólo consideraremos los siguientes componentes:


\begin{figure}[h!]
	\centering
	\includegraphics[width=10cm]{./Images/ram.png}
	\caption{Máquina RAM}
	\label{figram}
\end{figure}




\begin{itemize}
\item La memoria: representa el estado de la máquina en un momento del tiempo. Está dividida en múltiples secciones que llamamos posiciones de memoria, cada posición puede almacenar un número.

\item La entrada: son los datos sobre los cuáles se va a operar, estos pueden ingresar representados por: una pulsación del teclado, un clic, información que viaja por Internet, la señales de un sensor, etc.
 
\item El programa: es una secuencia de instrucciones que le dice a la máquina cómo modificar su memoria de acuerdo a la entrada que recibe y al estado mismo de la memoria. 

\item La salida: es el resultado que se obtiene después de ejecutar total o parcialmente el programa. Decimos \emph{parcialmente}, porque no tenemos que esperar hasta el final para compartir información con el usuario, por ejemplo, durante la ejecución podemos ir mostrando una barra de carga o mostrando mensajes sobre los datos que estamos procesando.
\end{itemize}



De una u otra forma, el programa tendrá que interactuar con todos los componentes de la máquina, luego, los lenguajes de programación deben proveer mecanismos para manipular y representar la memoria, la entrada y la salida de datos. Cabe anotar que, como el programa depende del estado actual de la máquina, también se debería contar con estructuras que permitan repetir o ejecutar ciertas instrucciones de acuerdo a los datos almacenados en memoria. Por ejemplo, se deben ejecutar algunas instrucciones si la máquina entra o en un estado de error o se deben repetir algunas instrucciones mientras que una persona esté presionando una tecla en un videojuego. A propósito veamos con un ejemplo práctico todos estos conceptos.


\section{Modelando un videojuego}


\begin{figure}[h!]
	\centering
	\includegraphics[width=12cm]{./Images/pong.png}
	\caption{El legendario PONG.}
	\label{figpong}
\end{figure}

Supongamos un videojuego sencillo como \emph{Pong} (ver figura \ref{figpong}). El juego es una especie de \emph{tenis} en el que dos jugadores, cada uno en control de una barra que representa una raqueta, deben golpear una pelota e intentar que el otro jugador no alcance a responder el golpe, para marcar un punto. 

Intentemos comprender cómo representar este juego en nuestra \emph{Máquina RAM}. No lo haremos exhaustivo pero un bosquejo, que permita entender a nivel general las mecánicas de la programación. Empecemos de lo más fácil a lo más complejo:

\begin{itemize}

	
	\item La entrada: en este caso la entrada serían las teclas que presionan los jugadores para mover arriba y abajo cada una de las raquetas.
	
	\item La salida: son los gráficos que vemos en la pantalla y que representan lo que está pasando en estado de la máquina mientras simula el videojuego.
	
	\item La memoria: para poder representar el estado del juego en la máquina es necesario tener espacios de memoria donde guardemos:
	
	\begin{itemize}
		\item La puntuación del jugador 1.
		\item La puntuación del jugador 2.
		\item La posición en el eje Y de la raqueta del jugador 1.
		\item La posición en el eje Y de la raqueta del jugador 2.
		\item La posición en el eje X de la pelota.
		\item La posición en el eje Y de la pelota.
		\item La velocidad de la pelota.
		\item La dirección en el eje X de la pelota, es decir, si va hacia la derecha o la izquierda.
		\item La dirección en el eje Y de la pelota, es decir, si va hacia arriba o abajo.
	\end{itemize}

	Con todos estos datos, es posible saber el estado del juego y así, el computador puede pintar en pantalla qué es exactamente lo que está ocurriendo.
	
	\item El programa: para que nuestra máquina simule realmente el videojuego tendríamos que tener en cuenta darle instrucciones para cada uno de los siguientes casos: 
	
	\begin{itemize}
		\item Pintar en la pantalla las raquetas, la pelota y la puntuación.
		
		\item \textbf{Si} el jugador 1 presiona la tecla arriba, mover la posición de su raqueta hacia arriba; \textbf{Si} presiona la tecla abajo, mover su raqueta hacia abajo. De igual forma habría que configurar un par de teclas para que el jugador 2 pueda moverse.
		
		\item Mover la posición de la pelota de acuerdo a su velocidad y dirección.
		
		\item \textbf{Si} la pelota colisiona con una raqueta, cambiar su dirección.
		
		\item \textbf{Si} la pelota toca la esquina izquierda, sumar más uno a la puntuación del jugador 1; \textbf{si} la pelota toca la esquina derecha, sumar a la puntuación del jugador 2.
				
		\item \textbf{Repetir} desde el primer paso, hasta que los jugadores se salgan del juego.
	\end{itemize}
\end{itemize}

Nótese que usamos palabras importantes (\textbf{Si}, \textbf{Repetir}) para definir cuando ejecutar o no una instrucción, o cuantas veces repetir una serie de pasos en nuestro código.

Hecho todo esto, y si la Máquina RAM entendiera español, más o menos podría simular \emph{Pong} para nosotros. Sin embargo, nuestros computadores no son tan inteligentes, luego debemos ser mucho más explícitos en las instrucciones que utilizamos, tema que comenzaremos a explorar en el próximo capítulo.

Para concluir, como podemos ver el programa termina siendo una serie de instrucciones para que el computador sepa que hacer en un momento determinado. Al principio puede parecer complejo entender los límites de lo que podemos programar o incluso más difícil aún, saber cómo expresar nuestras ideas en instrucciones que entienda la máquina. Similar a aprender un nuevo idioma, con poco de práctica y algo de imaginación, estas barreras irán desapareciendo a medida que avancemos.

\section{Problemas}
\begin{Exercise}[title={Acercamiento inicial}]	
	
Defina con sus propias palabras, pero tratando de ser muy claro y ordenado, las entradas, las salidas, la memoria y el programa para que la Máquina RAM pueda simular las siguientes situaciones:

	\begin{enumerate}
		\item Una calculadora que permita sumar o restar dos números.
		
		\item El videojuego \emph{Flappy Bird}. Es recomendable buscarlo en internet para tener claras las mecánicas del juego.
	\end{enumerate}
\end{Exercise}
\begin{Answer}	
	
Estos problemas no tienen una única solución, la idea es comenzar a interiorizar las mecánicas sobre las cuáles se escriben programas de computados (sin llegar a ser demasiado estrictos). El lector debería comparar sus respuestas con las soluciones propuestas y pensar en qué partes faltó claridad en su programa.
		
\begin{enumerate}
\item Calculadora:
	\begin{itemize}
		\item Entrada: números ingresados por teclado además de las teclas +, - y ENTER.
		\item Salida: resultado de la operación. También podríamos ir mostrando los datos ingresados por el usuario.
		\item Memoria: deberíamos poder almacenar el \emph{numero1}, el \emph{numero2}, la \emph{operación} a realizar y el resultado. Para la operación podríamos ser más específicos, ya que en la memoria sólo podemos almacenar números, el 0 representará restar y el 1 sumar.
		\item Programa:
		\begin{itemize}
			\item Almacenar el primer número digitado por el usuario en la memoria como  \emph{numero1}.					
			\item \textbf{Si} el usuario digita el signo -, almacenar en la memoria en \emph{operación} un 0.
			\item \textbf{Si} el usuario digita el signo +, almacenar en la memoria en \emph{operación} un 1.
			\item Almacenar el segundo número digitado por el usuario en la memoria como  \emph{numero2}.
			\item \textbf{Si} la \emph{operación} es 0, mostrar en la pantalla el resultado de restar: \emph{numero1 - numero2}.					
			\item \textbf{Si} la \emph{operación} es 1, mostrar en la pantalla el resultado de sumar: \emph{numero1 + numero2}.
			\item \textbf{Repetir} desde el primer paso hasta que el usuario se salga del programa.
		\end{itemize}
		
	\end{itemize}
	
\item Flappy birds:
	\begin{itemize}
		\item Entrada: el jugador tiene una única tecla ue presiona para que el pajaro aletee y vuele hacia arriba.
		\item Salida: en la pantalla debemos pintar el estado del juego, mostrando el jugador y los obstáculos.
		\item Memoria: deberíamos poder almacenar la \emph{puntación del jugador}, su \emph{posición}, la \emph{velocidad en el eje X} del jugador, la \emph{velocidad en el eje Y} y para mantener las cosas simples, sólo diremos que guardaremos la posición y tamaño de algunos obstáculos que irán apareciendo en la pantalla.
		\item Programa:
		\begin{itemize}
			\item Pintar en la pantalla el pájaro y los obstáculos visibles. Calcular los obstáculos visibles puede ser un poco complejo, pero por facilidad en la explicación no entraremos de momento en detalles		
			\item \textbf{Si} el jugador presiona la tecla de aletear, se debe incrementar la \emph{velocidad en el eje Y} de forma que el pájaro suba un poco.					
			\item Actualizar la posición del jugador de acuerdo a su velocidad en cada eje. A la velocidad en Y se le debe restar una cantidad, para que el impulso que tenía se vaya perdiendo y vuelva a comenzar a caer eventualmente.
			\item \textbf{Si} el jugador colisiona con un obstáculo pierde.
			\item \textbf{Repetir} desde el primer paso hasta que el jugador pierda.
		\end{itemize}
		
	\end{itemize}
\end{enumerate}
\newpage
\end{Answer}

\newpage
