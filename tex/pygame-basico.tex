\chapter{PyGame Básico}
\label{chap:motor-3d}

En este capítulo diseccionaremos un código que nos permite utilizar PyGame para crear experiencias más interactivas en Python, entre ellas, \emph{videojuegos}.

\section{Entendiendo el gameloop}

Antes de hablar de PyGame, es necesario comprender cómo funciona a grandes rasgos un \emph{videojuego}. Aquí palabra clave es: \textbf{REPETIR}.

Igual que pasa en una animación, cuando una serie de imagenes pasa a la velocidad suficiente, los personajes parecen estar vivos. Los videjuegos no son más que una seríe de imágenes calculadas muchas veces por segundo para dar la ilusión de movimiento. El computador debe estar \emph{repitiendo} constantemente una serie de instrucciones para tal efecto. A esa parte que repetimos lo llamamos \emph{gameloop} (algo como \emph{ciclo de juego}). Un \emph{gameloop} tiene diferentes instrucciones dependiendo del juego que se esté ejecutando, pero a grandes rasgos, podría verse más o menos así:

\begin{enumerate}
\item Se revisan que teclas ha pulsado el jugador.
\item Se actualiza la lógica del juego.
\item Se pinta en pantalla una representación del estado del juego.
\item Se repite desde el paso hasta que termine el juego.
\end{enumerate}

\newpage

Suponiendo que vamos a pintar un jugador que se mueve hacia la derecha cuando se presiona la tecla con la misma dirección, un posible \emph{bosquejo} en Python podría verse como el código \ref{cod-pygame-gameloop}. 


\lstinputlisting[language=Python,caption=Código base para el motor.,label=cod-pygame-gameloop,basicstyle=\fontsize{9}{9}\selectfont\ttfamily]{./py/pygame/gameloop.py}


\section{Introducción a PyGame}

\subsection{Instalación}

Para instalar PyGame usaremos una herramienta que permite descargar librerías para Python, \emph{pip}. Ya debería estar instalado si se descargó Python de \url{https://python.org}, pero en caso de tener algún problema puede consultarse el siguiente enlace \url{https://pip.pypa.io/en/stable/installing/}.

Bastará con abrir una consola del sistema operativo (FALTA: Añadir explicación VS Code), y ejecutar el siguiente comando:

\begin{lstlisting}
pip install pygame
\end{lstlisting}

\subsection{Instrucciones básicas}

Lo primero, será aprender cómo crear una ventana básica en PyGame. Para tal fin, es necesario importar la librería en nuestro código y usar las siguientes instrucciones:

\begin{enumerate}
\item \textbf{pygame.init()}: que sirve para arrancar los módulos importados de PyGame. \url{https://devdocs.io/pygame/ref/pygame#pygame.init}

\item \textbf{pygame.display.set\_mode(resolucion, flags)}: que realiza la configuración de la pantalla, siendo resolución una tupla donde determinamos el tamaño de la misma y, opcionalmente, se puede enviar pygame.FULLSCREEN para el modo pantalla completa (para salir de la pantalla completa en Windows o Linux presiona Alt+F4, en Mac Command+W). En el siguiente están todas las opciones de configuración permitidas: \url{https://devdocs.io/pygame/ref/display#pygame.display.set_mode}.
\end{enumerate}

El código básico para comenzar a crear un juego podría ser algo así:

\lstinputlisting[language=Python,caption=Código base para el motor.,label=cod-pygame-gameloop,basicstyle=\fontsize{9}{9}\selectfont\ttfamily]{./py/pygame/iniciar.py}

\begin{figure}[h!]
	\centering
	\includegraphics[width=6cm]{./Images/pygame-pantalla.png}
	\caption{La pantalla en PyGame.}
	\label{pygame-pantalla}
\end{figure}

Para pintar en pantalla, es necesario tener en cuenta que para PyGame la esquina superior izquierda de la pantalla es la coordenada $(0,0)$, y que la componente en $X$ crece hacia la derecha, mientras que la componente en $Y$ aumenta a medida que se acerca más al límite inferior de la pantalla, como se ve en la figura \ref{pygame-pantalla}.

Dicho esto, en nuestro caso vamos a utilizar la función para pintar rectángulos \textbf{pygame.draw.rect}. Sin embargo, para mayor facilidad dicho llamado lo abstraeremos dentro de una función propia que nos sea más fácil de recordar y utilizar, la llamaremos \textbf{dibujar\_rectangulo} y recibirá los siguientes parámetros:

\begin{itemize}
	\item La coordenada en X de la pantalla donde se va a pintar el rectángulo.
	\item La coordenada en Y de la pantalla donde se va a pintar el rectángulo.
	\item El ancho del rectángulo.
	\item El alto del rectángulo.
	\item El color en RGB (Red-rojo, Green-verde, Blue-azul), representado como una tupla de 3 valores, cada uno de ellos entre 0 y 255. Por ejemplo, los siguientes son algunos colores válidos:
	\begin{itemize}
		\item (0, 0, 0): negro.
		\item (255, 0, 0): rojo.
		\item (0, 255, 0): verde.
		\item (0, 0, 255): azul.
		\item (255, 255, 255): blanco.
	\end{itemize}
\end{itemize}

En el código \ref{cod-pygame-pintar} se relaciona la función que usaremos, junto con un ejemplo de como llamarla.

\lstinputlisting[language=Python,caption=La función que usaremos para el pintado.,label=cod-pygame-pintar,basicstyle=\fontsize{9}{9}\selectfont\ttfamily]{./py/pygame/pintar.py}

Finalmente, para los eventos, es necesario recorrer un arreglo que se obtiene al llamar a \textbf{pygame.event.get()}, dicha función retorna, entre otras cosas, las teclas presionadas y si el jugador ha cerrado la ventana del juego. Para facilitar también las cosas en este punto, vamos a crear una función llamada \textbf{actualizar\_eventos}, así:

\begin{itemize}
\item La función nos retorna -1 si terminó el juego, 0 en cualquier otro caso.
\item De paso, la función va a mapear las teclas presionadas a variables globales, desde la que luego podremos saber qué estaba pulsando el usuario.
\end{itemize} 

En el código \ref{cod-pygame-eventos} podemos ver la función, cómo mapear las teclas arriba, abajo, izquierda y derecha (para ver todas las demás opciones está este link \url{https://devdocs.io/pygame/ref/key}). Al final del código está un posible ejemplo sobre cómo utilizar todo esto.

\lstinputlisting[language=Python,caption=Los eventos en PyGame y cómo integrarlos en el gameloop.,label=cod-pygame-eventos,basicstyle=\fontsize{9}{9}\selectfont\ttfamily]{./py/pygame/eventos.py}

\subsection{Ejemplo completo}

A continuación, tenemos un caso completo que permite pintar un rectágulo rojo en la pantalla y moverlo con las teclas direccionales. 

Nótese que se han agregado un par de líneas con un \emph{reloj}, lo anterior para que el juego no vaya a ir más rápido en unos computadores que en otros. La tasa de refresco máxima la hemos fijado en 60 cuadros por segundo (FPS - Frames per second).

\lstinputlisting[language=Python,caption=Código base para el motor.,label=cod-motor-3d-pygame,basicstyle=\fontsize{9}{9}\selectfont\ttfamily]{./py/pygame/completo.py}
