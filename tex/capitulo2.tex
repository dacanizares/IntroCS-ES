\chapter{Nociones básicas}

En este capítulo se exponen la mayoría de conceptos involucrados en la programación de ordenadores a nivel básico. No se pretende que el lector domine todos los conceptos con sólo ver este capítulo, al contrario, la idea es dar un vistazo general sobre lo que utilizaremos y profundizaremos a lo largo del libro, de forma que, teniendo una imagen más completa de la programación los temas a tratar no parezcan obstáculos sin sentido, que sólo tienden a ralentizar el proceso de aprendizaje. 

Lo primero será comprender cómo las instrucciones de un programa operan directamente en la máquina, para esto introduciremos un modelo abstracto que representará nuestro computador: la Máquina de Acceso Aleatorio (RAM, por sus siglas en inglés).

\section{Máquina RAM}

La idea es mantener el modelo lo más simple posible, razón por la cuál sólo consideraremos los siguientes componentes:

\begin{itemize}
\item La memoria: representa el estado de la máquina en un momento del tiempo. Está dividida en múltiples secciones que llamamos posiciones de memoria, cada posición tiene asociados un número (generalmente un hexadecimal) que permite distinguir las posiciones entre sí y un valor, que representa el dato almacenado.

\item La entrada: son los datos sobre los cuáles se va a operar. Puede ser una pulsación del teclado, un clic, un dato que viaja por Internet, la información de un sensor, etc.
 
\item El programa: es una secuencia de instrucciones que le dice a la máquina cómo modificar su memoria de acuerdo a la entrada que recibe y al estado mismo de la memoria. 

\item La salida: es el resultado que se obtiene después de ejecutar total o parcialmente el programa.
\end{itemize}


\begin{figure}[h!]
\centering
  \begin{tikzpicture}
  \coordinate (programa) at (2,1);
  \coordinate (pa) at (0,2);   
  \coordinate (pb) at (2,2);
  \coordinate (pc) at (2,0);
  \coordinate (pd) at (0,0);   
  
  \coordinate (memoria) at (8,1);
  \coordinate (ma) at (8,1.25);   
  \coordinate (mb) at (10,1.25);
  \coordinate (mc) at (10,0);
  \coordinate (md) at (8,0);  
  \coordinate (mml) at (8,0.75); 
  \coordinate (mmr) at (10,0.75); 
  
  
  % programa
  \draw (pa) -- (pb) node[above, midway] {Programa};
  \draw (pb) -- (pc);
  \draw (pc) -- (pd);
  \draw (pd) -- (pa);   
  % lineas codigo
  \foreach \y in {0.2, 0.4,...,1.8}
  \draw (0.2, \y) -- (1.8, \y);  
  
  %memoria
  \draw (ma) -- (mb) node[above, midway] {Memoria};
  %\draw (mb) -- (mc);
  %\draw (mc) -- (md);
  %\draw (md) -- (ma); 
  %memset
  \draw (mml) -- (mmr); 
  % lineas mem
  \foreach \x in {8, 8.4,...,10}   	
  \draw (\x,0.75) -- (\x,1.25);  
  
  %ram
  \node [right=1cm of programa] [box] (ram) {RAM};  
  \node [above=1cm of ram] (salida) {Salida};
  \node [below=1cm of ram] (entrada) {Entrada}; 
  
  % arrows  	
  \draw [arrow] (ram.north) -- node[] {} (salida.south); 	\draw [arrow] (entrada.north) -- node[] {} (ram.south);
  \draw [arrow] (programa.east) -- node[] {} (ram.west);
  \draw [arrow] (memoria.west) -- node[] {} (ram.east); 	
  \draw [arrow] (ram.east) -- node[] {} (memoria.west);  
  \end{tikzpicture}

\caption{Máquina RAM}
\label{figram}
\end{figure}

De una u otra forma, el programa tendrá que interactuar con todos los componentes de la máquina, luego, los lenguajes de programación deben proveer mecanismos para manipular y representar la memoria, la entrada y la salida de datos. Cabe anotar que, como el programa depende del estado actual de la máquina, también se debería contar con estructuras que permitan repetir o ejecutar ciertas instrucciones de acuerdo a los datos almacenados en memoria. Por ejemplo, se deben ejecutar algunas instrucciones si la máquina entra o en un estado de error o se deben repetir algunas instrucciones para realizar una multiplicación (similar a como lo hacemos en papel).

Ahora, la pregunta es: ¿Cómo hace todo esto Python en un computador real? Después de muchos intentos con distintas explicaciones, creo que la mejor forma de responder esta pregunta es mediante ejemplos reales en el lenguaje de programación. Se recomienda encarecidamente al lector, que pruebe todos los programas en su máquina, para que los vea en funcionamiento. De igual forma, nada de esto tendrá sentido si el lector pasa rápidamente sobre este libro y no se toma el tiempo para entender qué hace cada instrucción.

\section{Programación en Python}

Aunque en un computador todo termina siendo unos y ceros, Python tiene algunas abstracciones básicas (otras un poco más avanzadas, que veremos en el capítulos posteriores) que harán todo más fácil. Una abstracción permite manipular la máquina sin necesidad de conocer a fondo su funcionamiento. Por ejemplo, el pedal de un carro permite que usted acelere sin necesidad de conocer el funcionamiento del motor. Si hablamos de la memoria, en Python contamos con diferentes \textbf{tipos de datos} que determinan las acciones podemos realizar con los datos almacenados en memoria, sin necesidad de conocer a profundidad qué hace el computador con ellos. A saber:

\begin{itemize}
\item Números: con los números podemos efectuar operaciones aritméticas como sumas (+), restas (-), multiplicaciones (*), divisiones(/), residuos(\%) y potencias (**). En Python, los enteros se llaman \textbf{int} (de \emph{integer}, en inglés) y los que tienen decimales se conocen como \textbf{float}. Por ejemplo: 0, -1, 1, 40, 52, son enteros y 0.1, 0.7, -1.8 son decimales.

\item Valores lógicos: en Python se conocen como \textbf{bool} (booleanos) y sólo tienen dos valores posibles: verdadero (\textbf{True}) o falso (\textbf{False}). Podemos aplicar los operadores Y (\textbf{and}), O(\textbf{or}), Negación (\textbf{not}), entre otros, para componer proposiciones más complejas.  Ver código \ref{cod-asignacion}.

\item Texto: se conocen como \textbf{strings} o secuencias de caracteres. Empiezan por una comilla doble o sencilla y terminan por el mismo símbolo, esto, con el fin de distinguir los textos arbitrarios definidos por el programador de las instrucciones propias del programa. Ver código \ref{cod-asignacion}.
\end{itemize}

Python no sólo facilita la representación de los datos en memoria, sino también su almacenamiento y manipulación. Para esto, existe el concepto de \textbf{variables}, que permiten dar un nombre arbitrario \footnote{Sólo hay tres restricciones: usar únicamente letras, números o guiones bajos, no empezar por un número y no dejar espacios en blanco} y un valor a partes de la memoria solamente utilizando el operador de igualdad (=), como se ve en el código \ref{cod-asignacion}. Python detecta automáticamente el tipo dato que se le asigna a una variable. \\

\newpage

\lstinputlisting[language=Python,caption=Asignaciones en Python,label=cod-asignacion]{./py/nociones/asignacion.py}

Si ejecutamos ese código, el computador no mostrará nada, eso es porque aún no hemos definido instrucciones para la salida de datos. La forma básica de mostrar información en la pantalla es usando la instrucción \emph{print}. Basta con escribirla y seguido de un espacio colocar el valor, la operación o la variable que se desea mostrar. En Python podemos comentar el código, añadiendo el caracter \#, de forma que podamos escribir aclaraciones del programa, sin afectar su comportamiento. Todo lo anterior se ve en el código \ref{cod-salida}. \\


\lstinputlisting[language=Python,caption=Salida de datos en Python,label=cod-salida]{./py/nociones/salida.py}

Por otro, la entrada de datos básica se hace con la instrucción \emph{input()}; entre los paréntesis se puede colocar un mensaje para aparezca a la persona que ejecuta el programa. Luego de ingresar los caracteres por teclado y presionar la tecla \emph{enter}, los datos quedan almacenados en la variable a la cuál se haya asignado la instrucción. Por ejemplo, el código \ref{cod-suma} muestra como solicitar dos números por teclado para luego imprimir el valor que resulta al sumarlos \footnote{Como lo importante es aprender, por ahora confiaremos en que el usuario no cometerá errores al ingresar los datos. He visto códigos que complican al estudiante con mil y una validaciones, bajo la excusa de que eso no se puede omitir en un programa comercial. Eso no tiene sentido, al fin y al cabo, cuando se realizan programas grandes, se usan otras herramientas que ayudan a prevenir estos errores.}. \\

\newpage

\lstinputlisting[language=Python,caption=Lectura y suma de dos números en Python,label=cod-suma]{./py/nociones/suma.py}

En Python también es posible condicionar la ejecución de ciertas instrucciones utilizando la expresión \emph{if}, que traduce \emph{si} en español. Dicha expresión, debe ir acompañada de una \emph{condición}, que no es más que la evaluación, \emph{verdadera} o \emph{falsa}, de una proposición utilizando los operadores lógicos: es igual (==) \footnote{Nótese que se usa doble igual para distinguir la comparación de la asignación.}, mayor ($>$), menor($<$), mayor o igual ($>=$), menor o igual ($<=$) o diferente (!=). Por ejemplo, el código \ref{cod-if} sólo ejecuta la instrucción que imprime \emph{Es cero}, si  el valor ingresado es igual a 0. Las instrucciones que están condicionadas \textbf{deben ir tabuladas}, como se observa en el código de ejemplo, y pueden ser una o más instrucciones de cualquier tipo. \\

\lstinputlisting[language=Python,caption=Un \emph{if} en Python,label=cod-if]{./py/nociones/if.py}


La instrucción \emph{if} puede ir acompañada de un \emph{else}, que traduce \emph{si no} en español, y que se utiliza para indicar qué instrucciones ejecutar cuando la condición no se cumple. Esta situación se puede observar en el código \ref{cod-ifelse}, en el cuál se reciben dos números y se imprime cuál de ellos es el mayor (en caso de ser iguales, no importará cuál mostremos). Nótese que el texto entre comillas se muestra tal cuál en patalla, mientras que el valor después de la coma se reemplaza por el valor que tenga la variable en el momento de la ejecución. \\

\lstinputlisting[language=Python,caption=Un \emph{if-else} en Python,label=cod-ifelse]{./py/nociones/ifelse.py}

Si se necesitan condicionar más caminos, se puede utilizar la expresión \emph{elif}, que es una abreviación de \emph{else if}. Por ejemplo, si quisiéramos saber cuál es el mayor de tres números (llámense $a$, $b$ y $c$), habría tres opciones: 
\begin{itemize}
\item Que $a$ sea mayor que $b$ y que $c$, en tal caso, $a$ sería el mayor.

\item Si no, se debe descartar entre $b$ y $c$ cuál es mayor, luego, si $b$ es mayor que $c$, $b$ es el mayor.

\item Si ninguna de las anteriores condiciones se cumple, es porque el mayor era $c$. 
\end{itemize}


Lo expresado anteriormente se observa en el código \ref{cod-ifelifelse}.  \\

\lstinputlisting[language=Python,caption=El mayor de tres números,label=cod-ifelifelse]{./py/nociones/ifelifelse.py}

En cuanto a la repetición de instrucciones, se utiliza la instrucción \emph{for}, que traduce \emph{para} en español. Veamos como funciona: \\

\lstinputlisting[language=Python,caption=Repetición de instrucciones en Python,label=cod-for]{./py/nociones/for.py}

El código anterior imprime los números desde el cero hasta el nueve (como puede ver, el límite superior no se incluye). Su lectura en español nos ayudará a entenderlo: \emph{para i en el rango de 0 a 10, imprima i}. La $i$ es una variable cualquiera, por lo cuál podemos utilizar otro nombre; es común usar $i$ por su relación con los índices y subíndices en matemáticas. Los dos números dentro del paréntesis de \emph{range} indican desde que valor inicia la repetición y en cuál termina. Al igual que con la estructura \emph{if}, las instrucciones que se repiten deben ir tabuladas.

Es posible mezclar todas las instrucciones que hemos visto (y las que veremos), por ejemplo, tener programas como el del código \ref{cod-forif} que recibe dos números, imprime todos los enteros que hay entre ellos, indicando si son negativos o positivos.


\lstinputlisting[language=Python,caption=Repeticiones y decisiones en Python,label=cod-forif]{./py/nociones/forif.py}

Hasta aquí hemos visto las instrucciones básicas usaremos para controlar todos los componentes de nuestra máquina RAM. Sin embargo en los ejemplos propuesto, no se modificó demasiado la memoria del equipo. A continuación hablaremos sobre algunos mecanismos propios de la programación imperativa que permiten resolver problemas mucho más complejos.



\section{Pensamiento imperativo}

Los algoritmos, al contrario de lo que parece, no son complicados para la mayoría de personas. Por ejemplo las multiplicaciones, divisiones, sumas y restas que hacemos en el papel son algoritmos que aprendimos por allá en la escuela primaria. Estamos en capacidad de enseñar a otras personas las instrucciones necesarias para realizar dichas operaciones; sin embargo, cuando se trata de explicarle a una máquina, las restricciones propias de memoria y procesamiento, hacen que el proceso sea un poco menos sencillo.

Muchas personas, sin saberlo, realizan pequeños programas de computador usando hojas de cálculo, determinando qué campos operar y dónde almacenar los resultados de dichas operaciones. La idea de los algoritmos que vamos a escribir es similar, la diferencia es que \emph{los datos no son estáticos y una posición de memoria cambiará muchas veces durante la ejecución del programa}, por lo cuál debemos abstraernos un poco más, de forma que podamos \emph{comprender el significado de nuestras instrucciones en el tiempo}.

Mediante unos problemas, exploraremos las soluciones típicas que se pueden dar a los problemas usando \emph{programación imperativa}. Como podrá ver el lector, en ocasiones es posible encontrar repuestas matemáticas más cortas y rápidas, pero en algunos casos no es posible este tipo de aproximaciones. Dicho esto, ¡Es hora de programar!

\subsection{Las sumas de Gauss}

\begin{minipage}{.7\textwidth} 
Cuando Gauss estaba en la escuela una profesora quería mantener ocupados a sus estudiantes, como no preparó clase, les puso la aburrida tarea de sumar los números enteros entre 1 y 100. La sorpresa de la maestra fue mayúscula cuando Gauss exclamó, después de muy poco tiempo: ¡5050! Efectivamente era la respuesta correcta. La pregunta es, ¿cómo lo hizo Gauss? A continuación construiremos la respuesta.
\end{minipage}
\begin{minipage}{.30\textwidth}
  \centering
  \includegraphics[height=4cm]{./Images/gauss.jpg}
\end{minipage}

\subsubsection{Solución imperativa}
Si Gauss hubiese tenido un computador a la mano, seguro habría escrito un programa en Python, cómo vimos, es posible repetir instrucciones de forma muy simple, utilizando la instrucción \emph{for}. Ahora, la pregunta es ¿cómo repetimos las instrucciones, de manera que al finalizar, el estado de la máquina tenga la respuesta a este problema?

La respuesta es corta, pero no muy intuitiva, en especial si esta es la primera vez que usted deja el mundo del papel y lápiz, por el de los bits. Así que, antes de continuar, es recomendable que se tome un par de minutos y escriba un programa de computador, con las instrucciones que usted cree, darán la respuesta. No importa si al final el código tiene errores, poner a trabajar las neuronas es más importante que llegar a la solución al primer intento. Una vez tenga su respuesta, continúe leyendo.
\\

Ya que no tenemos una hoja de papel para hacer un montón de operaciones, sólo posiciones de memoria que guardan bits (representadas en Python por el concepto de variables), necesitamos pensar muy bien como expresar las instrucciones. Hagámonos la siguiente pregunta, ¿será necesario almacenar en una variable diferente cada operación realizada? La respuesta es no. Imagine que usted tiene solo un pequeño trozo de papel, ¿cómo lo utilizaría para sumar los números del 1 al 100? Usted podría realizar operaciones y a medida que avanza ir borrando lo que ya no necesita guardarlas todas. Por ejemplo podría:

\begin{itemize}
\item Sumar 1 + 2 = 3. Como ya sabe la respuesta, puede borrar la operación y sólo conservar en su mente el número 3.
\item Luego, hay que sumar 3 + 3 = 6. De igual forma, sólo es necesario conservar el 6 presente, el resto se puede borrar.
\item La siguiente suma en el trozo de papel sería 6 + 4 = 10. Siguiendo el mismo órden de ideas, pasamos a la siguiente suma.
\item 10 + 5 = 15
\item 15 + 6 = 21
\item (...)
\item Y repetir, en el mismo pedazo de papel, hasta sumar los 100 números. 
\end{itemize}

En el computador la idea es la misma. No necesitamos 100 variables, usaremos una sola variable que iremos cambiando a medida que avance el tiempo, de forma que se acumule la suma. Ahora, si observamos cuidadosamente el proceso anterior, podemos encontrar que estamos sumando el valor que vamos acumulando, con cada uno de los 100 números. Luego, si damos el nombre de $suma$ a la variable en la que acumulamos el resultado, y le damos el nombre de $i$ a cada uno de los 100 número, la operación en el computador sería: 

\begin{equation}
suma = suma + i
\end{equation}

Puede revisar el código \ref{cod-for} para mayor claridad sobre el tratamiento que tendremos con la variable $i$.

Si metemos la expresión dentro de un \emph{for}, ¿funcionará? de hecho no. Notemos que el computador acumula haciendo $suma$ igual a lo que tenía más $i$. Y entonces, ¿qué tiene $suma$ la primera vez? Nada, pero hay que decírselo al computador iniciándola en cero. Al final, nuestro programa será como el que se observa en el código \ref{cod-gauss}. \\

\lstinputlisting[language=Python,caption=La suma de Gauss en Python,label=cod-gauss]{./py/nociones/gauss.py}

\subsubsection{Ejecución en la máquina RAM}

Vamos a ver cómo se modifica la memoria de la máquina hasta dar con la respuesta. La principal razón para no repetir este tipo de cosas es que \textbf{el programador debe aprender a leer el código por lo que significa y no por la forma en que la máquina lo ejecuta}; de otra manera nada ganaríamos con tener procesadores cada vez más rápidos, si a la hora de programar tenemos que repasar como ejecuta el computador las miles de iteraciones de nuestros programas. Por otro lado, el lector debe tener completamente claro lo que acabamos de programar, para que el comportamiento del resto de programas que veremos le resulte mucho más intuitivo.

Hecha esta aclaración, hagamos el seguimiento de la ejecución del programa:

\begin{itemize}
\item Se inicia la variable $suma$ en 0.
\item Se entra al ciclo en cuál la $i$ arranca en 1 y termina en 100.
	\begin{itemize}
	\item En la primera iteración $suma=0+1$, luego, $suma = 1$
	\item En la segunda iteración $suma=1+2$, luego, $suma = 3$
	\item En la tercera iteración $suma=3+3$, luego, $suma = 6$
	\item En la cuarta iteración $suma=6+4$, luego, $suma = 10$
	\item En la quinta iteración $suma=10+5$, luego, $suma = 15$
	\item ...
	\item En la penúltima iteración $suma=4851+99$, luego, $suma = 4950$
	\item En la última iteración $suma=4950+100$, luego, $suma = 50501$
	\end{itemize}
\item Cuando termina el ciclo se imprime el resultado $5050$
\end{itemize}

Como usted puede ver, el análisis previo es exactamente lo que hace el programa. Al principio puede resultar un poco difícil, pero usted tiene que ganar confianza como programador para poder enfrentar cada vez retos más difíciles y apasionantes.

\begin{quote}
\emph{La visualización (...) remarcaba precisamente aquello que el estudiante tiene que aprender a ignorar, reforzaba precisamente lo que el estudiante tiene que desaprender}.
Profesor Edsger Dijsktra, en su ensayo sobre la crueldad de enseñar realmente ciencias de la computación.
\end{quote}

\subsubsection{Solución matemática}

Aunque nuestro programa resultó funcionar muy bien, es claro que Gauss no pudo contar un computador en su época. Así que veamos cómo lo hizo.

Gauss descubrió una fórmula que permite hallar fácilmente la sumatoria de los $n$ primeros números:

\begin{equation}
1+2+3+\cdots+n = \frac{n(n+1)}{2}
\end{equation}

Si reemplazamos la $n$ por 100 tenemos lo siguiente:

\begin{equation}
1+2+3+\cdots+100 = \frac{100(100+1)}{2}
\end{equation}

Que efectivamente es 5050. ¿Pero cómo intuyó esto Gauss? La respuesta la encontramos si organizamos los números como se ve en la figura \ref{figgauss}.

\begin{figure}[h!]
\centering
\begin{tabular}{| c | c | c | c | c | c | c |}
\hline
1 & 2 & 3 & $\cdots$ & (n-2) & (n+1) & n \\ \hline
n & (n-1) & (n-2) & $\cdots$ & 3 & 2 & 1 \\ \hline
\end{tabular}
\caption{Suma de los primeros $n$ números}
\label{figgauss}
\end{figure}

Es decir, tomamos la serie de forma ascendente y luego de forma descendente. Ahora ¿qué pasa si sumamos verticalmente? En la primera obtenemos $n+1$, en la segunda $2+(n-1)=n+1$, y si seguimos, veremos que en todas obtenemos $n+1$. Como teníamos $n$ números, tenemos $n$ veces $n+1$ que es lo mismo que $n \cdot (n+1)$. Finalmente, la respuesta la dividimos entre dos, porque tomamos la serie dos veces, llegando a la respuesta \cite{graham1994concrete}.

La idea de sumar dos veces la serie resultó bastante útil, y nuestro programa se podría reducir a una línea -imprimir la operación. Las soluciones matemáticas suelen ser mucho más elegantes, rápidas y poderosas que las soluciones a código puro y duro, sin embargo no todo puede resolverse en el computador con una fórmula tan sencilla. Generalmente, se deben mezclar ambos enfoques para resolver problemas más complejos.



\subsubsection{Del papel al computador}

Es necesario aclarar, que para pasar la fórmula anterior al computador hay que tener especial cuidado en el orden en el que la máquina ejecuta las operaciones. A saber:

	\begin{enumerate}
	\item Se resuelve lo que esté dentro de paréntesis \textbf{()}.
	\item Se resuelven las potencias \textbf{**}.
	\item Se resuelven las multiplicaciones, divisiones y residuos  \textbf{* / \%}. 
	\item Se resuelven las sumas y restas  \textbf{+} y \textbf{-}.
	\end{enumerate}
	
Así que, para que el resultado sea el esperado, es necesario utilizar signos de agrupación como se ve en el código \ref{cod-gauss-formula}. \\


\lstinputlisting[language=Python,caption=Las sumas de Gauss.,label=cod-gauss-formula]{./py/nociones/gauss-formula.py}

\subsection{Los números primos}

Los números primos son muy importantes en la computación, ya que la criptografía actual esta basada principalmente en ellos. Sin embargo, los primos han fascinado a la gente desde tiempos inmemoriales. Por ejemplo, los griegos demostraron que existen infinitos números primeros \cite{discretemath}. Para entender qué son estos números, debemos definir el concepto de \emph{divisivilidad}.

\subsubsection{Divisibilidad}

Sean $a$ y $b$ dos enteros cualquiera. Decimos que $a$ es divisible entre $b$ si al realizar la división $a/b$ el residuo es cero.

Por ejemplo, 4 es divisible entre 2, porque al realizar la división 4/2 el residuo es cero; 5 no es divisible entre 2, porque al realizar la división 5/2 el residuo es uno.

Para hallar el residuo de una división en el computador, utilizamos el operador módulo (\%), como se observa en el código \ref{cod-mod}. \\

\lstinputlisting[language=Python,caption=Hallando el módulo o residuo de una división.,label=cod-mod]{./py/nociones/codmod.py}



\subsubsection{Números primos}

Un número entero $p$ mayor que uno, es  primo si solamente es divisible por $1$ y por $p$.  

Por ejemplo, 2 es primo porque sólo lo divisible entre 1 y 2, pero 4 no es primo, porque es divisible entre 1, 2 y 4. Los siguientes son los primeros 100 números primos:

2 3 5 7 11 13 17 19 23 29 31 37 41 43 47 53 59 61 67 71 73 79 83 89 97 101 103 107 109 113 127 131 137 139 149 151 157 163 167 173 179 181 191 193 197 199 211 223 227 229 233 239 241 251 257 263 269 271 277 281 283 293 307 311 313 317 331 337 347 349 353 359 367 373 379 383 389 397 401 409 419 421 431 433 439 443 449 457 461 463 467 479 487 491 499 503 509 521 523 541 

Conociendo los conceptos matemáticos, nuestra misión será crear un programa de computador que determine la cantidad de divisores de un número y posteriormente indique si es primo o no (en el próximo capítulo veremos cómo podemos hacer está última tarea más rápidamente).

\subsubsection{Solución}

La idea es simple: nos dan un número $n$, contamos cuántos divisores tiene, si al final tiene sólo dos divisores (1 y $n$) entonces es primo.

Si recordamos, todos los números son divisibles entre uno y entre el número mismo. Así que estas dos verificaciones son triviales y podemos omitirlas. Tenemos que centrarnos en determinar si algún otro número es divisor. Pero la pregunta es, ¿cuántos números diferentes debemos chequear? hay infinitos números, y nuestro programa no terminaría nunca de revisarlos todos. De hecho, en algún punto se bloquearía, porque la memoria de la máquina es finita. 

Luego, tenemos que determinar con cuidado qué números necesitamos verificar. Eso es algo fácil si lo pensamos así: ningún número es divisible por uno mayor que él. Por ejemplo, 3 no es divisible entre 5 (el 3 en el 5 está cero veces y sobran 3), dicho de otra forma, es imposible dividir el 3 en 5 partes enteras.

Dicho esto, bastará con \textbf{contar} qué números mayores que uno, pero menores que $n$, son divisores del número. Para contar en un programa de computador, se utiliza un mecanismo similar al del ejercicio anterior: definimos una variable, la iniciamos en algún valor (usualmente cero) y cada vez que se cumpla cierta condición le sumamos +1. Es como tener un papel en el que realizamos una raya cada vez que queremos contar algo.

Para nuestro caso, esa variable la llamaremos $divisores$, la iniciaremos en 2 (recuerde que todos los números tienen de entrada al uno y a él mismo como divisores) y cada vez que encontremos un divisor ejecutaremos la instrucción:
\begin{equation}
divisores = divisores + 1
\end{equation}

En el código \ref{cod-primos} puede ver el programa completo. Sólo falta agregar, que al final es necesario verificar que el número no sea uno, ya que por convención el uno no es primo. \\

\newpage

 \lstinputlisting[language=Python,caption=Divisibilidad y números primos.,label=cod-primos]{./py/nociones/primos.py}



\subsection{Las letras}


Supongamos que van a ingresarnos $n$ palabras. La idea es contar cuántas palabras consecutivas son iguales. Por ejemplo:

\begin{itemize}
	\item Si $n$ es 5, y letras, ingresadas en orden son:
	\item A
	\item A
	\item B
	\item A
	\item A
	\item La respuesta es dos. Porque se ingresaron dos As consecutivas.
\end{itemize}



\subsubsection{Solución}

Para este problema sólo basta tener en cuenta si la nueva letra es diferente al anterior, en tal caso se incrementa el número de grupos.

Para programarlo, debemos crear un espacio en memoria en el que podamos hacer seguimiento a la letra \emph{anterior}. Al terminar de leer cada letra y mirar si se forma un grupo, debe actualizarse el anterior, para efectos de tener los datos preparados para la siguiente iteración.

La única parte que es menos intuitiva, es la forma en que haremos que al poner al principio la variable \emph{anterior}. Todo esto se puede ver en el código \ref{cod-magnets}. \\

 \lstinputlisting[language=Python,caption=Solución del problema letras.,label=cod-magnets]{./py/nociones/letras.py}
 
 
\newpage
\section{Problemas}
En este capítulo hemos aprendido lo básico de la programación y de las soluciones imperativa: cómo sumar elementos, cómo contar elementos y vigilar un elemento anterior. \\

Mediante los problemas propuestos usted podrá auto-evaluarse y poner a prueba sus conocimientos. Para saber si su solución está correcta, piense en algunas entradas y las salidas respectivas a cada una de ellas, luego ejecute su programa y valide si el resultado fue correcto. Si no es así, piense ¿qué puede estar fallando?

\vfill

\begin{Exercise}[title={Calentamiento}]
Realice un programa en Python para resolver los siguientes problemas:
\begin{enumerate}
\item Se va a recibir un número, se garantiza que será 0 o 1. En caso de recibir un 0 se debe imprimir 1, en caso de recibir 1 se debe imprimir 0. No se puede utilizar la instrucción \emph{if}.

\item ¿Cómo realizar una multiplicación sin usar el signo por (*)? Encuentre dos soluciones diferentes para éste problema.

\item ¿Cómo realizar una potencia sin usar el signo de potenciación (**)?
\end{enumerate}
\end{Exercise}
\begin{Answer}

\lstinputlisting[language=Python,caption=Solución de los problemas de calentamiento.,label=cod-calentamiento]{./py/nociones/calentamiento.py}
\newpage
\end{Answer}
\vfill
\begin{Exercise}[title={La Copia de Seguridad}]
Bill quiere guardar sus documentos que pesan $x$ gigas en unos discos que tienen $q$ gigas de capacidad. Por ejemplo, si tiene que guardar 6 gigas en discos de 4 gigas, necesitará 2 discos (aunque el segundo lo llene sólo parcialmente).

¿Podrá usted ayudar a Bill construyendo un programa para que él sepa cuántos discos debe comprar? Lea por teclado las variables $x$ y $q$, para calcular la respuesta.

Nota: recuerde que los códigos deben funcionar para cualquier entrada, no sólo para las que aquí se proponen. Asuma de momento que siempre le van a dar datos válidos.

\end{Exercise}

\newpage
\begin{Answer}
 \lstinputlisting[language=Python,caption=Solución del problema La Copia de Seguridad.,label=cod-seguridad]{./py/nociones/copiaseguridad.py}
 
\end{Answer}

\begin{Exercise}[title={El Precio del Videojuego}]
	Joe Cortana tiene una tienda de empeños para videojugadores. Todos los días entran a su tienda cientos de vendedores que quieren cambiar sus antiguos videojuegos por dinero. 
	
	Joe es muy cuidadoso y ha estudiado muy bien como se comporta el mercado. Ha descubierto que en las mañanas llegan los mejores juegos, de esos clásicos que son muy apetecidos por los coleccionistas. A medida que pasan las horas, cada vez recibe juegos más malos, terminando en ET (considerado el peor juego de la historia).
	
	Joe Cortana tiene disponibles $x$ dólares todos los días. Su misión es decirle, dada una lista de $n$ videojuegos con sus precios (todo esto se debe leer por teclado), cuántos títulos alcanza a comprar, de forma que esté adquiriendo siempre lo mejor que vaya entrando a su tienda.	
	
	Por ejemplo: si tiene 10 dólares, y van a ofrecerle 3 juegos que valen respectivamente 6, 4 y 3 dólares, él compra dos juegos.
	
	\textbf{Explicación}: 
	\begin{itemize}
		\item El primero vale 6, y lo compra, quedándole 4 dólares disponibles.
		
		\item El segundo vale 4 dólares, de forma que lo compra y no le queda nada.
		
		\item El tercero vale sólo un dólar, pero ya no tiene dinero. 
		
		\item En definitiva, alcanza a comprar 2 juegos.
	\end{itemize}
	
	Nota: recuerde que los códigos deben funcionar para cualquier entrada, no sólo para las que aquí se proponen. Asuma de momento que siempre le van a dar datos válidos.
	
\end{Exercise}
\begin{Answer}
\lstinputlisting[language=Python,caption=Solución el Precio del Videojuego.,label=cod-vjprecio]{./py/nociones/vjprecio.py}
\newpage
\end{Answer}


\newpage
